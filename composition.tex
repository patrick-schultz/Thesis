% -*- root: thesis.tex -*-

\chapter{Composition of $\mathbb{L}$-coalgebras}

For this section, we will continue to let $\mathbb{D}=\Sq(\mathcal{D})$ be the double category of squares in a 2-category $\mathcal{D}$.

In an algebraic weak factorization system, the categories $\LCoalg$ and $\RAlg$ respectively play the roles of the left and right classes of morphisms of the weak factorization system. In an ordinary weak factorization system, these two classes of morphisms are closed under composition. In \cite{garner:soa}, this is strengthened to a composition functor
\[
	\LCoalg \prodb_C \LCoalg \to \LCoalg
\]
and in \cite{riehl:nwfs-model}, it is shown that colax morphisms of awfs preserve this composition. Similarly, there is a composition functor on $\RAlg$ which is preserved by lax morphisms of awfs.

In this section, we will generalize these results to the setting of bimonads in $\FF{\mathbb{D}}$.

First, recall from \cite{street:ftm} the following proposition.

\begin{proposition}\label{Prop:EMObject}
	Let $C$ be a category, and $\mathbb{L}=(L,\epsilon,\delta)$ be a comonad on $C$. The category of coalgebras $\LCoalg$ has a universal property as follows:
	\begin{itemize}
	 	\item There is a forgetful functor $U\colon \LCoalg \to C$ and a natural transformtion $\alpha\colon U \Rightarrow LU$, satisfying $\epsilon U\circ\alpha = \id_U$ and $\delta U \circ \alpha = L\alpha\circ\alpha$.
	 	\item $(U,\alpha)$ is universal among such pairs satisfying such equations. Given another such pair $(F,\beta)$, where $F\colon X\to C$, there exists a unique functor $\hat{F}\colon X\to \LCoalg$ such that $U\hat{F}=F$ and $\alpha\hat{F}=\beta$.
	 \end{itemize} 
\end{proposition}

For the rest of this section, assume that $\mathcal{D}$ \emph{has EM-objects for comonads}, i.e. for every comonad $\mathbb{L}$ in $\mathcal{D}$ there is an object $\LCoalg$ satisfying the universal property above.

The goal of this section will be to prove the following theorem:

\begin{theorem}
	There is a lax double-functor
	\[
		\Coalg\colon \DComon(\mathbb{F}\mathrm{F}(\Sq(\mathcal{D}))) \to \Span(\mathcal{D}_0)
	\] 
	where $\mathcal{D}_0$ is the ordinary category underlying the (strict) 2-category $\mathcal{D}$, which is the identity on the vertical categories, and which takes a comonad $(E,\eta,\epsilon,\delta)$ in $\mathbb{F}\mathrm{F}(\Sq(\mathcal{D}))$ to the span
	\[
	\begin{tikzcd}[column sep=large]
		C & \LCoalg \lar[swap]{\dom U} \rar{\cod U} & C.
	\end{tikzcd}
	\]
\end{theorem}

The corollary of most importance to us is

\begin{corollary}
	For any awfs $(E,\eta,\mu,\epsilon,\delta)$ on an object $C$ in $\mathcal{D}$, the multiplication $\mu$ induces a composition functor on $\LCoalg$, and the functor between EM-objects induced by any colax morphism of awfs preserves this composition.
\end{corollary}
\begin{proof}
	Any awfs $(E,\eta,\mu,\epsilon,\delta)$ has an underlying object in $\DComon(\mathbb{F}\mathrm{F}(\Sq(\mathcal{D})))$, by simply forgetting $\mu$. The lax double-functor $\mathrm{Coalg}$ takes this to a span
	\[
	\begin{tikzcd}[column sep=large]
		C & \LCoalg \lar[swap]{\dom U} \rar{\cod U} & C.
	\end{tikzcd}
	\]
	The multiplication $\mu$ provides this object in $\DComon(\mathbb{F}\mathrm{F}(\Sq(\mathcal{D})))$ with a monad structure, and lax double-functors preserve monads, so $\mu$ induces a monad structure on this span. A multiplication on this span is a morphism $m$:
	\[
	\begin{tikzcd}[row sep=small,sloped,pos=.1]
		{} & X \dlar[swap]{\dom UP_1} \drar{\cod UP_2} \ar{dd}[sloped=false,pos=.5]{m} & \\
		C && C \\
		& \LCoalg \ular{\dom U} \urar[swap]{\cod U}
	\end{tikzcd}
	\]
	where $X$ is the pullback in the composite span
	\[
	\begin{tikzcd}[row sep=small]
		{} && X \dlar[swap]{P_1} \drar{P_2} && \\
		& \LCoalg \dlar[swap,sloped,pos=.1]{\dom U} \drar[sloped,pos=.1]{\cod U}
			&& \LCoalg \dlar[swap,sloped,pos=.1]{\dom U} \drar[sloped,pos=.1]{\cod U} & \\
		C && C && C.
	\end{tikzcd}
	\]

	The morphism $m$ is the composition structure that we want. If $\mathcal{D}=\mathrm{Cat}$ is the 2-category of small categories, then an object $(f,g)$ in $X$ is a pair of morphisms in $C$ equipped with coalgebra structures, such that $\cod f = \dom g$, and $m(f,g)$ is 
\end{proof}

Now consider a comonad in $\FF{D}$ on an object $C$, i.e. a functorial factorization with half of the awfs structure. We can combine the universal properties of EM-objects and arrow objects into a universal property for $\LCoalg$, where now $\mathbb{L}$ is the comonad in $\mathcal{D}$ arising from the comonad in $\FF{D}$.

\begin{lemma}
	Let $(E,\eta,\epsilon,\delta)$ be a comonad in $\FF{D}$ on an object $C$. There is a 2-cell
	\[
	\begin{tikzcd}[row sep=0ex, column sep=4ex, bend angle=15]
		{} & |[alias=domA]| C^2 \drar[bend left][]{\cod} & \\
		\LCoalg \urar[bend left][]{U} \drar[bend right][swap]{U} && C \\
		& |[alias=codA]| C^2 \urar[bend right][swap]{E}
		\twocellA{\alpha}
	\end{tikzcd}
	\]
	satisfying equations
	\begin{gather}
	\begin{tikzcd}[ampersand replacement=\&, row sep=tiny, baseline=(B.base)]
		|[alias=B]| \LCoalg \rar{U} \drar[bend right=20][swap]{U}
			\& |[alias=domA]| C^2 \rar[bend left=60][domB]{\dom} \rar[][codB,swap]{\cod}
			\& C \\
		\& |[alias=codA]| C^2 \urar[bend right=20][swap]{E} \&
		\twocellA{\alpha}
		\twocellB{\kappa}
	\end{tikzcd}
	=
	\begin{tikzcd}[ampersand replacement=\&, bend angle=30, baseline=(B.base)]
		|[alias=B]| \LCoalg \rar{U}
			\& C^2 \rar[bend left][domA]{\dom} \rar[bend right][codA,swap]{E}
			\& C
		\twocellA{\eta}
	\end{tikzcd} \label{Eq:LCoalg1}
	\\
	\begin{tikzcd}[ampersand replacement=\&, row sep=tiny, baseline=(B.base)]
		{} \& |[alias=domA]| C^2 \drar[bend left=20][]{\cod} \& \\
		|[alias=B]| \LCoalg \urar[bend left=20][]{U} \rar[swap]{U}
			\& |[alias=codA]| C^2 \rar[][domB]{E} \rar[bend right=60][codB,swap]{\cod}
			\& C
		\twocellA{\alpha}
		\twocellB{\epsilon}
	\end{tikzcd}
	=
	\begin{tikzcd}[ampersand replacement=\&]
		X \rar{U} \& C^2 \rar{\cod} \& C
	\end{tikzcd} \label{Eq:LCoalg2}
	\\
	\begin{tikzcd}[ampersand replacement=\&, row sep=tiny, baseline=(B.base)]
		{} \& |[alias=domA]| C^2 \ar[bend left=20]{drr}{\cod} \&[-2em]\&[-2em] \\
		|[alias=B]| \LCoalg \urar[bend left=20][]{U} \rar[swap]{U}
			\& |[alias=codA]| C^2 \ar{rr}[domB]{E} \drar[bend right=30][swap]{L}
			\&\& C \\
		\&\& |[alias=codB]| C^2 \urar[bend right=30][swap]{E} \&
		\twocellA{\alpha}
		\twocellB{\delta}
	\end{tikzcd}
	=
	\begin{tikzcd}[ampersand replacement=\&, row sep=tiny, baseline=(B.base)]
		{} \&[-2em]\&[-2em] |[alias=domA]| C^2 \drar[bend left=20]{\cod} \& \\
		|[alias=B]| \LCoalg \ar[bend left=20]{urr}{U} \ar{rr}[domB]{U}  \drar[bend right=30][swap]{U}
			\&\& |[alias=codA]| C^2 \rar[swap]{E}
			\& C. \\
		\& |[alias=codB]| C^2 \urar[bend right=30][swap]{L} \&\&
		\twocellA{\alpha}
		\twocellB{\vec{\alpha}}
	\end{tikzcd} \label{Eq:LCoalg3}
	\end{gather}
	where $\vec{\alpha}$ is the unique 2-cell such that $\dom\vec{\alpha}=\id_{\dom U}$ and $\cod\vec{\alpha}=\alpha$, the existence of which is implied by Equation~\ref{Eq:LCoalg1}.

	Given any object $X$, together with a morphism $F\colon X\to C^2$ and a 2-cell $\beta\colon \cod F\Rightarrow EF$ satisfying equations
	\begin{compactenum}
		\item $\beta\circ\kappa F=\eta F$
		\item $\epsilon F\circ\beta = \id_{\cod F}$
		\item $\delta F\circ\beta = E\vec{\beta}\circ\beta$
	\end{compactenum}
	where $\vec{\beta}\colon F\Rightarrow LF$ is the unique 2-cell such that $\dom\vec{\beta}=\id_{\dom F}$ and $\cod\vec{\beta}=\beta$; there is a unique morphism $\hat{F}\colon X\to \LCoalg$ such that $U\hat{F}=F$ and $\alpha\hat{F}=\vec{\beta}$.
\end{lemma}
\begin{proof}
	$U$ is simply the $U$ from proposition~\ref{Prop:EMObject}, while the 2-cell $\alpha$ there is the 2-cell $\vec{\alpha}$ here. The equation $\vec{\epsilon}U\circ\vec{\alpha}=\id_{F}$ implies that $\dom \vec{\alpha}=\id_{\dom U}$. With that observation, the rest of the equations follow immediately from the universal property of $C^2$ and the equations $\epsilon U\circ\alpha = \id_U$ and $\delta U \circ \alpha = L\alpha\circ\alpha$ from Proposition~\ref{Prop:EMObject}.
\end{proof}


