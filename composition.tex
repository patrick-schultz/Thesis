% -*- root: thesis.tex -*-

\chapter{Composition of $\mathbb{L}$-coalgebras}

For this section, we will continue to let $\mathbb{D}=\Sq(\mathcal{D})$ be the double category of squares in a 2-category $\mathcal{D}$ with arrow objects.

In an algebraic weak factorization system, the categories $\LCoalg$ and $\RAlg$ respectively play the roles of the left and right classes of morphisms of the weak factorization system. In an ordinary weak factorization system, these two classes of morphisms are closed under composition. In \cite{garner:soa}, this is strengthened to a composition functor
\[
	\LCoalg \prodb_C \LCoalg \to \LCoalg
\]
and in \cite{riehl:nwfs-model}, it is shown that colax morphisms of awfs preserve this composition. Similarly, there is a composition functor on $\RAlg$ which is preserved by lax morphisms of awfs.

In this section, we will generalize these results to the setting of bimonads in $\FFD$.

First, recall from \cite{street:ftm} the following proposition.

\begin{proposition}\label{Prop:EMObject}
	Let $C$ be a category, and $\mathbb{L}=(L,\epsilon,\delta)$ be a comonad on $C$. The category of coalgebras $\LCoalg$ has a universal property as follows:
	\begin{itemize}
	 	\item There is a forgetful functor $U\colon \LCoalg \to C$ and a natural transformtion $\alpha\colon U \Rightarrow LU$, satisfying $\epsilon U\circ\alpha = \id_U$ and $\delta U \circ \alpha = L\alpha\circ\alpha$.
	 	\item $(U,\alpha)$ is universal among such pairs satisfying such equations. Given another such pair $(F,\beta)$, where $F\colon X\to C$, there exists a unique functor $\hat{F}\colon X\to \LCoalg$ such that $U\hat{F}=F$ and $\alpha\hat{F}=\beta$.
	 \end{itemize}
	 Any colax morphism of comonads $(F,\phi)\colon(C,L_1,\epsilon_1,\delta_1)\to(D,L_2,\epsilon_2,\delta_2)$ induces a functor $\tilde{F}\colon\LCoalg[1]\to\LCoalg[2]$ such that $U\tilde{F}=FU$.
\end{proposition}

For the rest of this section, assume that $\mathcal{D}$ \emph{has EM-objects for comonads}, i.e. for every comonad $\mathbb{L}$ in $\mathcal{D}$ there is an object $\LCoalg$ satisfying the universal property above.

The main goal of this section will be to prove the following theorem:

\begin{theorem}\label{Thm:CoalgLaxFunctor}
	There is a lax double functor
	\[
		\Coalg\colon \DComon(\FF(\Sq(\mathcal{D}))) \to \Span(\mathcal{D}_0)
	\] 
	where $\mathcal{D}_0$ is the ordinary category underlying the (strict) 2-category $\mathcal{D}$, which is the identity on the vertical categories, and which takes a comonad $(E,\eta,\epsilon,\delta)$ in $\FF(\Sq(\mathcal{D}))$ to the span
	\[
	\begin{tikzcd}[column sep=large]
		C & \LCoalg \lar[swap]{\dom U} \rar{\cod U} & C.
	\end{tikzcd}
	\]
\end{theorem}

Before we get to the proof of Theorem~\ref{Thm:CoalgLaxFunctor}, we will need to establish several technical lemmas.

Consider a comonad in $\FFD$ on an object $C$, i.e. a functorial factorization with half of the awfs structure. We can combine the universal properties of EM-objects and arrow objects into a universal property for $\LCoalg$, where now $\mathbb{L}$ is the comonad in $\mathcal{D}$ arising from the comonad in $\FFD$.

\begin{lemma}\label{Lem:FFLCoalgUniversalProperty}
	Let $(E,\eta,\epsilon,\delta)$ be a comonad in $\FFD$ on an object $C$. There is a 2-cell
	\[
	\begin{tikzcd}[row sep=0ex, column sep=4ex, bend angle=15]
		{} & |[alias=domA]| C^2 \drar[bend left][]{\cod} & \\
		\LCoalg \urar[bend left][]{U} \drar[bend right][swap]{U} && C \\
		& |[alias=codA]| C^2 \urar[bend right][swap]{E}
		\twocellA{\alpha}
	\end{tikzcd}
	\]
	satisfying equations
	\begin{gather}
	\begin{tikzcd}[ampersand replacement=\&, row sep=tiny, baseline=(B.base)]
		|[alias=B]| \LCoalg \rar{U} \drar[bend right=20][swap]{U}
			\& |[alias=domA]| C^2 \rar[bend left=60][domB]{\dom} \rar[][codB,swap]{\cod}
			\& C \\
		\& |[alias=codA]| C^2 \urar[bend right=20][swap]{E} \&
		\twocellA{\alpha}
		\twocellB{\kappa}
	\end{tikzcd}
	=
	\begin{tikzcd}[ampersand replacement=\&, bend angle=30, baseline=(B.base)]
		|[alias=B]| \LCoalg \rar{U}
			\& C^2 \rar[bend left][domA]{\dom} \rar[bend right][codA,swap]{E}
			\& C
		\twocellA{\eta}
	\end{tikzcd} \label{Eq:LCoalg1}
	\\
	\begin{tikzcd}[ampersand replacement=\&, row sep=tiny, baseline=(B.base)]
		{} \& |[alias=domA]| C^2 \drar[bend left=20][]{\cod} \& \\
		|[alias=B]| \LCoalg \urar[bend left=20][]{U} \rar[swap]{U}
			\& |[alias=codA]| C^2 \rar[][domB]{E} \rar[bend right=60][codB,swap]{\cod}
			\& C
		\twocellA{\alpha}
		\twocellB{\epsilon}
	\end{tikzcd}
	=
	\begin{tikzcd}[ampersand replacement=\&]
		X \rar{U} \& C^2 \rar{\cod} \& C
	\end{tikzcd} \label{Eq:LCoalg2}
	\\
	\begin{tikzcd}[ampersand replacement=\&, row sep=tiny, baseline=(B.base)]
		{} \& |[alias=domA]| C^2 \ar[bend left=20]{drr}{\cod} \&[-2em]\&[-2em] \\
		|[alias=B]| \LCoalg \urar[bend left=20][]{U} \rar[swap]{U}
			\& |[alias=codA]| C^2 \ar{rr}[domB]{E} \drar[bend right=30][swap]{L}
			\&\& C \\
		\&\& |[alias=codB]| C^2 \urar[bend right=30][swap]{E} \&
		\twocellA{\alpha}
		\twocellB{\delta}
	\end{tikzcd}
	=
	\begin{tikzcd}[ampersand replacement=\&, row sep=tiny, baseline=(B.base)]
		{} \&[-2em]\&[-2em] |[alias=domA]| C^2 \drar[bend left=20]{\cod} \& \\
		|[alias=B]| \LCoalg \ar[bend left=20]{urr}{U} \ar{rr}[domB]{U}  \drar[bend right=30][swap]{U}
			\&\& |[alias=codA]| C^2 \rar[swap]{E}
			\& C. \\
		\& |[alias=codB]| C^2 \urar[bend right=30][swap]{L} \&\&
		\twocellA{\alpha}
		\twocellB{\vec{\alpha}}
	\end{tikzcd} \label{Eq:LCoalg3}
	\end{gather}
	where $\vec{\alpha}$ is the unique 2-cell such that $\dom\vec{\alpha}=\id_{\dom U}$ and $\cod\vec{\alpha}=\alpha$, the existence of which is implied by Equation~\ref{Eq:LCoalg1}.

	Given any object $X$, together with a morphism $F\colon X\to C^2$ and a 2-cell $\beta\colon \cod F\Rightarrow EF$ satisfying equations
	\begin{compactenum}
		\item $\beta\circ\kappa F=\eta F$
		\item $\epsilon F\circ\beta = \id_{\cod F}$
		\item $\delta F\circ\beta = E\vec{\beta}\circ\beta$
	\end{compactenum}
	where $\vec{\beta}\colon F\Rightarrow LF$ is the unique 2-cell such that $\dom\vec{\beta}=\id_{\dom F}$ and $\cod\vec{\beta}=\beta$; there is a unique morphism $\hat{F}\colon X\to \LCoalg$ such that $U\hat{F}=F$ and $\alpha\hat{F}=\vec{\beta}$.
\end{lemma}
\begin{proof}
	$U$ is simply the $U$ from proposition~\ref{Prop:EMObject}, while the 2-cell $\alpha$ there is the 2-cell $\vec{\alpha}$ here. The equation $\vec{\epsilon}U\circ\vec{\alpha}=\id_{F}$ implies that $\dom \vec{\alpha}=\id_{\dom U}$. With that observation, the rest of the equations follow immediately from the universal property of $C^2$ and the equations $\epsilon U\circ\alpha = \id_U$ and $\delta U \circ \alpha = L\alpha\circ\alpha$ from Proposition~\ref{Prop:EMObject}.
\end{proof}

We will now prove a couple of simple lemmas to establish the existence of certain 2-cells in $\mathcal{D}$ using the arrow object universal property. For each of these lemmas, let $(E_1,\eta_1,\epsilon_1,\delta_1)$ and $(E_2,\eta_2,\epsilon_2,\delta_2)$ be two comonads in $\FF(\Sq(\mathcal{D}))$, both on the same object $C$; let $X$ be the pullback
\[
\begin{tikzcd}[row sep=small]
	{} & X \dlar[swap]{P_1} \drar{P_2} % \ar[phantom]{dd}[pos=.1,rotate=45]{\lefthalfcup}
	& \\
	\LCoalg[1] \drar[sloped,pos=.8,swap]{\cod U}
		&& \LCoalg[2] \dlar[sloped,pos=.8]{\dom U} \\
	& C &
\end{tikzcd}
\]
let $m$ be the 2-cell
\[
\begin{tikzcd}[bend angle=20,row sep=small]
	{} & C^2 \drar[bend left][sloped,domA,pos=.2,inner sep=1pt]{\dom} \drar[bend right][swap,sloped,codA,inner sep=1pt]{\cod} & \\
	X \urar{UP_1} \drar[swap]{UP_2} && C \\
	& C^2 \urar[bend left][sloped,domB,inner sep=1pt]{\dom} \urar[bend right][swap,sloped,codB,pos=.2,inner sep=1pt]{\cod} &
	\twocellA{\kappa}
	\twocellB{\kappa}
\end{tikzcd}
\]
and let $\vec{m}\colon X\to C^2$ be the corresponding 1-cell with $\kappa\vec{m}=m$.

\begin{lemma}\label{Lem:Zeta}
	There is a 2-cell
	\[
	\begin{tikzcd}[row sep=-.5em]
		{} & |[alias=domA]| \LCoalg[1] \drar[bend left=15][]{U} & \\
		X \urar[bend left=15][]{P_1} \ar[bend right=20]{rr}[swap,codA]{\vec{m}} && C^2
		\twocellA{\zeta}
	\end{tikzcd}
	\]
	such that $\dom\zeta=\id$ and 
	\[
	\begin{tikzcd}[row sep=-.5em,baseline=(B.base)]
		{} & |[alias=domA]| \LCoalg[1] \drar[bend left=15][]{U} && \\
		|[alias=B]| X \urar[bend left=15][]{P_1} \ar[bend right=20]{rr}[swap,codA]{\vec{m}} 
			&& C^2 \rar{\cod}
			& C
		\twocellA{\zeta}
	\end{tikzcd}
	=
	\begin{tikzcd}[bend angle=20,row sep=small,baseline=(B.base)]
		{} & C^2 \drar[sloped,pos=.2]{\cod} & \\
		|[alias=B]| X \urar{UP_1} \drar[swap]{UP_2} && C \\
		& C^2 \urar[bend left][sloped,domA,pos=.6,inner sep=1pt]{\dom} 
				\urar[bend right][swap,sloped,codA,pos=.2,inner sep=1pt]{\cod} &
		\twocellA{\kappa}
	\end{tikzcd}
	\]
\end{lemma}
\begin{proof}
	Equation~\eqref{Eq:ArrowObject2} becomes
	\[
	\begin{tikzcd}[bend angle=30]
		X \rar{\vec{m}} & C^2 \rar[bend left][domA]{\dom} \rar[bend right][codA,swap]{\cod} & C
		\twocellA{\kappa}
	\end{tikzcd}
	=
	\begin{tikzcd}[bend angle=20,row sep=small,baseline=(B.base)]
		{} & C^2 \drar[bend left][sloped,domA,pos=.2,inner sep=1pt]{\dom} \drar[bend right][swap,sloped,codA,inner sep=1pt]{\cod} & \\
		|[alias=B]| X \urar{UP_1} \drar[swap]{UP_2} && C \\
		& C^2 \urar[bend left][sloped,domB,inner sep=1pt]{\dom} \urar[bend right][swap,sloped,codB,pos=.2,inner sep=1pt]{\cod} &
		\twocellA{\kappa}
		\twocellB{\kappa}
	\end{tikzcd}
	\]
	which is simply the definition of $\vec{m}$.
\end{proof}

\begin{lemma}\label{Lem:Nu}
	There is a 2-cell
	\[
	\begin{tikzcd}[row sep=0ex, column sep=4ex, bend angle=15]
		{} & |[alias=domA]| \LCoalg[2] \drar[bend left][]{U} & \\
		X \urar[bend left][]{P_2} \drar[bend right][swap]{\vec{m}} && C^2 \\
		& |[alias=codA]| C^2 \urar[bend right][swap]{R_1}
		\twocellA{\nu}
	\end{tikzcd}
	\]
	such that $\cod\nu=\id$ and
	\[
	\begin{tikzcd}[row sep=0ex, column sep=4ex, bend angle=15, baseline=(B.base)]
		{} & |[alias=domA]| \LCoalg[2] \drar[bend left][]{U} && \\
		|[alias=B]| X \urar[bend left][]{P_2} \drar[bend right][swap]{\vec{m}} && C^2 \rar{\dom} & C \\
		& |[alias=codA]| C^2 \urar[bend right][swap]{R_1} &&
		\twocellA{\nu}
	\end{tikzcd}
	=
	\begin{tikzcd}[row sep=tiny, column sep=4ex, bend angle=25, baseline=(B.base)]
		{} & \LCoalg[2] \rar{U}
			& C^2 \drar[bend left]{\dom} & \\
		|[alias=B]| X \urar[bend left]{P_2} \rar{P_1} \ar[bend right]{drr}[swap,codA]{\vec{m}}
			& |[alias=domA]| \LCoalg[1] \rar{U} \drar[swap]{U}
			& |[alias=domB]| C^2 \rar{\cod}
			& C \\
		&& |[alias=codB]| C^2 \urar[bend right][swap]{E_1} &
		\twocellA{\zeta}
		\twocellB{\alpha_1}
	\end{tikzcd}
	\]
\end{lemma}
\begin{proof}
	We just need to verify Equation~\eqref{Eq:ArrowObject2}:
	\begin{multline*}
	\begin{tikzcd}[row sep=tiny, column sep=4ex, bend angle=25, baseline=(B.base), ampersand replacement=\&]
		{} \& \LCoalg[2] \rar{U}
			\& C^2 \drar[bend left]{\dom} \& \\
		|[alias=B]| X \urar[bend left]{P_2} \rar{P_1} \ar[bend right]{drr}[swap,codA]{\vec{m}}
			\& |[alias=domA]| \LCoalg[1] \rar{U} \drar[swap]{U}
			\& |[alias=domB]| C^2 \rar{\cod}
			\& C \\
		\&\& |[alias=codB]| C^2 \urar[][domC,inner sep=0pt]{E_1} \urar[bend right=50][swap,codC]{\cod} \&
		\twocellA{\zeta}
		\twocellB{\alpha_1}
		\twocellC{\epsilon_1}
	\end{tikzcd}
	\\
	=
	\begin{tikzcd}[row sep=tiny, column sep=4ex, bend angle=25, baseline=(B.base), ampersand replacement=\&]
		{} \& \LCoalg[2] \rar{U}
			\& C^2 \drar[bend left]{\dom} \& \\
		|[alias=B]| X \urar[bend left]{P_2} \rar{P_1} \ar[bend right=60]{rr}[swap,codA]{\vec{m}}
			\& |[alias=domA]| \LCoalg[1] \rar{U}
			\& |[alias=domB]| C^2 \rar{\cod}
			\& C \\
		\twocellA{\zeta}
	\end{tikzcd}
	\\
	=
	\begin{tikzcd}[column sep=4ex, bend angle=25, baseline=(B.base), ampersand replacement=\&]
		X \rar{P_2} 
			\& \LCoalg[1] \rar{U} 
			\& C^2 \rar[bend left][domA]{\dom} \rar[bend right][swap,codA]{\cod}
			\& C
		\twocellA{\kappa}
	\end{tikzcd}
	\end{multline*}
	where the first equation follows from \eqref{Eq:LCoalg2}, and the second by reducing $\cod\zeta$ using Lemma~\ref{Lem:Zeta}.
\end{proof}

\begin{proof}[Proof of Theorem \ref{Thm:CoalgLaxFunctor}]
	For notational convenience, let $G=\Coalg$ be the lax double functor we need to establish. Both the double categories $\DComon(\FF(\Sq(\cat{D})))$ and $\Span(\cat{D}_0)$ have $\cat{D}_0$ as vertical category, and $G_0$ is simply the identity. From the statement of the theorem, $G$ takes an object in $\DComon(\FF(\Sq(\cat{D})))$ to the span
	\[
	\begin{tikzcd}[column sep=large]
		C & \LCoalg \lar[swap]{\dom U} \rar{\cod U} & C.
	\end{tikzcd}
	\]

	To define the behavior of G on 2-cells, consider a 2-cell in $\DComon(\FF(\Sq(\cat{D})))$:
	\[
	\begin{tikzcd}[column sep=4em]
		C \rar[tick][domA]{(E_1,\eta_1,\epsilon_1,\delta_1)} \dar[swap]{F} & C \dar{F} \\
		D \rar[tick][codA,swap]{(E_2,\eta_2,\epsilon_2,\delta_2)} & D.
		\twocellA{\phi}
	\end{tikzcd}
	\]
	By Corollary~\ref{Cor:RLMon}, $\phi$ induces a colax morphism of comonads from $L_1$ to $L_2$, hence by Proposition~\ref{Prop:EMObject} there is an induced morphim $\tilde{\phi}$ between the EM-objects such that $U\tilde{\phi}=F^2U$. We can then define $G\phi$ to be the morphism of spans
	\[
	\begin{tikzcd}
		C \dar[swap]{F} 
			& C^2 \lar[swap]{\dom} \dar{F^2}
			& \LCoalg[1] \lar[swap]{U} \dar{\tilde{\phi}} \rar{U}
			& C^2 \rar{\cod} \dar{F^2}
			& C \dar{F} \\
		D & D^2 \lar{\dom}
			& \LCoalg[2] \lar{U} \rar[swap]{U}
			& D^2 \rar[swap]{\cod}
			& D.
	\end{tikzcd}
	\]

	Next we must define the coherence data $G_I$ and $G_{\otimes}$. We will define $G_I$ to be the morphims of spans
	\[
	\begin{tikzcd}[row sep=small,column sep=small]
		{} & C \dlar[swap]{\id} \drar{\id} \ar{dd}{G_I} & \\
		C && C \\
		& \LCoalg[I] \ular{\dom U} \urar[swap]{\cod U}
	\end{tikzcd}
	\]
	defined via Lemma~\ref{Lem:FFLCoalgUniversalProperty} by the equations $UG_I=i\colon C\to C^2$ and $\alpha_I G_I$ is the identity on $\dom i=\cod i$. The conditions of the lemma are trivially satisfied.

	We will similarly use Lemma~\ref{Lem:FFLCoalgUniversalProperty} to define $G_{\otimes}$. Let $X$, $\vec{m}$, $\zeta$, and $\nu$ be as defined earlier in the section. $G_{\otimes}$ is a morphism of spans
	\[
	\begin{tikzcd}[row sep=small,column sep=small]
		{} & X \dlar[swap]{\dom UP_1} \drar{\cod UP_2} \ar{dd}{G_{\otimes}} & \\
		C && C. \\
		& \LCoalg[1\otimes2] \ular{\dom U} \urar[swap]{\cod U}
	\end{tikzcd}
	\]
	We will define $G_{\otimes}$ to be the 1-cell such that $UG_{\otimes}=\vec{m}$ and
	\[
	\begin{tikzcd}[row sep=0ex, column sep=4ex, bend angle=15, baseline=(B.base)]
		{} && |[alias=domA]| C^2 \drar[bend left][]{\cod} & \\
		|[alias=B]| X \rar{G_{\otimes}} 
			& \LCoalg \urar[bend left][]{U} \drar[bend right][swap]{U} 
			&& C \\
		&& |[alias=codA]| C^2 \urar[bend right][swap]{E_{1\otimes2}}
		\twocellA{\alpha_{1\otimes2}}
	\end{tikzcd}
	=
	\begin{tikzcd}[bend angle=25, row sep=0ex, column sep=3ex]
		& |[alias=domA]| \LCoalg[2] \rar{U} \ar{ddr}[description]{U}
			& |[alias=domB]| C^2 \drar[bend left][]{\cod} & \\
		X \urar[bend left][]{P_2} \drar[bend right][swap]{\vec{m}}
			&&& C \\
		& |[alias=codA]| C^2 \rar[swap]{R_1}
			& |[alias=codB]| C^2 \urar[bend right][swap]{E_2} &
		\twocellA{\nu}
		\twocellB{\alpha_2}
	\end{tikzcd}
	\]
	In other words, in the notation of Lemma~\ref{Lem:FFLCoalgUniversalProperty} let $F=\vec{m}$ and $\beta=E_2\nu\circ\alpha_2P_2$, and define $G_{\otimes}=\hat{F}$.

	We now need to check equations 1-3 of Lemma~\ref{Lem:FFLCoalgUniversalProperty} to verify that $G_{\otimes}$ is well defined. We will check these equationally to save space, but the reader may want to draw out the diagrams for themselves to follow along. For the first equation:
	\begin{alignat*}{2}
		\hspace{1em}&\hspace{-1em} E_2\nu \circ \alpha_2P_2 \circ \kappa\vec{m} && \\
		&= E_2\nu \circ \alpha_2P_2 \circ \kappa UP_2 \circ \kappa UP_1 
			&& \text{Def of $\vec{m}$} \\
		&= E_2\nu \circ (\alpha_2\circ\kappa U)P_2 \circ \kappa UP_1 && \\
		&= E_2\nu \circ \eta_2UP_2 \circ \kappa UP_1 
			&& \text{Eq \eqref{Eq:LCoalg1}} \\
		&= \eta_2R_1\vec{m} \circ \dom\nu \circ \kappa UP_1 
			&& \text{Interchange} \\
		&= \eta_2R_1\vec{m} \circ E_1\zeta \circ \alpha_1P_1 \circ \kappa UP_1 
			&\qquad& \text{Def of $\nu$} \\
		&= \eta_2R_1\vec{m} \circ E_1\zeta \circ (\alpha_1\circ\kappa U)P_1 && \\
		&= \eta_2R_1\vec{m} \circ E_1\zeta \circ \eta_1UP_1 
			&&\text{Eq \eqref{Eq:LCoalg1}} \\
		&= \eta_{1\otimes 2}\vec{m} \circ \dom\zeta 
			&&\text{Interchange; Def of $\eta_{1\otimes 2}$} \\
		&= \eta_{1\otimes 2}\vec{m} 
			&&\dom\zeta=\id
	\end{alignat*}
	and the second:
	\begin{alignat*}{2}
		\hspace{1em}&\hspace{-1em} \epsilon_{1\otimes 2}\vec{m} \circ E_2\nu \circ \alpha_2P_2 && \\
		&= \epsilon_2R_1\vec{m} \circ E_2\nu \circ \alpha_2P_2  
			&& \text{Def of $\epsilon_{1\otimes2}$} \\
		&= \cod\nu \circ (\epsilon_2U\circ\alpha_2)P_2 
			&\qquad& \text{Interchange} \\
		&= \id_{\cod\vec{m}}. 
			&&\text{Eq \eqref{Eq:LCoalg2}; $\cod\nu=\id$}
	\end{alignat*}

	The third equation is a bit trickier to prove. We will need to prove two intermediate equations first, using the arrow object universal property.
	\begin{lemma*}
		\begin{equation}\label{Eq:CompLem1}
			i^L\vec{m} \circ L_1\zeta \circ \vec{\alpha}_1P_1 = \vec{\beta}\circ\zeta
		\end{equation}
	\end{lemma*}
	\begin{proof}
		We must show the 2-cells become equal upon composition with $\dom$ and $\cod$:
		\[
			\dom(i^L\vec{m} \circ L_1\zeta \circ \vec{\alpha}_1P_1) = \id_{\dom\vec{m}} = \dom(\vec{\beta}\circ\zeta)
		\]
		and
		\begin{alignat*}{2}
			\hspace{1em}&\hspace{-1em} \cod(i^L\vec{m} \circ L_1\zeta \circ \vec{\alpha}_1P_1) && \\
			&= \cod i^L\vec{m} \circ E_1\zeta \circ \cod\vec{\alpha}_1P_1 &\qquad& \\
			&= \eta_2R_1\vec{m} \circ E_1\zeta \circ \alpha_1P_1 
				&& \text{Def of $i^L$, $\vec{\alpha}$} \\
			&= \eta_2R_1\vec{m} \circ \dom\nu
				&& \text{Def of $\nu$} \\
			&= E_2\nu \circ \eta_2UP_2
				&& \text{Interchange} \\
			&= E_2\nu \circ (\alpha_2 \circ \kappa U)P_2
				&& \text{Eq \eqref{Eq:LCoalg1}} \\
			&= (E_2\nu \circ \alpha_2P_2) \circ \kappa UP_2 && \\
			&= \cod\vec{\beta} \circ \cod\zeta
				&& \text{Def of $\vec{\beta}$, $\zeta$} \\
			&= \cod(\vec{\beta}\circ\zeta). &&
		\end{alignat*}
	\end{proof}
	\begin{lemma*}
		\begin{equation}\label{Eq:CompLem2}
			R_1\vec{\beta} \circ \nu = w\vec{m} \circ L_2\delta_1^R\vec{m} \circ L_2\nu \circ \vec{\alpha}_2P_2
		\end{equation}
	\end{lemma*}
	\begin{proof}
		Again we must prove equality after composing with $\dom$ and $\cod$:
		\begin{alignat*}{2}
			\hspace{1em}&\hspace{-1em} \dom(R_1\vec{\beta} \circ \nu) && \\
			&= E_1\vec{\beta} \circ \dom\nu && \\
			&= E_1\vec{\beta} \circ E_1\zeta \circ \alpha_1P_1
				&& \text{Def of $\nu$} \\
			&= E_1(\vec{\beta}\circ\zeta) \circ \alpha_1P_1 && \\
			&= E_1(i^L\vec{m} \circ L_1\zeta \circ \vec{\alpha}_1P_1) \circ \alpha_1P_1
				&& \text{Eq \eqref{Eq:CompLem1}} \\
			&= E_1i^L\vec{m} \circ E_1L_1\zeta \circ (E_1\vec{\alpha}_1\circ\alpha_1)P_1 && \\
			&= E_1i^L\vec{m} \circ E_1L_1\zeta \circ (\delta_1U\circ\alpha_1)P_1
				&& \text{Eq \eqref{Eq:LCoalg3}} \\
			&= E_1i^L\vec{m} \circ \delta_1\vec{m} \circ E_1\zeta \circ \alpha_1P_1
				&& \text{Interchange} \\
			&= \dom w\vec{m} \circ \dom\delta_1^R\vec{m} \circ \dom\nu \circ \dom\vec{\alpha}_2P_2
				&\qquad& \text{Defs of $w$, $\delta^R$, $\nu$, $\vec{\alpha}$} \\
			&= \dom (w\vec{m} \circ L_2\delta_1^R\vec{m} \circ L_2\nu \circ \vec{\alpha}_2P_2)
		\end{alignat*}
		and
		\begin{alignat*}{2}
			\hspace{1em}&\hspace{-1em} \cod(R_1\vec{\beta} \circ \nu) && \\
			&= \cod\vec{\beta} \circ \cod\nu && \\
			&= E_2\nu \circ \alpha_2P_2
				&& \text{Defs of $\vec{\beta}$, $\nu$} \\
			&= E_2(p^R\circ\delta_1^R)\vec{m} \circ E_2\nu \circ \alpha_2P_2
				&& p^R\circ\delta^R = \id \\
			&= E_2p^R\vec{m} \circ E_2\delta_1^R\vec{m} \circ E_2\nu \circ \alpha_2P_2 && \\
			&= \cod w\vec{m} \circ \cod L_2\delta_1^R\vec{m} \circ \cod L_2\nu \circ \cod\vec{\alpha}_2P_2
				&\qquad& \text{Defs of $w$, $L$, $\vec{\alpha}$} \\
			&= \cod(w\vec{m} \circ L_2\delta_1^R\vec{m} \circ L_2\nu \circ \vec{\alpha}_2P_2)
		\end{alignat*}
	\end{proof}

	Now we are prepared to prove the third equation of Lemma~\ref{Lem:FFLCoalgUniversalProperty} validating our definition of $G_{\otimes}$:
	\begin{alignat*}{2}
		\hspace{1em}&\hspace{-1em} \delta_{1\otimes2}\vec{m} \circ E_2\nu \circ \alpha_2P_2 && \\
		&= (E_2w \circ \delta_2R_{1\odot1} \circ E_2\delta_1^R)\vec{m} \circ E_2\nu \circ \alpha_2P_2
			&& \text{Def of $\delta_{1\otimes2}$} \\
		&= E_2(w\vec{m} \circ L_2\delta_1^R\vec{m} \circ L_2\nu) \circ (\delta_2U \circ \alpha_2)P_2
			&& \text{Interchange} \\
		&= E_2(w\vec{m} \circ L_2\delta_1^R\vec{m} \circ L_2\nu) \circ (E_2\vec{\alpha}_2 \circ \alpha_2)P_2
			&\qquad& \text{Eq \eqref{Eq:LCoalg3}} \\
		&= E_2(w\vec{m} \circ L_2\delta_1^R\vec{m} \circ L_2\nu \circ \vec{\alpha}_2P_2) \circ \alpha_2P_2 && \\
		&= E_2(R_1\vec{\beta}\circ\nu) \circ \alpha_2P_2
			&& \text{Eq \eqref{Eq:CompLem2}} \\
		&= E_{1\otimes 2}\vec{\beta} \circ E_2\nu \circ \alpha_2P_2
			&& \text{Def of $E_{1\otimes2}$}
	\end{alignat*}

	The verification that the definitions of $G_I$ and $G_{\otimes}$ form natural families, and of the coherence axioms for a lax double functor, is tedious, but follows from what we have presented here without requiring any new ideas or ingenuity.
\end{proof}

\begin{corollary}\label{Cor:CoalgComp}
	For any awfs $(E,\eta,\mu,\epsilon,\delta)$ on an object $C$ in $\mathcal{D}$, the multiplication $\mu$ induces a composition functor on $\LCoalg$, and the functor between EM-objects induced by any colax morphism of awfs preserves this composition.
\end{corollary}
\begin{proof}
	Any awfs $(E,\eta,\mu,\epsilon,\delta)$ has an underlying object in $\DComon(\FF(\Sq(\mathcal{D})))$, by simply forgetting $\mu$. The lax double-functor $\mathrm{Coalg}$ takes this to a span
	\[
	\begin{tikzcd}[column sep=large]
		C & \LCoalg \lar[swap]{\dom U} \rar{\cod U} & C.
	\end{tikzcd}
	\]
	The multiplication $\mu$ provides this object in $\DComon(\FF(\Sq(\mathcal{D})))$ with a monad structure, and lax double-functors preserve monads, so $\mu$ induces a monad structure on this span. A multiplication on this span is a morphism $\pi$:
	\[
	\begin{tikzcd}[row sep=small,sloped,pos=.1]
		{} & X \dlar[swap]{\dom UP_1} \drar{\cod UP_2} \ar{dd}[sloped=false,pos=.5]{\pi} & \\
		C && C \\
		& \LCoalg \ular{\dom U} \urar[swap]{\cod U}
	\end{tikzcd}
	\]
	where $X$ is the pullback in the composite span
	\[
	\begin{tikzcd}[row sep=small]
		{} && X \dlar[swap]{P_1} \drar{P_2} && \\
		& \LCoalg \dlar[swap,sloped,pos=.1]{\dom U} \drar[sloped,pos=.1]{\cod U}
			&& \LCoalg \dlar[swap,sloped,pos=.1]{\dom U} \drar[sloped,pos=.1]{\cod U} & \\
		C && C && C.
	\end{tikzcd}
	\]

	The morphism $\pi$ is the composition structure that we want. If $\mathcal{D}=\mathrm{Cat}$ is the 2-category of small categories, then an object $(f,g)$ in $X$ is a pair of morphisms in $C$ equipped with coalgebra structures, such that $\cod f = \dom g$, and $\pi(f,g)$ is a morphism equipped with a coalgebra structure, with $\dom \pi(f,g)=\dom f$ and $\cod \pi(f,g)=\cod g$.

	Of course, what we really want is that the morphism underlying the coalgebra $\pi(f,g)$ is the composition $g\circ f$. To see that this is the case, notice that the composition $\pi$ is given by $G(\vec{\mu})\circ G_{\otimes}$, where $G(\vec{\mu})$ is the 1-cell between EM-objects induced by the colax morphism of comonads $\vec{\mu}$. Using the fact that $\vec{\mu}$ is a globular 2-cell in $\FF(\Sq(\mathcal{D}))$, we have $U\pi=UG(\vec{\mu})G_{\otimes}=UG_{\otimes}=\vec{m}$. Looking at the definition of $\vec{m}\colon X\to C^2$, it is clear that it is precisely the functor that sends the pair $(f,g)$ to the composition $gf$, ignoring the coalgebra structures.
\end{proof}

Corollary~\ref{Cor:CoalgComp} actually has a converse, which was used extensively in \cite{riehl:nwfs-model}. We can prove this converse in our generalized framework as well.

\begin{proposition}
	Let $(E,\eta,\epsilon,\delta)$ be a comonad in $\FFD$. A multiplication map on the span $\Coalg E$ determines a monad structure on $E$. 

	Furthermore, given two bimonads $(E_1,\eta_1,\mu_1,\epsilon_1,\delta_1)$ and $(E_2,\eta_2,\mu_2,\epsilon_2,\delta_2)$, and a morphism $\theta\colon E_1\to E_2$ in $\Comon(\FFD)$, if the induced map $\tilde{\theta}\colon\LCoalg[1]\to\LCoalg[2]$ commutes with the multiplication, then $\theta$ is in fact a bimonad morphism.
\end{proposition}
\begin{proof}
	We will again need to start with some preliminary constructions.

	Let $\hat{L}\colon C^2\to\LCoalg$ be the morphism determined by $U\hat{L}=L$ and $\alpha\hat{L}=\delta$.

	Let $(\hat{L},\hat{L}R)$ be the map into the pushout $X$ as in the diagram
	\[
	\begin{tikzcd}[row sep=small,column sep=small,bend angle=25]
		{} &[2em]& \LCoalg \drar{\cod U} & \\
		C^2 \ar[bend left]{urr}{\hat{L}}
			\rar{(\hat{L},\hat{L}R)}
			\ar[bend right]{drr}{\hat{L}R}
		& X \urar{P_1} \drar[swap]{P_2}
		&& C. \\
		&& \LCoalg \urar[swap]{\dom U} &
	\end{tikzcd}
	\]
	This is well defined since $\cod U\hat{L}=E=\dom U\hat{L}R$.

	Consider the morphism $\vec{m}(\hat{L},\hat{L}R)\colon C^2\to C^2$. We compute
	\[
		\dom\vec{m}(\hat{L},\hat{L}R) = \dom UP_1(\hat{L},\hat{L}R) = \dom U\hat{L} = \dom L = \dom
	\]
	and
	\[
		\cod\vec{m}(\hat{L},\hat{L}R) = \cod UP_2(\hat{L},\hat{L}R) = \cod U\hat{L}R = \cod LR = ER.
	\]
	There is a 2-cell $\psi\colon\vec{m}(\hat{L},\hat{L}R)\Rightarrow\id_{C^2}$ with $\dom\psi=\id_{\dom}$ and $\cod\psi=\epsilon R$. That $\psi$ is well defined comes down to the computation
	\[
		\cod\psi \circ \kappa\vec{m}(\hat{L},\hat{L}R) = \epsilon R \circ \eta R \circ \dom\vec{\eta} = \kappa R \circ \eta = \epsilon\circ\eta = \kappa.
	\]

	We will define the monad multiplication $\mu\colon ER\Rightarrow E$ to be the composition
	\[
		\mu = E\psi \circ \alpha\pi(\hat{L},\hat{L}R)
	\]
	which makes sense since $\cod\alpha\pi(\hat{L},\hat{L}R) = EU\pi(\hat{L},\hat{L}R)=E\vec{m}(\hat{L},\hat{L}R)$.
	%Todo: Verify monad axioms. Surprisingly hard!
\end{proof}