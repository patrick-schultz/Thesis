% -*- root: thesis.tex -*-

\chapter{2-Fold Double Categories}

In this section we will propose a generalization of the 2-fold monoidal categories as used in \cite{garner:soa}.

A 2-fold double category $\mathbb{D}$ is a structure which has two different underlying double categories, both of which have the same vertical category $\dvert(\mathbb{D})$. We will start with a concise formal definition, and then expand on the definition more concretely.

\begin{definition}
	A \emph{2-fold double category} $\mathbb{D}$ with vertical category $\dvert(\mathbb{D})=\cat{D}_0$ is a 2-fold monoid object in the 2-category $\mathcal{C}at/\cat{D}_0$ of categories over $\cat{D}_0$.
\end{definition}

Breaking this down, we have a category $\cat{D}_1$, a functor $p\colon\cat{D}_1\to\cat{D}_0$, two functors $\otimes, \odot\colon \cat{D}_1\times_{\cat{D}_0}\cat{D}_1\to\cat{D}_1$ commuting with $p$, and two functors $I,\perp\colon\cat{D}_0\to\cat{D}_1$ which are sections of $p$, such that $\otimes$, $\odot$, $I$, and $\perp$ satisfy all the axioms of a 2-fold monoidal category. In particular, each fiber of $p$ has a 2-fold monoidal structure.

This definition also implies that $\mathbb{D}$ has two underlying double categories, both with vertical category $\cat{D}_0$ and with source and target functors both equal to $p\colon\cat{D}_1\to\cat{D}_0$. The double category $\mathbb{D}_{\otimes}$ has the rest of the double category structure given by the functors $I$ and $\otimes$, while the double category $\mathbb{D}_{\odot}$ uses the functors $\perp$ and $\odot$.

Using this double category interpretation, we will find it convenient to think of a 2-fold double category as a double category with two different but interacting horizontal compositions. Notice that from this perspective, all horizontal 1-cells are endomorphisms.

For any object $C$ there are 2-cells
\begin{equation}\label{Eq:2FoldCoherenceCellsA}
\begin{tikzcd}[column sep=large]
	C \rar[tick][domA]{\perp_C \otimes \perp_C} \dar[equal] 
		& C \dar[equal] \\
	C \rar[tick][codA,swap]{\perp_C} 
		& C
	 \twocellA{m}
\end{tikzcd} \qquad
\begin{tikzcd}[column sep=large]
	C \rar[tick][domA]{I_C} \dar[equal] 
		& C \dar[equal] \\
	C \rar[tick][codA,swap]{I_C\odot I_C} 
		& C
	 \twocellA{c}
\end{tikzcd} \qquad
\begin{tikzcd}
	C \rar[tick][domA]{I_C} \dar[equal]
		& C \dar[equal] \\
	C \rar[tick][codA,swap]{\perp_C} 
		& C
	 \twocellA{j} 
\end{tikzcd}
\end{equation}
and for any four horizontal morphisms $\begin{tikzcd}[baseline,column sep=2.5ex] W,X,Y,Z\colon C \rar[tick]& C \end{tikzcd}$ there is a 2-cell
\begin{equation}\label{Eq:2FoldCoherenceCellsB}
\begin{tikzcd}[column sep=huge]
	C \rar[tick][domA]{(W\odot X)\otimes(Y\odot Z)} 
			\dar[equals] 
		& C \dar[equals] \\
	C \rar[tick][codA,swap]{(W\otimes Y)\odot(X\otimes Z)} 
		& C.
	\twocellA{z}
\end{tikzcd}
\end{equation}
These are natural in the sense that, for any vertical morphism $f\colon C\to D$ we have an equality
\[
\begin{tikzcd}[column sep=large]
	C \rar[tick][domA]{\perp_C \otimes \perp_C} 
			\dar[equal] 
		& C \dar[equal] \\
	C \rar[tick][codA,domB]{\perp_C} 
			\dar[swap]{f} 
		& C \dar{f} \\
	D \rar[tick][codB,swap]{\perp_D} 
		& D
	\twocellA{m}
	\twocellB{\perp_f}
\end{tikzcd}
=
\begin{tikzcd}[column sep=large]
	C \rar[tick][domA]{\perp_C \otimes \perp_C} 
			\dar[swap]{f} 
		& C \dar{f} \\
	D \rar[tick][codA,domB,swap]{\perp_D \otimes \perp_D} 
			\dar[equal] 
		& D \dar[equal] \\
	D \rar[tick][codB,swap]{\perp_D} 
		& D
	\twocellA{\perp_f \otimes \perp_f}
	\twocellB{m} 
\end{tikzcd}
\]
and similarly for $c$ and $j$, and for any four 2-cells $\theta_1,\dots,\theta_4$ of the apropriate form, we have an equality
\[
\begin{tikzcd}[column sep=17ex]
	C \rar[tick][domA]{(W\odot X)\otimes(Y\odot Z)} 
			\dar[equals] 
		& C \dar[equals] \\
	C \rar[tick][codA,domB]{(W\otimes Y)\odot(X\otimes Z)} 
			\dar[swap]{f} 
		& C \dar{f} \\
	D \rar[tick][codB,swap]{(W'\otimes Y')\odot(X'\otimes Z')} 
		& D
	\twocellA{z}
	\twocellB{(\theta_1\otimes\theta_3)\odot(\theta_2\otimes\theta_4)}
\end{tikzcd}
=
\begin{tikzcd}[column sep=17ex]
	C \rar[tick][domA]{(W\odot X)\otimes(Y\odot Z)} 
			\dar[swap]{f} 
		& C \dar{f} \\
	C \rar[tick][codA,domB,swap]{(W'\odot X')\otimes(Y'\odot Z')} 
			\dar[equals]  
		& C \dar[equals] \\
	D \rar[tick][codB,swap]{(W'\otimes Y')\odot(X'\otimes Z')} 
		& D
	\twocellA{(\theta_1\odot\theta_2)\otimes(\theta_3\odot\theta_4)}
	\twocellB{z}
\end{tikzcd}
\]

\begin{definition}
	A monad in a 2-fold double category $\mathbb{D}$ is a monad in $\mathbb{D}_{\otimes}$; a comonad in $\mathbb{D}$ is a comonad in $\mathbb{D}_{\odot}$. Furthermore, we define the categories $\mathrm{Mon}(\mathbb{D})=\mathrm{Mon}(\mathbb{D}_{\otimes})$ and $\mathrm{Comon}(\mathbb{D})=\mathrm{Comon}(\mathbb{D}_{\odot})$.
\end{definition}