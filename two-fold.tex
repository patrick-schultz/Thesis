% -*- root: thesis.tex -*-

\chapter{2-Fold double categories}

%In this section we will propose a generalization of the 2-fold monoidal categories as used in \cite{garner:soa}.

It is well known that the notion of bialgebra or bimonoid---an object with both monoid and comonoid structures which are compatible in a certain sense---makes sense not only in a symmetric monoidal category, but also in more general \emph{braided} monoidal categories. A bimonoid in a braided monoidal category $\cat{C}$ can be defined to be a monoid in the category of comonoids in $\cat{C}$, or equivalently as a comonoid in the category of monoids in $\cat{C}$. The braiding is necessary to ensure that the monoidal structure in $\cat{C}$ lifts to a product in $\Mon(\cat{C})$ and $\Comon(\cat{C})$.

Less well known is the fact that the definition of bimonoid works just as well in a more general context still: so-called 2-fold monoidal categories. A 2-fold monoidal category has two different monoidal structures, call them $(\otimes,I)$ and $(\odot,\perp)$, which are themselves compatible in certain sense. This compatibility can be stated in a way analagous to the definition of bimonoid given in the previous paragraph: a (strict) 2-fold monoidal category is a monoid object in the category $\Cat{StrMonCat}_l$ of strict monoidal categories and lax functors, or equivalently a monoid object in the category $\Cat{StrMonCat}_c$ of strict monoidal categories and colax functors. Notice that monoid objects in the category of strict monoidal categories and \emph{strong} monoidal functors (in which the components of the lax structure are isomorphisms) are precisely (strict) braided monoidal categories.

More concretely, the compatibility between the monoidal structures amounts to the existence of maps
\[
	m\colon \perp\otimes\perp\to\perp, \qquad c\colon I\to I\odot I, \qquad j\colon I\to \perp,
\]
making $(\perp,j,m)$ a $\otimes$-monoid and $(I,j,c)$ a $\odot$-comonoid, and a natural family of maps
\[
	z_{A,B,C,D}\colon (A\odot B)\otimes(C\odot D) \to (A\otimes C)\odot(B\otimes D)
\]
satisfying some coherence axioms.

\begin{example}\leavevmode
\begin{compactitem}
	\item Any braided monoidal category can be made into a 2-fold monoidal category in which the two monoidal structures coincide.
	\item Any monoidal category $(\cat{C},\otimes,I)$ with finite products has a 2-fold monoidal structure with $(\odot,\perp)$ given by the product and terminal object. Dually, a monoidal category $(\cat{C},\odot,\perp)$ with finite coproducts has a 2-fold monoidal structure with $(\otimes,I)$ given by the coproduct and initial object.
\end{compactitem}
\end{example}

Because the $\odot$-monoidal structure is lax monoidal with respect to the $\otimes$-monoidal structure, it lifts to the category $\Mon_{\otimes}(\cat{C})$ of $\otimes$-monoids in $\cat{C}$. Dually, the $\otimes$-monoidal structure lifts to the category $\Comon_{\odot}(\cat{C})$ of $\odot$-comonoids in $\cat{C}$. Thus, we could define the category of bimonoids in $\cat{C}$ to be either $\Comon_{\odot}(\Mon_{\otimes}(\cat{C}))$ or $\Mon_{\otimes}(\Comon_{\odot}(\cat{C}))$, and it turns out that these are canonically isomorphic. In either case, a bimonoid is an object $A$ with a $\otimes$-monoid structure $(\eta,\mu)$ and a $\odot$-comonoid structure $(\epsilon,\delta)$, such that the following four diagrams commute:
\begin{equation}
\begin{gathered}
\begin{tikzcd}[ampersand replacement=\&]
	I \rar{\eta} \dar[swap]{c} \& A \dar{\delta} \\
	I \odot I \rar[swap]{\eta\odot\eta} \& A\odot A,
\end{tikzcd}
\qquad
\begin{tikzcd}[ampersand replacement=\&]
	A\otimes A \rar{\mu} \dar[swap]{\epsilon\otimes\epsilon} \& A \dar{\epsilon} \\
	\perp\otimes\perp \rar[swap]{m} \& \perp,
\end{tikzcd}
\qquad
\begin{tikzcd}[column sep=small,ampersand replacement=\&]
	{} \& A \drar{\epsilon} \& \\
	I \urar{\eta} \ar{rr}[swap]{j} \&\& \perp,
\end{tikzcd}
\\
\begin{tikzcd}[ampersand replacement=\&]
	A\otimes A \ar{rr}{\mu} \dar[swap]{\delta\otimes\delta} \&\& A \dar{\delta} \\
	(A\odot A)\otimes(A\odot A) \rar[swap]{z_{A,A,A,A}}
		\& (A\otimes A)\odot(A\otimes A) \rar[swap]{\mu\odot\mu}
		\& A\odot A.
\end{tikzcd}
\end{gathered}
\end{equation}

We would now like to generalize this 2-fold monoidal category definition to double categories, where there are two different horizontal compositions which are compatible in a way analagous to the two monoidal structures in a 2-fold monoidal category. We will start with a concise formal definition, and then expand on the definition more concretely.

\begin{definition}
	A \emph{2-fold double category} $\mathbb{D}$ with vertical category $\dvert(\mathbb{D})=\cat{D}_0$ is a 2-fold monoid object in the 2-category $\mathcal{C}at/\cat{D}_0$ of categories over $\cat{D}_0$.
\end{definition}

Breaking this down, we have a category $\cat{D}_1$, a functor $p\colon\cat{D}_1\to\cat{D}_0$, two functors $\otimes, \odot\colon \cat{D}_1\times_{\cat{D}_0}\cat{D}_1\to\cat{D}_1$ commuting with $p$, and two functors $I,\perp\colon\cat{D}_0\to\cat{D}_1$ which are sections of $p$, such that $\otimes$, $\odot$, $I$, and $\perp$ satisfy all the axioms of a 2-fold monoidal category. In particular, each fiber of $p$ has a 2-fold monoidal structure.

A monoid object in $\mathcal{C}at/\dblcat{D}_0$ is equivalently a double category where the source and target functors $s,t\colon \dblcat{D}_1\to \dblcat{D}_0$ are equal, and with the vertical category $\dblcat{D}_0$. Conversely, any double category $\dblcat{D}$ in which all horizontal 1-cells have equal domain and codomain, and all 2-cells have equal vertical 1-cells as domain and codomain, is equivalently a monoid object in $\mathcal{C}at/\dblcat{D}_0$. We will alternate between these two descriptions as convenient.

Using this shift of perspective, $\mathbb{D}$ has two underlying double categories, both with vertical category $\cat{D}_0$ and with source and target functors both equal to $p\colon\cat{D}_1\to\cat{D}_0$. The double category $\mathbb{D}_{\otimes}$ has the rest of the double category structure given by the functors $I$ and $\otimes$, while the double category $\mathbb{D}_{\odot}$ uses the functors $\perp$ and $\odot$.

Using this double category interpretation, we will find it convenient to think of a 2-fold double category as a double category with two different but interacting horizontal compositions. Notice that from this perspective, all horizontal 1-cells are endomorphisms.

\begin{remark}
	It may seem somewhat ad hoc to force a 2-fold monoid object in a slice of $\twocat{Cat}$ into a double category mold, with the odd looking restriction to having only endomorphisms in the horizontal direction. We will make essential use of double functors from $\dblcat{D}_{\odot}$ and $\dblcat{D}_{\otimes}$ to genuine double categories (without the endomorphism restriction), and it is mostly for this reason that we have found the double categorical perspective useful, if perhaps only psychologically.

	We did give some thought to how one might define a 2-fold double category with non-endomorphism horizontal 1-cells and 2-cells, and while it seems like there might be a workable definition, it would require a very large increase in complexity. As we are mostly interested in the monads and comonads in a 2-fold double category, which are structures on endomorphism horizontal 1-cells, this restriction was of no concern to this work.
\end{remark}

Now let us explicitly look at the 2-fold monoidal structure from the double categorical perspective. For any object $C$ there are 2-cells
\begin{equation}\label{Eq:2FoldCoherenceCellsA}
\begin{tikzcd}[column sep=large]
	C \rar[tick][domA]{\perp_C \otimes \perp_C} \dar[equal] 
		& C \dar[equal] \\
	C \rar[tick][codA,swap]{\perp_C} 
		& C
	 \twocellA{m}
\end{tikzcd} \qquad
\begin{tikzcd}[column sep=large]
	C \rar[tick][domA]{I_C} \dar[equal] 
		& C \dar[equal] \\
	C \rar[tick][codA,swap]{I_C\odot I_C} 
		& C
	 \twocellA{c}
\end{tikzcd} \qquad
\begin{tikzcd}
	C \rar[tick][domA]{I_C} \dar[equal]
		& C \dar[equal] \\
	C \rar[tick][codA,swap]{\perp_C} 
		& C
	 \twocellA{j} 
\end{tikzcd}
\end{equation}
and for any four horizontal morphisms $\begin{tikzcd}[baseline,column sep=2.5ex] W,X,Y,Z\colon C \rar[tick]& C \end{tikzcd}$ there is a 2-cell
\begin{equation}\label{Eq:2FoldCoherenceCellsB}
\begin{tikzcd}[column sep=huge]
	C \rar[tick][domA]{(W\odot X)\otimes(Y\odot Z)} 
			\dar[equals] 
		& C \dar[equals] \\
	C \rar[tick][codA,swap]{(W\otimes Y)\odot(X\otimes Z)} 
		& C.
	\twocellA{z}
\end{tikzcd}
\end{equation}
These are natural in the sense that, for any vertical morphism $f\colon C\to D$ we have an equality
\[
\begin{tikzcd}[column sep=large]
	C \rar[tick][domA]{\perp_C \otimes \perp_C} 
			\dar[equal] 
		& C \dar[equal] \\
	C \rar[tick][codA,domB]{\perp_C} 
			\dar[swap]{f} 
		& C \dar{f} \\
	D \rar[tick][codB,swap]{\perp_D} 
		& D
	\twocellA{m}
	\twocellB{\perp_f}
\end{tikzcd}
=
\begin{tikzcd}[column sep=large]
	C \rar[tick][domA]{\perp_C \otimes \perp_C} 
			\dar[swap]{f} 
		& C \dar{f} \\
	D \rar[tick][codA,domB,swap]{\perp_D \otimes \perp_D} 
			\dar[equal] 
		& D \dar[equal] \\
	D \rar[tick][codB,swap]{\perp_D} 
		& D
	\twocellA{\perp_f \otimes \perp_f}
	\twocellB{m} 
\end{tikzcd}
\]
and similarly for $c$ and $j$, and for any four 2-cells $\theta_1,\dots,\theta_4$ of the apropriate form, we have an equality
\[
\begin{tikzcd}[column sep=17ex]
	C \rar[tick][domA]{(W\odot X)\otimes(Y\odot Z)} 
			\dar[equals] 
		& C \dar[equals] \\
	C \rar[tick][codA,domB]{(W\otimes Y)\odot(X\otimes Z)} 
			\dar[swap]{f} 
		& C \dar{f} \\
	D \rar[tick][codB,swap]{(W'\otimes Y')\odot(X'\otimes Z')} 
		& D
	\twocellA{z}
	\twocellB{(\theta_1\otimes\theta_3)\odot(\theta_2\otimes\theta_4)}
\end{tikzcd}
=
\begin{tikzcd}[column sep=17ex]
	C \rar[tick][domA]{(W\odot X)\otimes(Y\odot Z)} 
			\dar[swap]{f} 
		& C \dar{f} \\
	C \rar[tick][codA,domB,swap]{(W'\odot X')\otimes(Y'\odot Z')} 
			\dar[equals]  
		& C \dar[equals] \\
	D \rar[tick][codB,swap]{(W'\otimes Y')\odot(X'\otimes Z')} 
		& D
	\twocellA{(\theta_1\odot\theta_2)\otimes(\theta_3\odot\theta_4)}
	\twocellB{z}
\end{tikzcd}
\]

\begin{definition}
	A monad in a 2-fold double category $\mathbb{D}$ is a monad in $\mathbb{D}_{\otimes}$; a comonad in $\mathbb{D}$ is a comonad in $\mathbb{D}_{\odot}$. Furthermore, we define the categories $\mathrm{Mon}(\mathbb{D})=\mathrm{Mon}(\mathbb{D}_{\otimes})$ and $\mathrm{Comon}(\mathbb{D})=\mathrm{Comon}(\mathbb{D}_{\odot})$.
\end{definition}

So a monad $X$ and a comonad $Y$ in $\mathbb{D}$ are given by 2-cells
\[
\begin{tikzcd}
	C \rar[tick][domA]{I_C} \dar[equal]
		& C \dar[equal] \\
	C \rar[tick][codA,swap]{X} 
		& C
	\twocellA{\eta}
\end{tikzcd} \qquad
\begin{tikzcd}
	C \rar[tick][domA]{X\otimes X} \dar[equal]
		& C \dar[equal] \\
	C \rar[tick][codA,swap]{X} 
		& C
	\twocellA{\mu} 
\end{tikzcd} \qquad
\begin{tikzcd}
	C \rar[tick][domA]{X} \dar[equal]  
		& C \dar[equal] \\
	C \rar[tick][codA,swap]{\perp_C} 
		& C
	\twocellA{\epsilon}
\end{tikzcd} \qquad
\begin{tikzcd}
	C \rar[tick][domA]{X} \dar[equal] 
		& C \dar[equal] \\
	C \rar[tick][codA,swap]{X\odot X} 
		& C.
	\twocellA{\delta} 
\end{tikzcd}
\]

The categories $\Mon(\mathbb{D})$ and $\Comon(\mathbb{D})$ come naturally equipped with functors to $\mathcal{D}_0$, defined on objects and morphisms simply by applying $p$ to the underlying 1-cells and 2-cells respectively. It turns out that the interaction between the $\otimes$ and $\odot$ compositions in the 2-fold double category structure is precisely what is needed to lift $\odot$ to $\Mon(\mathbb{D})$ and to lift $\otimes$ to $\Comon(\mathbb{D})$. In this way, we can define double categories $\DMon(\mathbb{D})$ and $\DComon(\mathbb{D})$, both having $\mathcal{D}_0$ as vertical category.

These lifted compositions are defined as follows: Given two monads $(C,X,\eta,\mu)$ and $(C,Y,\eta',\mu')$ in $\mathbb{D}$, the horizontal composition 
\[
\begin{tikzcd}[column sep=large]
	C \rar[tick]{(X,\eta,\mu)} & C \rar[tick]{(Y,\eta',\mu')} & C
\end{tikzcd}
\]
is the monoid with underlying horizontal 1-cell $X\odot Y$ and unit and multiplication 2-cells
\[
\begin{tikzcd}[column sep=7ex]
	C \rar[tick][domA]{I_C} \dar[equal] 
		& C \dar[equal] \\
	C \rar[tick][codA,domB]{I_C\odot I_C} 
			\dar[equal] 
		& C \dar[equal] \\
	C \rar[tick][codB,swap]{X\odot Y}
		& C
	\twocellA{c}
	\twocellB{\eta\odot\eta'}
\end{tikzcd}
\qquad
\begin{tikzcd}[column sep=14ex]
	C \rar[tick][domA]{(X\odot Y)\otimes(X\odot Y)} 
			\dar[equal] 
		& C \dar[equal] \\
	C \rar[tick][codA,domB]{(X\otimes X)\odot(Y\otimes Y)} 
			\dar[equal] 
		& C \dar[equal] \\
	C \rar[tick][codB,swap]{X\odot Y}
		& C.
	\twocellA{z}
	\twocellB{\mu\odot\mu'}
\end{tikzcd}
\]
The unit for this composition is $I_C$, given the trivial monad structure with $\eta=\mu=\id_{I_C}$.

Similarly, the horizontal composition of two 2-cells in $\DMon(\mathbb{D})$ is the $\odot$ product of the underlying 2-cells in $\mathbb{D}$. The fact that this commutes with the unit and multiplication defined above follows from the naturality of $c$ and $z$.

In this same way, we can define the horizontal composition of two 1-cells $(X,\epsilon,\delta)$ and $(Y,\epsilon',\delta')$ in $\DComon(\mathbb{D})$ to be a comonad with underlying horizontal 1-cell $X\otimes Y$, with horizontal unit $\perp$ with the trivial comonad structure.

This allows us to define (ordinary) categories $\Mon(\DComon(\mathbb{D}))$ and $\Comon(\DMon(\mathbb{D}))$. Furthermore, these two categories are equivalent, leading to the next definition.

\begin{definition}\label{Def:Bimonad}
	A \emph{bimonad} in a 2-fold double category $\mathbb{D}$ is a monad in $\DComon(\mathbb{D})$, or equivalently a comonad in $\DMon(\mathbb{D})$. We can define a category of bimonads in $\mathbb{D}$ as
	\[
		\Bimon(\mathbb{D}) := \Mon(\DComon(\mathbb{D})) \simeq \Comon(\DMon(\mathbb{D}))
	\]
\end{definition}

Concretely, a bimonad in $\mathbb{D}$ is a tuple $(X,\eta,\mu,\epsilon,\delta)$ where $X$ is a horizontal 1-cell, $(X,\eta,\mu)$ is a monad and $(X,\epsilon,\delta)$ is a comonad as above, such that four equations hold:
\begin{equation}\label{Eq:Bimonoid}
\begin{gathered}
\begin{tikzcd}[ampersand replacement=\&]
	C \rar[tick][domA]{I_C} \dar[equal] 
		\& C \dar[equal] \\
	C \rar[tick][codA,domB]{X} \dar[equal] 
		\& C \dar[equal] \\
	C \rar[tick][codB,swap]{X\odot X} \& C
	\twocellA{\eta}
	\twocellB{\delta}
\end{tikzcd}
=
\begin{tikzcd}[ampersand replacement=\&]
	C \rar[tick][domA]{I_C} \dar[equal] 
		\& C \dar[equal] \\
	C \rar[tick][codA,domB]{I_C\odot I_C} \dar[equal] 
		\& C \dar[equal] \\
	C \rar[tick][codB,swap]{X\odot X} \& C
	\twocellA{c}
	\twocellB{\eta\odot\eta}
\end{tikzcd}
\qquad
\begin{tikzcd}[ampersand replacement=\&]
	C \rar[tick][domA]{X\otimes X} \dar[equal] 
		\& C \dar[equal] \\
	C \rar[tick][codA,domB]{X} \dar[equal] 
		\& C \dar[equal] \\
	C \rar[tick][codB,swap]{\perp_C} \& C
	\twocellA{\mu}
	\twocellB{\epsilon}
\end{tikzcd}
=
\begin{tikzcd}[ampersand replacement=\&]
	C \rar[tick][domA]{X\otimes X} \dar[equal] 
		\& C \dar[equal] \\
	C \rar[tick][codA,domB]{\perp_C\otimes\perp_C} \dar[equal] 
		\& C \dar[equal] \\
	C \rar[tick][codB,swap]{\perp_C} \& C
	\twocellA{\epsilon\otimes\epsilon}
	\twocellB{m}
\end{tikzcd}
\\
\begin{tikzcd}[ampersand replacement=\&]
	C \rar[tick][domA]{I_C} \dar[equal] 
		\& C \dar[equal] \\
	C \rar[tick][codA,domB]{X} \dar[equal] 
		\& C \dar[equal] \\
	C \rar[tick][codB,swap]{\perp_C} \& C
	\twocellA{\eta}
	\twocellB{\epsilon}
\end{tikzcd}
=
\begin{tikzcd}[ampersand replacement=\&]
	C \rar[tick][domA]{I_C} \dar[equal] 
		\& C \dar[equal] \\
	C \rar[tick][codA,swap]{\perp_C} \& C
	\twocellA{j}
\end{tikzcd}
\qquad
\begin{tikzcd}[ampersand replacement=\&,column sep=14ex]
	C \rar[tick][domA]{X\otimes X} \dar[equal] 
		\& C \dar[equal] \\
	C \rar[tick][codA,domB]{(X\odot X)\otimes(X\odot X)} \dar[equal] 
		\& C \dar[equal] \\
	C \rar[tick][codB,domC]{(X\otimes X)\odot(X\otimes X)} \dar[equal] 
		\& C \dar[equal] \\
	C \rar[tick][codC,swap]{X\odot X} \& C
	\twocellA{\delta\otimes\delta}
	\twocellB{z}
	\twocellC{\mu\odot\mu}
\end{tikzcd}
=
\begin{tikzcd}[ampersand replacement=\&]
	C \rar[tick][domA]{X\otimes X} \dar[equal]
		\& C \dar[equal] \\
	C \rar[tick][codA,domB]{X} \dar[equal] 
		\& C \dar[equal] \\
	C \rar[tick][codB,swap]{X\odot X} \& C
	\twocellA{\mu}
	\twocellB{\delta}
\end{tikzcd}
\end{gathered}
\end{equation}
A bimonoid morphism is simply a 2-cell which is simultaineously a monoid morphism and a comonoid morphism.