% -*- root: thesis.tex -*-

\chapter{2-Fold Double Categories}

In this section we will propose a generalization of the 2-fold monoidal categories as used in \cite{garner:soa}.

A 2-fold double category $\mathbb{D}$ is a structure which has two different underlying double categories, both of which have the same vertical category $\dvert(\mathbb{D})$. We will start with a concise formal definition, and then expand on the definition more concretely.

\begin{definition}
	A \emph{2-fold double category} $\mathbb{D}$ with vertical category $\dvert(\mathbb{D})=\cat{D}_0$ is a 2-fold monoid object in the 2-category $\mathcal{C}at/\cat{D}_0$ of categories over $\cat{D}_0$.
\end{definition}

Breaking this down, we have a category $\cat{D}$, a functor $p\colon\cat{D}\to\cat{D}_0$, two functors $\otimes, \odot\colon \cat{D}\times_{\cat{D}_0}\cat{D}\to\cat{D}$ commuting with $p$, and two functors $I,\perp\colon\cat{D}_0\to\cat{D}$ which are sections of $p$, such that $\otimes$, $\odot$, $I$, and $\perp$ satisfy all the axioms of a 2-fold monoidal category. In particular, each fiber of $p$ has a 2-fold monoidal structure.

We will find it convenient to present this structure in the form of a double category $\mathbb{D}$, as follows: 
\begin{itemize}
	\item The objects and vertical morphisms of $\mathbb{D}$ are those of $\cat{D}_0$, so that $\dvert(\mathbb{D})=\cat{D}_0$.
	\item The horizontal morphisms of $\mathbb{D}$ are the objects $X$ of $\cat{D}$, with $p(X)$ as both domain and codomain. We will draw these as marked arrows 
	\[
	\begin{tikzcd}
		p(X) \rar[tick]{X} & p(X).
	\end{tikzcd}
	\]
	\item The 2-cells are the morphisms of $\cat{D}$. So a morphism $\phi\colon X\to Y$ in $\cat{D}$ with $p(\phi)=f\colon C\to D$ would be drawn as
	\[\begin{tikzcd}
		C \rar[tick][domA]{X} \dar[swap]{f} 
			& C \dar{f} \\
		D \rar[tick][codA,swap]{Y} 
			& D
		 \twocellA{\phi}
	\end{tikzcd}\]
\end{itemize}
The two tensor products of $\cat{D}$ provide two different horizontal compositions for $\mathbb{D}$. For any object $C$ there are 2-cells
\begin{equation}\label{Eq:2FoldCoherenceCellsA}
\begin{tikzcd}[column sep=large]
	C \rar[tick][domA]{\perp_C \otimes \perp_C} \dar[equal] 
		& C \dar[equal] \\
	C \rar[tick][codA,swap]{\perp_C} 
		& C
	 \twocellA{m}
\end{tikzcd} \qquad
\begin{tikzcd}[column sep=large]
	C \rar[tick][domA]{I_C} \dar[equal] 
		& C \dar[equal] \\
	C \rar[tick][codA,swap]{I_C\odot I_C} 
		& C
	 \twocellA{c}
\end{tikzcd} \qquad
\begin{tikzcd}
	C \rar[tick][domA]{I_C} \dar[equal]
		& C \dar[equal] \\
	C \rar[tick][codA,swap]{\perp_C} 
		& C
	 \twocellA{j} 
\end{tikzcd}
\end{equation}
and for any four horizontal morphisms $\begin{tikzcd}[baseline,column sep=2.5ex] W,X,Y,Z\colon C \rar[tick]& C \end{tikzcd}$ there is a 2-cell
\begin{equation}\label{Eq:2FoldCoherenceCellsB}
\begin{tikzcd}[column sep=huge]
	C \rar[tick][domA]{(W\odot X)\otimes(Y\odot Z)} 
			\dar[equals] 
		& C \dar[equals] \\
	C \rar[tick][codA,swap]{(W\otimes Y)\odot(X\otimes Z)} 
		& C.
	\twocellA{z}
\end{tikzcd}
\end{equation}
These are natural in the sense that, for any vertical morphism $f\colon C\to D$ we have an equality
\[
\begin{tikzcd}[column sep=large]
	C \rar[tick][domA]{\perp_C \otimes \perp_C} 
			\dar[equal] 
		& C \dar[equal] \\
	C \rar[tick][codA,domB]{\perp_C} 
			\dar[swap]{f} 
		& C \dar{f} \\
	D \rar[tick][codB,swap]{\perp_D} 
		& D
	\twocellA{m}
	\twocellB{\perp_f}
\end{tikzcd}
=
\begin{tikzcd}[column sep=large]
	C \rar[tick][domA]{\perp_C \otimes \perp_C} 
			\dar[swap]{f} 
		& C \dar{f} \\
	D \rar[tick][codA,domB,swap]{\perp_D \otimes \perp_D} 
			\dar[equal] 
		& D \dar[equal] \\
	D \rar[tick][codB,swap]{\perp_D} 
		& D
	\twocellA{\perp_f \otimes \perp_f}
	\twocellB{m} 
\end{tikzcd}
\]
and similarly for $c$ and $j$, and for any four 2-cells $\theta_1,\dots,\theta_4$ of the apropriate form, we have an equality
\[
\begin{tikzcd}[column sep=17ex]
	C \rar[tick][domA]{(W\odot X)\otimes(Y\odot Z)} 
			\dar[equals] 
		& C \dar[equals] \\
	C \rar[tick][codA,domB]{(W\otimes Y)\odot(X\otimes Z)} 
			\dar[swap]{f} 
		& C \dar{f} \\
	D \rar[tick][codB,swap]{(W'\otimes Y')\odot(X'\otimes Z')} 
		& D
	\twocellA{z}
	\twocellB{(\theta_1\otimes\theta_3)\odot(\theta_2\otimes\theta_4)}
\end{tikzcd}
=
\begin{tikzcd}[column sep=17ex]
	C \rar[tick][domA]{(W\odot X)\otimes(Y\odot Z)} 
			\dar[swap]{f} 
		& C \dar{f} \\
	C \rar[tick][codA,domB,swap]{(W'\odot X')\otimes(Y'\odot Z')} 
			\dar[equals]  
		& C \dar[equals] \\
	D \rar[tick][codB,swap]{(W'\otimes Y')\odot(X'\otimes Z')} 
		& D
	\twocellA{(\theta_1\odot\theta_2)\otimes(\theta_3\odot\theta_4)}
	\twocellB{z}
\end{tikzcd}
\]

A 2-fold double category $\mathbb{D}$ has two underlying ordinary double categories: $\mathbb{D}_{\otimes}$ using $\otimes$ for the horizontal composition, and $\mathbb{D}_{\odot}$ using $\odot$.