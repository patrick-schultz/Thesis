% -*- root: thesis.tex -*-

\chapter{CYCLIC 2-FOLD DOUBLE CATEGORIES}\label{Ch:Cyclic}

Recall the notion of a cyclic double category from~\cite{cgr:mates}. A cyclic double category $\mathbb{D}$ is a double category with an extra involutive operation. On objects and horizontal 1-cells $\begin{tikzcd}[baseline,column sep=2.5ex] X\colon C \rar[tick]& C \end{tikzcd}$, this operation is written
\[
\begin{tikzcd}
	C_1^{\bullet} \rar[tick][]{X^{\bullet}} & C_2^{\bullet}
\end{tikzcd}
\]
and respects horizontal identities and composition. The involution takes any vertical 1-cell $f\colon C\to D$ to some $\sigma f\colon D^{\bullet}\to C^{\bullet}$, and any 2-cell
\[
\begin{tikzcd}
	C_1 \rar[tick][domA]{X} \dar[swap]{f} 
		& C_2 \dar{g} \\
	D_1 \rar[tick][codA,swap]{Y}
		& D_2
	\twocellA{\theta}
\end{tikzcd}
\qquad\text{to}\qquad
\begin{tikzcd}
	D_1^{\bullet} \rar[tick][domA]{Y^{\bullet}} 
			\dar[swap]{\sigma f} 
		& D_2^{\bullet} \dar{\sigma g}\\
	C_1^{\bullet} \rar[tick][codA,swap]{X^{\bullet}}
		& C_2^{\bullet}
	\twocellA{\sigma\theta}
\end{tikzcd}
\]
respecting vertical identities and composition.

The next example is the fundamental example of a cyclic double category.

\begin{example}\label{Ex:CyclicLAdj}
	Recall the double categories $\LAdj(\twocat{D})$ and $\RAdj(\twocat{D})$ from~\cref{Ex:DoubleLAdj}.	If the 2-category $\twocat{D}$ has an involution $(-)^{\bullet}\colon\twocat{D}^{\text{co}}\to\twocat{D}$, such as $\twocat{C}at$ with $(-)^{\text{op}}$, then the double category $\LAdj(\twocat{D})$ has a natural cyclic action: on vertical 1-cells $\sigma (f\dashv g) = (g^{\bullet}\dashv f^{\bullet})$, and if $\phi$ is a 2-cell
	\[
	\begin{tikzcd}
		C \rar[tick][domA]{j} \dar[swap]{(f\dashv g)} 
			& C' \dar{(f'\dashv g')} \\
		D \rar[tick][codA,swap]{k} 
			& D'
		\twocellA{\phi}
	\end{tikzcd}
	\]
	with mate $\theta$ then $\sigma\phi=\theta^{\bullet}$:
	\[
	\begin{tikzcd}
		D^{\bullet} \rar[tick][domA]{k^{\bullet}} \dar[swap]{(g^{\bullet}\dashv f^{\bullet})} 
			& {D'}^{\bullet} \dar{({g'}^{\bullet}\dashv {f'}^{\bullet})} \\
		C^{\bullet} \rar[tick][codA,swap]{j^{\bullet}} 
			& {C'}^{\bullet}
		\twocellA{\theta^{\bullet}}
	\end{tikzcd}
	\]
	This cyclic action encodes the naturality of the mates correspondence using only a single double category, and is a convenient alternative to the isomorphism $\LAdj(\twocat{D})\cong\RAdj(\twocat{D})$. This simplification will be even more important when we need the multivariable mates correspondence in~\cref{Ch:PushoutProduct,Ch:DblMulti}.

	For a clear summary of the mates correspondence and the cyclic action on $\LAdj$, see~\cite{cgr:mates} Section 1. 
\end{example}

\begin{proposition}
	Let $\dblcat{D}$ be a cyclic double category with arrow objects. For any object $C$, $(C^{\bullet})^2=(C^2)^{\bullet}$, as witnessed by
	\[
	\begin{tikzcd}[bend angle=30]
	(C^2)^{\bullet} \rar[bend left,tick][domA]{\cod^{\bullet}}
		\rar[bend right,tick][codA,swap]{\dom^{\bullet}}
	& C^{\bullet}
	\twocellA{\sigma\kappa}
	\end{tikzcd}
	\]
	For any vertical 1-cell $F$, the lift to arrow objects of $F^{\bullet}$ is $(\hat{F})^{\bullet}$, as witnessed by the 2-cells
	\[
	\begin{tikzcd}
		(D^2)^{\bullet} \rar[tick][domA]{\cod^{\bullet}} \dar[swap]{(\hat{F})^{\bullet}} & D^{\bullet} \dar{F^{\bullet}} \\
		(C^2)^{\bullet} \rar[tick][codA,swap]{\cod^{\bullet}} & C^{\bullet}
		\twocellA{\sigma\gamma_0}
	\end{tikzcd}
	\qquad
	\begin{tikzcd}
		(D^2)^{\bullet} \rar[tick][domA]{\dom^{\bullet}} \dar[swap]{(\hat{F})^{\bullet}} & D^{\bullet} \dar{F^{\bullet}} \\
		(C^2)^{\bullet} \rar[tick][codA,swap]{\dom^{\bullet}} & C^{\bullet}
		\twocellA{\sigma\gamma_1}
	\end{tikzcd}
	\]
\end{proposition}
\begin{proof}
	It is a very simple matter to verify the universal properties of~\cref{Sec:ArrowObjects}
\end{proof}

We will generalize this to a cyclic action on a 2-fold double category. Suppose that $\mathbb{D}$ is a 2-fold double category. A cyclic action, written as above, must satisfy the following:
\begin{itemize}
	\item For every object $C$,
	\[
		I_{C^{\bullet}}=(\perp_C)^{\bullet} \qquad\text{and}\qquad \perp_{C^{\bullet}}=(I_C)^{\bullet}.
	\]
	\item For every composable pair of horizontal 1-cells $\begin{tikzcd}[baseline,column sep=2.5ex] X,Y\colon C \rar[tick]& C \end{tikzcd}$,
	\[
		(X\otimes Y)^{\bullet} = X^{\bullet}\odot Y^{\bullet} \qquad\text{and}\qquad
		(X\odot Y)^{\bullet} = X^{\bullet}\otimes Y^{\bullet}
	\]
	\item For every vertical 1-cell $f\colon C\to D$, there are equalities
	\[
	\begin{tikzcd}
		D^{\bullet} \rar[tick][codA]{I_{D^{\bullet}}} 
				\dar[swap]{\sigma f} 
			& D^{\bullet} \dar{\sigma f} \\
		C^{\bullet} \rar[tick][codA,swap]{I_{C^{\bullet}}}
			& C^{\bullet}
		\twocellA{I_{\sigma f}}
	\end{tikzcd}
	=
	\begin{tikzcd}
		D^{\bullet} \rar[tick][domA]{(\perp_D)^{\bullet}} 
				\dar[swap]{\sigma f} 
			& D^{\bullet} \dar{\sigma f} \\
		C^{\bullet} \rar[tick][codA,swap]{(\perp_C)^{\bullet}}
			& C^{\bullet}
		\twocellA{\sigma\perp_f}
	\end{tikzcd}
	\]
	\[
	\begin{tikzcd}
		D^{\bullet} \rar[tick][domA]{\perp_{D^{\bullet}}} 
				\dar[swap]{\sigma f} 
			& D^{\bullet} \dar{\sigma f} \\
		C^{\bullet} \rar[tick][codA,swap]{\perp_{C^{\bullet}}}
			& C^{\bullet}
		\twocellA{\perp_{\sigma f}}
	\end{tikzcd}
	=
	\begin{tikzcd}
		D^{\bullet} \rar[tick][domA]{(I_D)^{\bullet}} 
				\dar[swap]{\sigma f} 
			& D^{\bullet} \dar{\sigma f} \\
		C^{\bullet} \rar[tick][codA,swap]{(I_C)^{\bullet}}
			& C^{\bullet}
		\twocellA{\sigma I_f}
	\end{tikzcd}
	\]
	\item For every horizontally composable pair of 2-cells
	\[
	\begin{tikzcd}
		C \rar[tick][domA]{X} \dar[swap]{f} 
			& C \rar[tick][domB]{Y} \dar{f} 
			& C \dar{f} \\
		D \rar[tick][codA,swap]{X'}
			& D \rar[tick][codB,swap]{Y'}
			& D
		\twocellA{\theta}
		\twocellB{\phi}
	\end{tikzcd}
	\]
	there are equalities
	\[
	\begin{tikzcd}[column sep=large]
		D^{\bullet} \rar[tick][domA]{(X'\otimes Y')^{\bullet}} 
				\dar[swap]{\sigma f} 
			& D^{\bullet} \dar{\sigma f}\\
		C^{\bullet} \rar[tick][codA,swap]{(X\otimes Y)^{\bullet}}
			& C^{\bullet}
		\twocellA{\sigma(\theta\otimes\phi)}
	\end{tikzcd}
	=
	\begin{tikzcd}[column sep=large]
		D^{\bullet} \rar[tick][domA]{X'^{\bullet}\odot Y'^{\bullet}} 
				\dar[swap]{\sigma f} 
			& D^{\bullet} \dar{\sigma f}\\
		C^{\bullet} \rar[tick][codA,swap]{X^{\bullet}\odot Y^{\bullet}}
			& C^{\bullet}
		\twocellA{\sigma(\theta)\odot\sigma(\phi)}
	\end{tikzcd}
	\]
	\[
	\begin{tikzcd}[column sep=large]
		D^{\bullet} \rar[tick][domA]{(X'\odot Y')^{\bullet}} 
				\dar[swap]{\sigma f} 
			& D^{\bullet} \dar{\sigma f}\\
		C^{\bullet} \rar[tick][codA,swap]{(X\odot Y)^{\bullet}}
			& C^{\bullet}
		\twocellA{\sigma(\theta\odot\phi)}
	\end{tikzcd}
	=
	\begin{tikzcd}[column sep=large]
		D^{\bullet} \rar[tick][domA]{X'^{\bullet}\otimes Y'^{\bullet}} 
				\dar[swap]{\sigma f} 
			& D^{\bullet} \dar{\sigma f}\\
		C^{\bullet} \rar[tick][codA,swap]{X^{\bullet}\otimes Y^{\bullet}}
			& C^{\bullet}
		\twocellA{\sigma(\theta)\otimes\sigma(\phi)}
	\end{tikzcd}
	\]
\end{itemize}

One nice consequence of this definition is that a cyclic action on a 2-fold double category $\mathbb{D}$ induces a cyclic action on the category of bimonads $\mathrm{Bimon}(\mathbb{D})$.

\begin{proposition}\label{Prop:BimonCyclic}
	Suppose $\mathbb{D}$ is a cyclic 2-fold double category. Then the category $\mathrm{Bimon}(\mathbb{D})$ of bimonads in $\mathbb{D}$ carries a natural cyclic action (contravariant isomorphism).
\end{proposition}
\begin{proof}
	The involution $(-)^{\bullet}$ gives an isomorphism of double categories $\mathbb{D}_{\otimes}\cong\mathbb{D}_{\odot}^{\mathrm{op}}$. Therefore it also induces an isomorphism
\[
	\mathbb{M}\mathrm{on}(\mathbb{D}) = \mathbb{M}\mathrm{on}(\mathbb{D}_{\otimes}) 
		\cong \mathbb{M}\mathrm{on}(\mathbb{D}_{\odot}^{\mathrm{op}}) 
		\cong \mathbb{C}\mathrm{omon}(\mathbb{D}_{\odot})^{\mathrm{op}} 
		= \mathbb{C}\mathrm{omon}(\mathbb{D})^{\mathrm{op}}
\]
as well as an isomorphism
\begin{multline*}
	\mathrm{Bimon}(\mathbb{D}) = \mathrm{Comon}(\mathbb{M}\mathrm{on}(\mathbb{D}))
		\cong \mathrm{Comon}(\mathbb{C}\mathrm{omon}(\mathbb{D})^{\mathrm{op}})
		\\ \cong \mathrm{Mon}(\mathbb{C}\mathrm{omon}(\mathbb{D}))^{\mathrm{op}}
		= \mathrm{Bimon}(\mathbb{D})^{\mathrm{op}}.
\end{multline*}
\end{proof}

In more concrete terms, the involution takes a bimonad $(X,\eta,\mu,\epsilon,\delta)$ to $(X,\eta,\mu,\epsilon,\delta)^{\bullet}=(X^\bullet,\epsilon^\bullet,\delta^\bullet,\eta^\bullet,\delta^\bullet)$, swapping the monad and comonad structures. This is again a bimonad, as the top two equations of~\eqref{Eq:Bimonoid} are interchanged under the involution, while the bottom two equations are self-dual.

The action of the involution on bimonad morphisms can be broken down as in the following lemma.

\begin{lemma}\label{Lem:MonMorphismDuality}
	Let $(X,\eta,\mu,\epsilon,\delta)$ and $(Y,\eta',\mu',\epsilon',\delta')$ be bimonads in a cyclic 2-fold double category $\mathbb{D}$, and let $\phi$ be a 2-cell in $\mathbb{D}$
	\[
	\begin{tikzcd}
		C \rar[tick][domA]{X} \dar[swap]{f} & C \dar{f} \\
		D \rar[tick][codA,swap]{Y} & D.
		\twocellA{\phi}
	\end{tikzcd}
	\]
	Then $(f,\phi)$ is a monad morphism $X\to Y$ if and only if $(\sigma f,\phi^\bullet)$ is a comonad morphism $Y^\bullet\to X^\bullet$. Dually, $\phi$ is a comonad morphism $X\to Y$ if and only if $\phi^\bullet$ is a monad morphism $Y^\bullet\to X^\bullet$.
\end{lemma}
\begin{proof}
	Simply notice that the involution takes equations~\eqref{Eq:MonoidMorphismUnit} and~\eqref{Eq:MonoidMorphismMult} to the equations defining a comonad morphism in $\mathbb{D}$.
\end{proof}

This immediately implies a useful characterization of bimonoid morphisms.

\begin{corollary}\label{Cor:BimonMorphism}
	Given bimonads $(X,\eta,\mu,\epsilon,\delta)$ and $(Y,\eta',\mu',\epsilon',\delta')$ in a cyclic 2-fold double category $\mathbb{D}$, a bimonad morphism $X\to Y$ consists of a pair $(f,\phi)$ as above, such that:
	\begin{itemize}
		\item \emph{Either} $(f,\phi)$ is a monad morphism \emph{or} $(\sigma f,\phi^{\bullet})$ is a comonad morphism, \emph{and}
		\item \emph{Either} $(f,\phi)$ is a comonad morphism \emph{or} $(\sigma f,\phi^{\bullet})$ is a monad morphism.
	\end{itemize}
\end{corollary}