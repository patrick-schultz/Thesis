% -*- root: thesis.tex -*-

\chapter{A universal property for the pushout product}\label{Ch:PushoutProduct} 

We would now like to generalize the framework of cyclic 2-fold double categories to cyclic 2-fold double multicategories, in order to incorporate the multivariable adjunctions of awfs defined in \cite{riehl:nwfs-monoidal}.

[TODO: Review cyclic multicategories/double multicategories]

First, we will need to find a generalization of the universal property for arrow objects to (cyclic) double multicategories. In the theory of multivariable Quillen adjunctions, the lift of a multivariable adjunction to arrow categories is provided by the pushout/pullback product, so we will identify a universal property satisfied by this construction.

Define a cyclic double multicategory $\mathbb{J}$ as follows. The objects are $A_i$, $B_i$, for $i\in\{0,1,2\}$, and their duals. The horizontal 1-cells are $d^i_0,d^i_1\colon B_i\to A_i$. The vertical 1-cells are $F_i\colon (A_{i-1},A_{i+1})\to A^{\bullet}_i$ and $G_i\colon (B_{i-1},B_{i+1})\to B^{\bullet}_i$, which form two orbits under the cyclic action.

There are two types of 2-cells. There are
\[\begin{tikzcd}
	B_i \rar[][domA]{d^i_1} \dar[swap]{\id}
		& A_i \dar{\id} \\
	B_i \rar[][codA,swap]{d^i_0}
		& A_i 
	\twocellA{\alpha_i}
\end{tikzcd}\]
for each $i$. We will often draw these 2-cells globularly.

There are also 2-cells
\[\begin{tikzcd}[column sep=large]
	B_{i+1},B_{i-1} \rar[][domA]{d^{i+1}_{k_{i+1}},d^{i-1}_{k_{i-1}}} 
			\dar[swap]{G_i} 
		& A_{i+1},A_{i-1} \dar{F_i} \\
	B^{\bullet}_i \rar[][codA,swap]{d^{i\bullet}_{k_i}} 
		& A^{\bullet}_i
	\twocellA{\lambda^i_{k_{i+1},k_{i-1},k_{i}}}
\end{tikzcd}\]
for all choices of $(k_0,k_1,k_2)\in\{0,1\}^3$ except $(0,0,0)$.

Notice that there is at most one element of every hom-set, so all compositions and cyclic actions are uniquely defined. From now on, we will omit indices whenever doing so is unambiguous.

\begin{remark}
The cyclic double multicategory $\mathbb{J}$ is generated under composition by the $\alpha_i$ and the $\lambda^i_{k_{i+1},k_{i-1},k_{i}}$ with exactly one of $k_0,k_1,k_2$ equal to 1. These nine $\lambda$ generators are further generated under the cyclic action by only three, though there are many choices of which three. These generators satisfy the relations

\begin{gather*}
\begin{tikzcd}[bend angle=50,ampersand replacement=\&]
	B_1,B_2 \rar[][domA]{d_1,d_0} 
			\dar[swap]{G_0}
		\& A_1,A_2 \dar{F_0} \\
	B^{\bullet}_0 \rar[][codA,domB]{d^{\bullet}_0}
			\rar[bend right][codB,swap]{d^{\bullet}_1}
		\& A^{\bullet}_0
	\twocellA{\lambda}
	\twocellB{\alpha^{\bullet}}
\end{tikzcd}
=
\begin{tikzcd}[bend angle=50,ampersand replacement=\&]
	B_1,B_2 \ar[bend left]{r}[domA]{}{d_1,d_0}
			\rar[][codA,domB,swap]{d_0,d_0} 
			\dar[swap]{G_0} 
		\& A_1,A_2 \dar{F_0} \\
	B^{\bullet}_0 \rar[][codB,swap]{d^{\bullet}_1} 
		\& A^{\bullet}_0
	\twocellA{\alpha,\id}
	\twocellB{\lambda}
\end{tikzcd}
\\
\begin{tikzcd}[bend angle=50,ampersand replacement=\&]
	B_1,B_2 \rar[][domA]{d_0,d_1} 
			\dar[swap]{G_0} 
		\& A_1,A_2 \dar{F_0} \\
	B^{\bullet}_0 \rar[][codA,domB]{d^{\bullet}_0}	
			\rar[bend right][codB,swap]{d^{\bullet}_1}
		\& A^{\bullet}_0
	\twocellA{\lambda}
	\twocellB{\alpha^{\bullet}}
\end{tikzcd}
=
\begin{tikzcd}[bend angle=50,ampersand replacement=\&]
	B_1,B_2 \rar[bend left][domA]{d_0,d_1} 
			\rar[][codA,domB,swap]{d_0,d_0} 
			\dar[swap]{G_0} 
		\& A_1,A_2 \dar{F_0} \\
	B^{\bullet}_0 \rar[][codB,swap]{d^{\bullet}_1} 
		\& A^{\bullet}_0
	\twocellA{\id,\alpha}
	\twocellB{\lambda}
\end{tikzcd}
\\
\begin{tikzcd}[bend angle=50,ampersand replacement=\&]
	B_1,B_2 \rar[bend left][domA]{d_1,d_1} 
			\rar[][codA,domB,swap]{d_1,d_0} 
			\dar[swap]{G_0} 
		\& A_1,A_2 \dar{F_0} \\
	B^{\bullet}_0 \rar[][codB,swap]{d^{\bullet}_0} 
		\& A^{\bullet}_0
	\twocellA{\id,\alpha}
	\twocellB{\lambda}
\end{tikzcd}
=
\begin{tikzcd}[bend angle=50,ampersand replacement=\&]
	B_1,B_2 \rar[bend left][domA]{d_1,d_1} 
			\rar[][codA,domB,swap]{d_0,d_1} 
			\dar[swap]{G_0} 
		\& A_1,A_2 \dar{F_0} \\
	B^{\bullet}_0 \rar[][codB,swap]{d^{\bullet}_0} 
		\& A^{\bullet}_0
	\twocellA{\alpha,\id}
	\twocellB{\lambda}
\end{tikzcd}
\end{gather*}
and their reflections under the cyclic action.
\end{remark}

\begin{example}\label{Ex:PullbackProduct}
Let $\mathbb{M}\mathbf{Adj}$ be the double cyclic multicategory of categories, functors, and multivariable right adjunctions. Any multivariable right adjunction $F_0\colon \cat{A}_1\times \cat{A}_2\to \cat{A}_0$ extends to a functor $\widehat{\mathbb{F}}\colon\mathbb{J}\to\mathbb{M}\mathbf{Adj}$ as follows.
\begin{itemize}
	\item $B_i$ is sent to $\cat{A}_i^{\btwo}$, the arrow category of $\cat{A}_i$.
	\item The $d_1$ are sent to the domain functors $\dom\colon\cat{A}_i^{\btwo}\to\cat{A}_i$ and the $d_0$ are sent to the codomain functors $\cod\colon\cat{A}_i^{\btwo}\to\cat{A}_i$.
	\item The $\alpha$ are sent to the canonical natural transformations $\dom\Rightarrow\cod$.
	\item The $G_i$ are sent to functors $\hat{F}_i$. Given morphisms $f\colon A\to B\in\cat{A}_1$ and $g\colon X\to Y\in\cat{A}_2$, $\hat{F}_0(f,g)$ is defined as in the diagram
	\begin{equation}\label{E:PullbackProduct}
	\begin{tikzcd}[bend angle=15]
		F_0(B,Y) \arrow[bend left]{rrd}{F_0(1,g)}
			\arrow[bend right]{rdd}[swap]{F_0(f,1)}
			\arrow[dashed]{dr}{\hat{F}_0(f,g)}
		&[-4ex]& \\
		& F_0(A,Y)\!\!\underset{F_0(A,X)}{\prod}\!\!F_0(B,X)
			\rar{p_2}
			\dar[swap]{p_1}
		& F_0(B,X) \dar{F_0(f,1)} \\
		& F_0(A,Y) \rar[swap]{F_0(1,g)} & F_0(A,X)
	\end{tikzcd}
	\end{equation}
	It is a standard fact that the $\hat{F}_i$ form a two-variable adjunction between the arrow categories. 
	\item Looking at diagram \ref{E:PullbackProduct}, 
	\begin{align*}
		(\lambda^0_{1,0,0})_{f,g}&=p_1\colon \cod\hat{F}_0(f,g)\to F_0(\dom f,\cod g)\\
		(\lambda^0_{0,1,0})_{f,g}&=p_2\colon \cod\hat{F}_0(f,g)\to F_0(\cod f,\dom g)\\
		(\lambda^0_{0,0,1})_{f,g}&=\id\colon \dom\hat{F}_0(f,g)\to F_0(\cod f,\cod g).
	\end{align*}
	The three relations (1)-(3) then correspond precisely to the commutativity of the three regions in diagram \eqref{E:PullbackProduct}.
\end{itemize}
\end{example}

\begin{exercise}
Check that the mates of the morphism $p_1$ in diagram \ref{E:PullbackProduct} are $p_2$ and $\id$ in the two similar diagrams defining $\hat{F}_1$ and $\hat{F}_2$.
\end{exercise}

Let $\mathbb{I}$ be the sub-category of $\mathbb{J}$ consisting of just the 1-cells $F_i$. Let $\mathbf{CDMCat}$ denote the 2-category of cyclic double multicategories, functors, and horizontal transformations.

\begin{theorem}\label{Thm:MAdjArrowObjects}
	Fix a functor $\mathbb{F}\colon\mathbb{I}\to\mathbb{M}\mathbf{Adj}$. Then the functor $\hat{\mathbb{F}}\colon\mathbb{J}\to\mathbb{M}\mathbf{Adj}$ constructed in example \ref{Ex:PullbackProduct} is terminal in the category $\mathbf{CDMCat}_{\mathbb{F}}(\mathbb{J},\mathbb{M}\mathbf{Adj})$ of functors on $\mathbb{J}$ restricting to $\mathbb{F}$ on $\mathbb{I}$.
\end{theorem}
\begin{proof}
	Concretely, the theorem says that given the data of a functor $\mathbb{J}\to\mathbb{M}\mathbf{Adj}$, there is a unique 2-cell
	\[
	\begin{tikzcd}
		\cat{B}_1,\cat{B}_2 \rar[][domA]{H_1,H_2} 
				\dar[swap]{G_0} 
			& \cat{A}^{\mathbf{2}}_1,\cat{A}^{\mathbf{2}}_2 \dar{\hat{F}_0} \\
		\cat{B}^{\bullet}_0 \rar[][codA,swap]{H^{\bullet}_3}	
			& \cat{A}^{\bullet\mathbf{2}}_0
		\twocellA{\theta}
	\end{tikzcd}
	\]
	such that

	\begin{equation}\label{Eq:UProp1}
	\begin{tikzcd}
		\cat{B}_1,\cat{B}_2 \rar[][domA]{H_1,H_2} 
				\dar[swap]{G_0} 
			& \cat{A}^{\mathbf{2}}_1,\cat{A}^{\mathbf{2}}_2
				\rar[][domB]{\cod,\cod}
				\dar{\hat{F}_0}
			& \cat{A}_1,\cat{A}_2 \dar{F_0}\\
		\cat{B}^{\bullet}_0 \rar[][codA,swap]{H^{\bullet}_3}	
			& \cat{A}^{\bullet\mathbf{2}}_0 \rar[][codB,swap]{\dom^{\bullet}}
			& \cat{A}^{\bullet}_0
		\twocellA{\theta}
		\twocellB{\id}
	\end{tikzcd}
	=
	\begin{tikzcd}
		\cat{B}_1,\cat{B}_2
				\rar[][domA]{d_0,d_0} 
				\dar[swap]{G_0} 
			& \cat{A}_1,\cat{A}_2 \dar{F_0} \\
		\cat{B}^{\bullet}_0 \rar[][codA,swap]{d^{\bullet}_1} 
			& \cat{A}^{\bullet}_0
		\twocellA{\lambda}
	\end{tikzcd}
	\end{equation}
	%
	\begin{equation}\label{Eq:UProp2}
	\begin{tikzcd}
		\cat{B}_1,\cat{B}_2 \rar[][domA]{H_1,H_2} 
				\dar[swap]{G_0} 
			& \cat{A}^{\mathbf{2}}_1,\cat{A}^{\mathbf{2}}_2
				\rar[][domB]{\dom,\cod}
				\dar{\hat{F}_0}
			& \cat{A}_1,\cat{A}_2 \dar{F_0}\\
		\cat{B}^{\bullet}_0 \rar[][codA,swap]{H^{\bullet}_3}	
			& \cat{A}^{\bullet\mathbf{2}}_0 \rar[][codB,swap]{\cod^{\bullet}}
			& \cat{A}^{\bullet}_0
		\twocellA{\theta}
		\twocellB{p_1}
	\end{tikzcd}
	=
	\begin{tikzcd}
		\cat{B}_1,\cat{B}_2
				\rar[][domA]{d_1,d_0} 
				\dar[swap]{G_0} 
			& \cat{A}_1,\cat{A}_2 \dar{F_0} \\
		\cat{B}^{\bullet}_0 \rar[][codA,swap]{d^{\bullet}_0} 
			& \cat{A}^{\bullet}_0
		\twocellA{\lambda}
	\end{tikzcd}
	\end{equation}
	%
	\begin{equation}\label{Eq:UProp3}
	\begin{tikzcd}
		\cat{B}_1,\cat{B}_2 \rar[][domA]{H_1,H_2} 
				\dar[swap]{G_0} 
			& \cat{A}^{\mathbf{2}}_1,\cat{A}^{\mathbf{2}}_2
				\rar[][domB]{\cod,\dom}
				\dar{\hat{F}_0}
			& \cat{A}_1,\cat{A}_2 \dar{F_0}\\
		\cat{B}^{\bullet}_0 \rar[][codA,swap]{H^{\bullet}_3}	
			& \cat{A}^{\bullet\mathbf{2}}_0 \rar[][codB,swap]{\cod^{\bullet}}
			& \cat{A}^{\bullet}_0
		\twocellA{\theta}
		\twocellB{p_2}
	\end{tikzcd}
	=
	\begin{tikzcd}
		\cat{B}_1,\cat{B}_2
				\rar[][domA]{d_0,d_1} 
				\dar[swap]{G_0} 
			& \cat{A}_1,\cat{A}_2 \dar{F_0} \\
		\cat{B}^{\bullet}_0 \rar[][codA,swap]{d^{\bullet}_0} 
			& \cat{A}^{\bullet}_0
		\twocellA{\lambda}
	\end{tikzcd}
	\end{equation}

	Fix objects $B_1\in\cat{B}_1$, $B_2\in\cat{B}_2$. The $H_i$ are the functors sending $B_i$ to $H_i(B_i)=\alpha_{B_i}\colon d_1B_i\to d_0B_i$. The component of $\theta$ at $(B_1,B_2)$ is a square
	\[
	\begin{tikzcd}
	d_1G_0(B_1,B_2) \rar \dar
	& F_0(d_0B_1,d_0B_2) \dar \\
	d_0G_0(B_1,B_2) \rar
	& F_0(d_1B_1,d_0B_2) \!\!\! \underset{F_0(d_1B_1,d_1B_2)}{\prod} \!\!\! F_0(d_0B_1,d_1B_2)
	\end{tikzcd}
	\]
	The top arrow is uniquely determined by equation \eqref{Eq:UProp1}, while the components of the bottom arrow are uniquely determined by equations \eqref{Eq:UProp2} and \eqref{Eq:UProp3}.
\end{proof}

Now let $\dblcat{M}$ be any cyclic double multicategory. We will take Theorem~\ref{Thm:MAdjArrowObjects} as our definition of what it means for a cyclic double multicategory to have arrow objects. For future reference, we will spell this out more concretely.

Given an object $C$ of $\dblcat{M}$, an arrow object $C^2$ is an object together with a globular 2-cell $\kappa\colon\dom\Rightarrow\cod$ satisfying the same universal property as in Section~\ref{Sec:ArrowObjects} (this only involves the horizontal 2-category, so carries over unchanged).

Given a vertical 1-cell $F\colon(C_1,C_2)\to C_0^{\bullet}$, the lift to arrow objects $\hat{F}$ is a vertical 1-cell $\hat{F}$ together with 2-cells
\[
\begin{tikzcd}
	C_1^2,C_2^2 \rar[][domA]{\cod,\cod} \dar[swap]{\hat{F}}
		& C_1,C_2 \dar{F} \\
	C_0^{\bullet 2} \rar[][swap,codA]{\dom^{\bullet}}
		& C_0^{\bullet}
	\twocellA{\gamma_0}
\end{tikzcd}
\quad
\begin{tikzcd}
	C_1^2,C_2^2 \rar[][domA]{\dom,\cod} \dar[swap]{\hat{F}}
		& C_1,C_2 \dar{F} \\
	C_0^{\bullet 2} \rar[][swap,codA]{\cod^{\bullet}}
		& C_0^{\bullet}
	\twocellA{\gamma_1}
\end{tikzcd}
\quad
\begin{tikzcd}
	C_1^2,C_2^2 \rar[][domA]{\cod,\dom} \dar[swap]{\hat{F}}
		& C_1,C_2 \dar{F} \\
	C_0^{\bullet 2} \rar[][swap,codA]{\cod^{\bullet}}
		& C_0^{\bullet}
	\twocellA{\gamma_2}
\end{tikzcd}
\]
satisfying the equations
\begin{gather}
\begin{tikzcd}[bend angle=50,ampersand replacement=\&]
	C_1^2,C_2^2 \rar[][domA]{\dom,\cod} 
			\dar[swap]{G_0}
		\& C_1,C_2 \dar{F_0} \\
	C_0^{\bullet 2} \rar[][codA,domB]{\cod^{\bullet}}
			\rar[bend right][codB,swap]{\dom^{\bullet}}
		\& C_0^{\bullet}
	\twocellA{\gamma_1}
	\twocellB{\kappa^{\bullet}}
\end{tikzcd}
=
\begin{tikzcd}[bend angle=50,ampersand replacement=\&]
	C_1^2,C_2^2 \ar[bend left]{r}[domA]{}{\dom,\cod}
			\rar[][codA,domB,swap]{\cod,\cod} 
			\dar[swap]{G_0} 
		\& C_1,C_2 \dar{F_0} \\
	C_0^{\bullet 2} \rar[][codB,swap]{\dom^{\bullet}} 
		\& C_0^{\bullet}
	\twocellA{\kappa,\id}
	\twocellB{\gamma_0}
\end{tikzcd} \label{Eq:ArrowObjectMultiA}
\\
\begin{tikzcd}[bend angle=50,ampersand replacement=\&]
	C_1^2,C_2^2 \rar[][domA]{\cod,\dom} 
			\dar[swap]{G_0} 
		\& C_1,C_2 \dar{F_0} \\
	C_0^{\bullet 2} \rar[][codA,domB]{\cod^{\bullet}}	
			\rar[bend right][codB,swap]{\dom^{\bullet}}
		\& C_0^{\bullet}
	\twocellA{\gamma_2}
	\twocellB{\kappa^{\bullet}}
\end{tikzcd}
=
\begin{tikzcd}[bend angle=50,ampersand replacement=\&]
	C_1^2,C_2^2 \rar[bend left][domA]{\cod,\dom} 
			\rar[][codA,domB,swap]{\cod,\cod} 
			\dar[swap]{G_0} 
		\& C_1,C_2 \dar{F_0} \\
	C_0^{\bullet 2} \rar[][codB,swap]{\dom^{\bullet}} 
		\& C_0^{\bullet}
	\twocellA{\id,\kappa}
	\twocellB{\gamma_0}
\end{tikzcd} \label{Eq:ArrowObjectMultiB}
\\
\begin{tikzcd}[bend angle=50,ampersand replacement=\&]
	C_1^2,C_2^2 \rar[bend left][domA]{\dom,\dom} 
			\rar[][codA,domB,swap]{\dom,\cod} 
			\dar[swap]{G_0} 
		\& C_1,C_2 \dar{F_0} \\
	C_0^{\bullet 2} \rar[][codB,swap]{\cod^{\bullet}} 
		\& C_0^{\bullet}
	\twocellA{\id,\kappa}
	\twocellB{\gamma_1}
\end{tikzcd}
=
\begin{tikzcd}[bend angle=50,ampersand replacement=\&]
	C_1^2,C_2^2 \rar[bend left][domA]{\dom,\dom} 
			\rar[][codA,domB,swap]{\cod,\dom} 
			\dar[swap]{G_0} 
		\& C_1,C_2 \dar{F_0} \\
	C_0^{\bullet 2} \rar[][codB,swap]{\cod^{\bullet}} 
		\& C_0^{\bullet}
	\twocellA{\kappa,\id}
	\twocellB{\gamma_2}
\end{tikzcd} \label{Eq:ArrowObjectMultiC}
\end{gather}
and which is universal, meaning that given any objects $X_0,X_1,X_2$, horizontal 1-cells $d_{i,0},d_{i,1}\colon X_i\to C_i$, a vertical 1-cell $G\colon X_1,X_2\to X_0^{\bullet}$, globular 2-cells $\alpha_i\colon d_{i,1}\Rightarrow d_{i,0}$, and 2-cells
\[
\begin{tikzcd}
	X_1,X_2 \rar[][domA]{d_{1,0},d_{2,0}} \dar[swap]{G}
		& C_1,C_2 \dar{F} \\
	X_0^{\bullet} \rar[][swap,codA]{d_{0,1}^{\bullet}}
		& C_0^{\bullet}
	\twocellA{\lambda_0}
\end{tikzcd}
\quad
\begin{tikzcd}
	X_1,X_2 \rar[][domA]{d_{1,1},d_{2,0}} \dar[swap]{G}
		& C_1,C_2 \dar{F} \\
	X_0^{\bullet} \rar[][swap,codA]{d_{0,0}^{\bullet}}
		& C_0^{\bullet}
	\twocellA{\lambda_0}
\end{tikzcd}
\quad
\begin{tikzcd}
	X_1,X_2 \rar[][domA]{d_{1,0},d_{2,1}} \dar[swap]{G}
		& C_1,C_2 \dar{F} \\
	X_0^{\bullet} \rar[][swap,codA]{d_{0,0}^{\bullet}}
		& C_0^{\bullet}
	\twocellA{\lambda_2}
\end{tikzcd}
\]
satisfying the three equations analagous to \eqref{Eq:ArrowObjectMultiA}--\eqref{Eq:ArrowObjectMultiC}, there exists a unique 2-cell
\[
\begin{tikzcd}
	X_1,X_2 \rar[][domA]{\hat{\alpha}_1,\hat{\alpha}_2} \dar[swap]{G}
		& C_1^2,C_2^2 \dar{\hat{F}} \\
	X_0^{\bullet} \rar[][codA,swap]{\hat{\alpha}_0^{\bullet}}
		& C_0^{\bullet 2}
	\twocellA{\theta}
\end{tikzcd}
\]
(where $\hat{\alpha}_i$ is the 1-cell determined by $\alpha_i$ by the universal property of the arrow object $C_i$) such that
\[
\begin{tikzcd}
	X_1,X_2 \rar[][domA]{\hat{\alpha}_1,\hat{\alpha}_2} \dar[swap]{G}
		& C_1^2,C_2^2 \rar[][domB]{} \dar{\hat{F}}
		& C_1,C_2 \dar{F} \\
	X_0^{\bullet} \rar[][codA,swap]{\hat{\alpha}_0^{\bullet}}
		& C_0^{\bullet 2} \rar[][codB]{}
		& C_0^{\bullet}
	\twocellA{\theta}
	\twocellB{\gamma_i}
\end{tikzcd}
=
\begin{tikzcd}
	X_1,X_2 \rar[][domA]{} \dar[swap]{G}
		& C_1,C_2 \dar{F} \\
	X_0^{\bullet} \rar[][swap,codA]{}
		& C_0^{\bullet}
	\twocellA{\lambda_i}
\end{tikzcd}
\]
for each $i\in\{0,1,2\}$.

Similarly, we define the lift of a vertical 1-cell $F\colon(C_1,\dots,C_n)\to C_0^{\bullet}$ to arrow objects to be a vertical 1-cell $\hat{F}$ together with $(n+1)$ 2-cells $\gamma_i$ satisfying $(n+1)$ equations analagous to \eqref{Eq:ArrowObjectMultiA}--\eqref{Eq:ArrowObjectMultiC} and which is universal in the analagous way.

\begin{definition}
	Let $\dblcat{M}$ be a double multicategory. We say $\dblcat{M}$ \emph{has arrow objects} if for every object $C$ there is an arrow object $C^2$, and if for every vertical 1-cell $F\colon(C_1,\dots,C_n)\to C_0^{\bullet}$ there is a lift to arrow objects $\hat{F}$.
\end{definition}

We have given the universal property of arrow objects and lifts of vertical 1-cells in ordinary double multicategories, but it is clear from the cyclical symmetry of the construction that a cyclic action respects arrow objects. Specifically, $(C^2)^{\bullet}=(C^{\bullet})^2$ for any object $C$, and $\sigma(\hat{F})=\widehat{\sigma F}$ for any vertical 1-cell $F$.