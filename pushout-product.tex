% -*- root: thesis.tex -*-

\chapter{A UNIVERSAL PROPERTY FOR THE PUSHOUT PRODUCT}\label{Ch:PushoutProduct} 

In this chapter we will begin the work of incorporating adjunctions of several variables into the framework given so far. These are essential to making precise the definitions of monoidal model category and of a model category enriched in a monoidal model category.

Recall that a monoidal category $\cat{M}$ is called \emph{biclosed} if the tensor product has adjoints in each variable, i.e.~if there are functors
$\hom_l,\hom_r\colon\cat{M}^{\text{op}}\times\cat{M}\to\cat{M}$ and isomorphisms
\[
	\cat{M}(A\otimes B,C) \cong \cat{M}(B,\hom_l(A,C)) \cong \cat{M}(A,\hom_r(B,C))
\]
natural in all three variables. If $\cat{M}$ has a model structure, then one of the requirements for $\cat{M}$ to be a monoidal model category is that the three bifunctors $\otimes$, $\hom_l$, and $\hom_r$ form a \emph{Quillen adjunction of two variables}. There are three equivalent conditions for this:
\begin{enumerate}
	\item Given any cofibrations $i\colon A\to B$ and $j\colon J\to K$, the map $i\hat{\otimes}j$ defined by the pushout
	\[
	\begin{tikzcd}[bend angle=15]
		A\otimes J \rar{A\otimes j} \dar[swap]{i\otimes J}
			& A\otimes K \dar \ar[bend left]{ddr}{i\otimes K} &[-2ex] \\
		B\otimes J \rar \ar[bend right]{drr}[swap]{B\otimes j}
			& A\otimes K \!\underset{A\otimes J}{\coprod}\!B\otimes J 
				\drar[dashed,start anchor={[xshift=-3ex,yshift=2ex]}][description]{i\hat{\otimes}j} & \\[-2ex]
		& & B\otimes K
	\end{tikzcd}
	\]
	is a cofibration (which is trivial if either $i$ or $j$ is).
	\item Given any cofibration $i\colon A\to B$ and fibration $f\colon X\to Y$, the map
	\[
		\hat{\hom}_l(i,f)\colon \hom_l(B,X) \to \hom_l(A,X)\!\!\!\!\bigtimes_{\hom_l(A,Y)}\!\!\!\!\hom_l(B,Y)
	\]
	is a fibration (which is trivial if either $i$ or $f$ is).
	\item Given any cofibration $j\colon J\to K$ and fibration $f\colon X\to Y$, the map
	\[
		\hat{\hom}_r(j,f)\colon \hom_y(K,X) \to \hom_r(J,X)\!\!\!\!\bigtimes_{\hom_r(J,Y)}\!\!\!\!\hom_r(K,Y)
	\]
	is a fibration (which is trivial if either $i$ or $f$ is).
\end{enumerate}
Proving the equivalence of these three conditions is a routine but tedious exercise in adjunctions. Another exercise in adjunctions shows that $\hat{\otimes}$, $\hat{\hom}_l$, and $\hat{\hom}_r$ in fact make up an adjunction of two variables on the arrow category $\cat{M}^2$.

In this chapter, we will give a universal property satisfied by the functors $\hat{\otimes}$, $\hat{\hom}_l$, and $\hat{\hom}_r$ which will trivialize these kinds of routine adjunction arguments, as well as making precise the clear symmetry involved. Then in~\cref{Ch:DblMulti} we will make use of this universal property in order to show that the algebraic analogue of Quillen adjunctions of two variables (defined in~\cite{riehl:nwfs-monoidal}) can be recovered as multivariable morphisms of bimonads in a precise sense.

\section{Review of Cyclic Double Multicategories}

Just as we needed the mates correspondence to define adjunctions of algebraic weak factorization systems, we will need an extension of the mates correspondence to multivariable adjunctions in order to define multivariable adjunctions of awfs. Fortunately, both of these tasks have been done in~\cite{riehl:nwfs-monoidal} and~\cite{cgr:mates}. We will review the necessary material from those papers in this section.

Recall that a multicategory is a structure like a category, but where morphisms are allowed to have a \emph{list} of objects as their domain, which we write as
\[
	f\colon (X_1,\dots,X_n) \to Y
\]
sometimes dropping the parenthesis when they are not needed for readability. The composition takes a composable configuration $f\circ(g_1,\dots,g_n)$, where $f$ is as above and $g_i$ is a morphism with codomain $X_i$, and produces a morphism with codomain $Y$ and with domain the concatenation of all the domains of the $g_i$.

\begin{example}
	Any monoidal category has an underlying multicategory, in which the morphisms $(X_1,\dots,X_n) \to Y$ are simply defined to be morphisms $X_1\otimes\cdots\otimes X_n\to Y$. In fact, monoidal categories can be defined to be the multicategories having a certain representability property. 
\end{example}

A \emph{cyclic} multicategory involves both a duality on the objects, and a cyclic action on the morphisms taking a morphism $f\colon (X_1,\dots,X_n) \to X_0$ to a morphism
\[
	 \sigma f\colon(X_0^{\bullet},X_1,\dots,X_{n-1})\to X_n^{\bullet}
\]
In other words, the cyclic action cyclically permutes the objects in the domain and codomain, applying the duality whenever an object moves from domain to codomain or vice versa. There are some axioms governing the interplay between the cyclic action and the composition. We will refer to~\cite{cgr:mates} for complete details on the material of this section.

The canonical example of a cyclic multicategory is $\MAdjC_l$, whose objects are categories, and morphisms are multivariable mutual left adjoints. For example, consider an adjunction of two variables
\[
\begin{tikzcd}
	\cat{K}\times \cat{M} \rar{\otimes} & \cat{N}
\end{tikzcd}
\quad
\begin{tikzcd}
	\op{\cat{K}}\times \cat{N} \rar{\hom_l} & \cat{M}
\end{tikzcd}
\quad
\begin{tikzcd}
	\op{\cat{M}}\times \cat{N} \rar{\hom_r} & \cat{K}
\end{tikzcd}
\]
\[
	\cat{N}(k\otimes m,n) \cong \cat{M}(m,\hom_l(k,n)) \cong \cat{K}(k,\hom_r(m,n))
\]

Here $\otimes$ is adjoint on the left, while the homs are adjoint on the right, but we can arrange for all three to be left adjoints as follows:
\[
\begin{tikzcd}
	\cat{K}\times \cat{M} \rar{\otimes} & \cat{N}
\end{tikzcd}
\quad
\begin{tikzcd}
	\op{\cat{N}}\times \cat{K} \rar{\hom_l^{\text{op}}} & \op{\cat{M}}
\end{tikzcd}
\quad
\begin{tikzcd}
	\cat{M}\times \op{\cat{N}} \rar{\hom_r^{\text{op}}} & \op{\cat{K}}
\end{tikzcd}
\]
\[
	\cat{N}(k\otimes m,n) \cong \op{\cat{M}}(\hom_l(k,n),m) \cong \op{\cat{K}}(\hom_r(m,n),k).
\]
Written as three mutual left adjoints, and swapping the order of the inputs to $\hom_l$, we expose the cyclical symmetry. There is similarly a cyclic multicategory $\MAdjC_r$ whose morphisms are mutual right adjoints.

Cyclic double multicategories, first introduced in~\cite{cgr:mates}, are like the cyclic double categories defined in~\ref{Ch:Cyclic} but where the vertical category is enlarged to a vertical cyclic multicategory. For complete details, see~\cite{cgr:mates}, but the following example should make the idea clear:

\begin{example}
	Generalizing~\cref{Ex:CyclicLAdj}, there is a cyclic double multicategory $\MAdj_l$ whose objects are categories, whose vertical cyclic multicategory is $\MAdjC_l$, and whose 2-cells
	\[
	\begin{tikzcd}
		\cat{M}_1,\cat{M}_2 \dar[swap]{G_0} \rar[][domA]{u,v} & \cat{N}_1,\cat{N}_2 \dar{F_0} \\
		\cat{M}_0 \rar[][swap,codA]{w} & \cat{N}_0
		\twocellA{\phi}
	\end{tikzcd}
	\]
	are natural transformations $\phi\colon F_0(u,v)\Rightarrow wG_0$. The cyclic action permutes 2-cells $\phi$ to 2-cells
	\[
	\begin{tikzcd}
		\cat{M}_2,\cat{M}_0^{\text{op}} \rar[][domA]{v,\op{w}} \dar[swap]{G_1}
			& \cat{N}_2,\cat{N}_0^{\text{op}} \dar{F_1} \\
		\cat{M}_1^{\text{op}} \rar[][swap,codA]{\op{u}}
			& \cat{N}_1^{\text{op}}
		\twocellA{\sigma\phi}
	\end{tikzcd}
	\qquad
	\begin{tikzcd}
		\cat{M}_0^{\text{op}},\cat{M}_1 \rar[][domA]{\op{w},u} \dar[swap]{G_2}
			& \cat{N}_0^{\text{op}},\cat{N}_1 \dar{F_2} \\
		\cat{M}_2^{\text{op}} \rar[][swap,codA]{\op{v}}
			& \cat{N}_2^{\text{op}}
		\twocellA{\sigma^2\phi}
	\end{tikzcd}
	\]
	where $(G_0,G_1,G_2)$ and $(F_0,F_1,F_2)$ are each systems of mutual left adjoints, and where $\sigma\phi$ and $\sigma^2\phi$ are the two mates of $\phi$.

	The details of the mates correspondence are significantly more complicated in the multivariable case. The advantage of working with cyclic double multicategories is that these details are not important: the properties of the mates correspondence that are needed in practice are captured by the cyclic action.
\end{example}

\section{The Universal Property}

% First, we will need to find a generalization of the universal property for arrow objects to (cyclic) double multicategories. In the theory of multivariable Quillen adjunctions, the lift of a multivariable adjunction to arrow categories is provided by the pushout/pullback product, so we will identify a universal property satisfied by this construction.



Define a cyclic double multicategory $\mathbb{J}$ as follows. The objects are $A_i$, $B_i$, for $i\in\{0,1,2\}$, and their duals. The horizontal 1-cells are $d^i_0,d^i_1\colon B_i\to A_i$. The vertical 1-cells are $F_i\colon (A_{i-1},A_{i+1})\to A^{\bullet}_i$ and $G_i\colon (B_{i-1},B_{i+1})\to B^{\bullet}_i$, which form two orbits under the cyclic action.

There are two types of 2-cells. There are
\[\begin{tikzcd}
	B_i \rar[][domA]{d^i_1} \dar[swap]{\id}
		& A_i \dar{\id} \\
	B_i \rar[][codA,swap]{d^i_0}
		& A_i 
	\twocellA{\alpha_i}
\end{tikzcd}\]
for each $i$. We will often draw these 2-cells globularly.

There are also 2-cells
\[\begin{tikzcd}[column sep=large]
	B_{i+1},B_{i-1} \rar[][domA]{d^{i+1}_{k_{i+1}},d^{i-1}_{k_{i-1}}} 
			\dar[swap]{G_i} 
		& A_{i+1},A_{i-1} \dar{F_i} \\
	B^{\bullet}_i \rar[][codA,swap]{d^{i\bullet}_{k_i}} 
		& A^{\bullet}_i
	\twocellA{\lambda^i_{k_{i+1},k_{i-1},k_{i}}}
\end{tikzcd}\]
for all choices of $(k_0,k_1,k_2)\in\{0,1\}^3$ except $(0,0,0)$.

Notice that there is at most one element of every hom-set, so all compositions and cyclic actions are uniquely defined. From now on, we will omit indices whenever doing so is unambiguous.

\begin{remark}\label{Rmk:JGens}
The cyclic double multicategory $\mathbb{J}$ is generated under composition by the $\alpha_i$ and the $\lambda^i_{k_{i+1},k_{i-1},k_{i}}$ with exactly one of $k_0,k_1,k_2$ equal to 1. These nine $\lambda$ generators are further generated under the cyclic action by only three, though there are many choices of which three. These generators satisfy the relations

\[
\begin{tikzcd}[bend angle=50,ampersand replacement=\&]
	B_1,B_2 \rar[][domA]{d_1,d_0} 
			\dar[swap]{G_0}
		\& A_1,A_2 \dar{F_0} \\
	B^{\bullet}_0 \rar[][codA,domB]{d^{\bullet}_0}
			\rar[bend right][codB,swap]{d^{\bullet}_1}
		\& A^{\bullet}_0
	\twocellA{\lambda}
	\twocellB{\alpha^{\bullet}}
\end{tikzcd}
=
\begin{tikzcd}[bend angle=50,ampersand replacement=\&]
	B_1,B_2 \ar[bend left]{r}[domA]{}{d_1,d_0}
			\rar[][codA,domB,swap]{d_0,d_0} 
			\dar[swap]{G_0} 
		\& A_1,A_2 \dar{F_0} \\
	B^{\bullet}_0 \rar[][codB,swap]{d^{\bullet}_1} 
		\& A^{\bullet}_0
	\twocellA{\alpha,\id}
	\twocellB{\lambda}
\end{tikzcd}
\]
\[
\begin{tikzcd}[bend angle=50,ampersand replacement=\&]
	B_1,B_2 \rar[][domA]{d_0,d_1} 
			\dar[swap]{G_0} 
		\& A_1,A_2 \dar{F_0} \\
	B^{\bullet}_0 \rar[][codA,domB]{d^{\bullet}_0}	
			\rar[bend right][codB,swap]{d^{\bullet}_1}
		\& A^{\bullet}_0
	\twocellA{\lambda}
	\twocellB{\alpha^{\bullet}}
\end{tikzcd}
=
\begin{tikzcd}[bend angle=50,ampersand replacement=\&]
	B_1,B_2 \rar[bend left][domA]{d_0,d_1} 
			\rar[][codA,domB,swap]{d_0,d_0} 
			\dar[swap]{G_0} 
		\& A_1,A_2 \dar{F_0} \\
	B^{\bullet}_0 \rar[][codB,swap]{d^{\bullet}_1} 
		\& A^{\bullet}_0
	\twocellA{\id,\alpha}
	\twocellB{\lambda}
\end{tikzcd}
\]
\[
\begin{tikzcd}[bend angle=50,ampersand replacement=\&]
	B_1,B_2 \rar[bend left][domA]{d_1,d_1} 
			\rar[][codA,domB,swap]{d_1,d_0} 
			\dar[swap]{G_0} 
		\& A_1,A_2 \dar{F_0} \\
	B^{\bullet}_0 \rar[][codB,swap]{d^{\bullet}_0} 
		\& A^{\bullet}_0
	\twocellA{\id,\alpha}
	\twocellB{\lambda}
\end{tikzcd}
=
\begin{tikzcd}[bend angle=50,ampersand replacement=\&]
	B_1,B_2 \rar[bend left][domA]{d_1,d_1} 
			\rar[][codA,domB,swap]{d_0,d_1} 
			\dar[swap]{G_0} 
		\& A_1,A_2 \dar{F_0} \\
	B^{\bullet}_0 \rar[][codB,swap]{d^{\bullet}_0} 
		\& A^{\bullet}_0
	\twocellA{\alpha,\id}
	\twocellB{\lambda}
\end{tikzcd}
\]
and their reflections under the cyclic action.
\end{remark}

\begin{example}\label{Ex:PullbackProduct}
Let $\MAdj_r$ be the cyclic double multicategory of categories, functors, and multivariable right adjunctions. If $\cat{A}_0$, $\cat{A}_1$, and $\cat{A}_2$ have the necessary pushouts and pullbacks, then any multivariable right adjunction $F_0\colon \cat{A}_1\times \cat{A}_2\to \cat{A}_0$ extends to a functor $\widehat{\mathbb{F}}\colon\mathbb{J}\to\MAdj_r$ as follows.
\begin{itemize}
	\item $B_i$ is sent to $\cat{A}_i^{\btwo}$, the arrow category of $\cat{A}_i$.
	\item The $d_1$ are sent to the domain functors $\dom\colon\cat{A}_i^{\btwo}\to\cat{A}_i$ and the $d_0$ are sent to the codomain functors $\cod\colon\cat{A}_i^{\btwo}\to\cat{A}_i$.
	\item The $\alpha$ are sent to the canonical natural transformations $\dom\Rightarrow\cod$.
	\item The $G_i$ are sent to functors $\hat{F}_i$. Given morphisms $f\colon A\to B\in\cat{A}_1$ and $g\colon X\to Y\in\cat{A}_2$, $\hat{F}_0(f,g)$ is defined as in the diagram
	\begin{equation}\label{Eq:PullbackProduct}
	\begin{tikzcd}[bend angle=15]
		F_0(B,Y) \arrow[bend left]{rrd}{F_0(1,g)}
			\arrow[bend right]{rdd}[swap]{F_0(f,1)}
			\arrow[dashed]{dr}{\hat{F}_0(f,g)}
		&[-4ex]& \\
		& F_0(A,Y)\!\!\underset{F_0(A,X)}{\bigtimes}\!\!F_0(B,X)
			\rar{p_2}
			\dar[swap]{p_1}
		& F_0(B,X) \dar{F_0(f,1)} \\
		& F_0(A,Y) \rar[swap]{F_0(1,g)} & F_0(A,X)
	\end{tikzcd}
	\end{equation}
	It is a standard fact that the $\hat{F}_i$ form a two-variable adjunction between the arrow categories. 
	\item Looking at diagram~\eqref{Eq:PullbackProduct}, 
	\begin{align*}
		(\lambda^0_{1,0,0})_{f,g}&=p_1\colon \cod\hat{F}_0(f,g)\to F_0(\dom f,\cod g)\\
		(\lambda^0_{0,1,0})_{f,g}&=p_2\colon \cod\hat{F}_0(f,g)\to F_0(\cod f,\dom g)\\
		(\lambda^0_{0,0,1})_{f,g}&=\id\colon \dom\hat{F}_0(f,g)\to F_0(\cod f,\cod g).
	\end{align*}
	The three relations (1)-(3) then correspond precisely to the commutativity of the three regions in diagram~\eqref{Eq:PullbackProduct}.
\end{itemize}
\end{example}

% \begin{exercise}
% Check that the mates of the morphism $p_1$ in diagram \ref{E:PullbackProduct} are $p_2$ and $\id$ in the two similar diagrams defining $\hat{F}_1$ and $\hat{F}_2$.
% \end{exercise}

Let $\mathbb{I}$ be the sub-category of $\mathbb{J}$ consisting of just the 1-cells $F_i$. Let $\mathbf{CDMCat}$ denote the 2-category of cyclic double multicategories, functors, and horizontal transformations.

\begin{theorem}\label{Thm:MAdjArrowObjects}
	Fix a functor $\mathbb{F}\colon\mathbb{I}\to\mathbb{M}\mathbf{Adj}_r$, whose image is a two-variable mutual right adjunction $(F_0,F_1,F_2)$ between categories $\cat{A}_0,\cat{A}_1,\cat{A}_2$ which have the necessary pushouts and pullbacks. Then the functor $\hat{\mathbb{F}}\colon\mathbb{J}\to\MAdj_r$ constructed in \cref{Ex:PullbackProduct} is terminal in the category $\Cat{CDMCat}_{\mathbb{F}}(\mathbb{J},\MAdj_r)$ of functors on $\mathbb{J}$ restricting to $\mathbb{F}$ on $\mathbb{I}$.
\end{theorem}
\begin{proof}
	Concretely, the theorem says that given the data of a functor $\mathbb{J}\to\MAdj_r$, determining 2-variable adjunctions $F_i$ and $G_i$ and the rest of the structure spelled out in~\cref{Rmk:JGens}, there is a unique 2-cell
	\[
	\begin{tikzcd}
		\cat{B}_1,\cat{B}_2 \rar[][domA]{H_1,H_2} 
				\dar[swap]{G_0} 
			& \cat{A}^{\mathbf{2}}_1,\cat{A}^{\mathbf{2}}_2 \dar{\hat{F}_0} \\
		\cat{B}^{\bullet}_0 \rar[][codA,swap]{H^{\bullet}_3}	
			& \cat{A}^{\bullet\mathbf{2}}_0
		\twocellA{\theta}
	\end{tikzcd}
	\]
	where $\hat{F}_i$ is the pullback product defined in~\eqref{Eq:PullbackProduct}, such that

	\begin{equation}\label{Eq:UProp1}
	\begin{tikzcd}
		\cat{B}_1,\cat{B}_2 \rar[][domA]{H_1,H_2} 
				\dar[swap]{G_0} 
			& \cat{A}^{\mathbf{2}}_1,\cat{A}^{\mathbf{2}}_2
				\rar[][domB]{\cod,\cod}
				\dar{\hat{F}_0}
			& \cat{A}_1,\cat{A}_2 \dar{F_0}\\
		\cat{B}^{\bullet}_0 \rar[][codA,swap]{H^{\bullet}_3}	
			& \cat{A}^{\bullet\mathbf{2}}_0 \rar[][codB,swap]{\dom^{\bullet}}
			& \cat{A}^{\bullet}_0
		\twocellA{\theta}
		\twocellB{\id}
	\end{tikzcd}
	=
	\begin{tikzcd}
		\cat{B}_1,\cat{B}_2
				\rar[][domA]{d_0,d_0} 
				\dar[swap]{G_0} 
			& \cat{A}_1,\cat{A}_2 \dar{F_0} \\
		\cat{B}^{\bullet}_0 \rar[][codA,swap]{d^{\bullet}_1} 
			& \cat{A}^{\bullet}_0
		\twocellA{\lambda}
	\end{tikzcd}
	\end{equation}
	%
	\begin{equation}\label{Eq:UProp2}
	\begin{tikzcd}
		\cat{B}_1,\cat{B}_2 \rar[][domA]{H_1,H_2} 
				\dar[swap]{G_0} 
			& \cat{A}^{\mathbf{2}}_1,\cat{A}^{\mathbf{2}}_2
				\rar[][domB]{\dom,\cod}
				\dar{\hat{F}_0}
			& \cat{A}_1,\cat{A}_2 \dar{F_0}\\
		\cat{B}^{\bullet}_0 \rar[][codA,swap]{H^{\bullet}_3}	
			& \cat{A}^{\bullet\mathbf{2}}_0 \rar[][codB,swap]{\cod^{\bullet}}
			& \cat{A}^{\bullet}_0
		\twocellA{\theta}
		\twocellB{p_1}
	\end{tikzcd}
	=
	\begin{tikzcd}
		\cat{B}_1,\cat{B}_2
				\rar[][domA]{d_1,d_0} 
				\dar[swap]{G_0} 
			& \cat{A}_1,\cat{A}_2 \dar{F_0} \\
		\cat{B}^{\bullet}_0 \rar[][codA,swap]{d^{\bullet}_0} 
			& \cat{A}^{\bullet}_0
		\twocellA{\lambda}
	\end{tikzcd}
	\end{equation}
	%
	\begin{equation}\label{Eq:UProp3}
	\begin{tikzcd}
		\cat{B}_1,\cat{B}_2 \rar[][domA]{H_1,H_2} 
				\dar[swap]{G_0} 
			& \cat{A}^{\mathbf{2}}_1,\cat{A}^{\mathbf{2}}_2
				\rar[][domB]{\cod,\dom}
				\dar{\hat{F}_0}
			& \cat{A}_1,\cat{A}_2 \dar{F_0}\\
		\cat{B}^{\bullet}_0 \rar[][codA,swap]{H^{\bullet}_3}	
			& \cat{A}^{\bullet\mathbf{2}}_0 \rar[][codB,swap]{\cod^{\bullet}}
			& \cat{A}^{\bullet}_0
		\twocellA{\theta}
		\twocellB{p_2}
	\end{tikzcd}
	=
	\begin{tikzcd}
		\cat{B}_1,\cat{B}_2
				\rar[][domA]{d_0,d_1} 
				\dar[swap]{G_0} 
			& \cat{A}_1,\cat{A}_2 \dar{F_0} \\
		\cat{B}^{\bullet}_0 \rar[][codA,swap]{d^{\bullet}_0} 
			& \cat{A}^{\bullet}_0
		\twocellA{\lambda}
	\end{tikzcd}
	\end{equation}

	Fix objects $B_1\in\cat{B}_1$, $B_2\in\cat{B}_2$. The $H_i$ are the functors sending $B_i$ to $H_i(B_i)=\alpha_{B_i}\colon d_1B_i\to d_0B_i$. The component of $\theta$ at $(B_1,B_2)$ is a square
	\[
	\begin{tikzcd}
	d_1G_0(B_1,B_2) \rar \dar
	& F_0(d_0B_1,d_0B_2) \dar \\
	d_0G_0(B_1,B_2) \rar
	& F_0(d_1B_1,d_0B_2) \!\!\! \underset{F_0(d_1B_1,d_1B_2)}{\prod} \!\!\! F_0(d_0B_1,d_1B_2)
	\end{tikzcd}
	\]
	The top arrow is uniquely determined by equation~\eqref{Eq:UProp1}, while the components of the bottom arrow are uniquely determined by equations~\eqref{Eq:UProp2} and~\eqref{Eq:UProp3}.
\end{proof}

\section{Arrow Objects in Cyclic Double Multicategories}

Now let $\dblcat{M}$ be any cyclic double multicategory. We will take \cref{Thm:MAdjArrowObjects} as our definition of what it means for a general cyclic double multicategory to have arrow objects. For future reference, we will spell this out more concretely.

Given an object $C$ of $\dblcat{M}$, an arrow object $C^2$ is an object together with a globular 2-cell $\kappa\colon\dom\Rightarrow\cod$ satisfying the same universal property as in \cref{Sec:ArrowObjects} (this only involves the horizontal 2-category, so carries over unchanged).

Given a vertical 1-cell $F\colon(C_1,C_2)\to C_0^{\bullet}$, the lift to arrow objects $\hat{F}$ is a vertical 1-cell $\hat{F}$ together with 2-cells
\[
\begin{tikzcd}
	C_1^2,C_2^2 \rar[][domA]{\cod,\cod} \dar[swap]{\hat{F}}
		& C_1,C_2 \dar{F} \\
	C_0^{\bullet 2} \rar[][swap,codA]{\dom^{\bullet}}
		& C_0^{\bullet}
	\twocellA{\gamma_0}
\end{tikzcd}
\quad
\begin{tikzcd}
	C_1^2,C_2^2 \rar[][domA]{\dom,\cod} \dar[swap]{\hat{F}}
		& C_1,C_2 \dar{F} \\
	C_0^{\bullet 2} \rar[][swap,codA]{\cod^{\bullet}}
		& C_0^{\bullet}
	\twocellA{\gamma_1}
\end{tikzcd}
\quad
\begin{tikzcd}
	C_1^2,C_2^2 \rar[][domA]{\cod,\dom} \dar[swap]{\hat{F}}
		& C_1,C_2 \dar{F} \\
	C_0^{\bullet 2} \rar[][swap,codA]{\cod^{\bullet}}
		& C_0^{\bullet}
	\twocellA{\gamma_2}
\end{tikzcd}
\]
satisfying the equations
\begin{equation} \label{Eq:ArrowObjectMultiA}
\begin{tikzcd}[bend angle=50,ampersand replacement=\&]
	C_1^2,C_2^2 \rar[][domA]{\dom,\cod} 
			\dar[swap]{G_0}
		\& C_1,C_2 \dar{F_0} \\
	C_0^{\bullet 2} \rar[][codA,domB]{\cod^{\bullet}}
			\rar[bend right][codB,swap]{\dom^{\bullet}}
		\& C_0^{\bullet}
	\twocellA{\gamma_1}
	\twocellB{\kappa^{\bullet}}
\end{tikzcd}
=
\begin{tikzcd}[bend angle=50,ampersand replacement=\&]
	C_1^2,C_2^2 \ar[bend left]{r}[domA]{}{\dom,\cod}
			\rar[][codA,domB,swap]{\cod,\cod} 
			\dar[swap]{G_0} 
		\& C_1,C_2 \dar{F_0} \\
	C_0^{\bullet 2} \rar[][codB,swap]{\dom^{\bullet}} 
		\& C_0^{\bullet}
	\twocellA{\kappa,\id}
	\twocellB{\gamma_0}
\end{tikzcd}
\end{equation}
\begin{equation} \label{Eq:ArrowObjectMultiB}
\begin{tikzcd}[bend angle=50,ampersand replacement=\&]
	C_1^2,C_2^2 \rar[][domA]{\cod,\dom} 
			\dar[swap]{G_0} 
		\& C_1,C_2 \dar{F_0} \\
	C_0^{\bullet 2} \rar[][codA,domB]{\cod^{\bullet}}	
			\rar[bend right][codB,swap]{\dom^{\bullet}}
		\& C_0^{\bullet}
	\twocellA{\gamma_2}
	\twocellB{\kappa^{\bullet}}
\end{tikzcd}
=
\begin{tikzcd}[bend angle=50,ampersand replacement=\&]
	C_1^2,C_2^2 \rar[bend left][domA]{\cod,\dom} 
			\rar[][codA,domB,swap]{\cod,\cod} 
			\dar[swap]{G_0} 
		\& C_1,C_2 \dar{F_0} \\
	C_0^{\bullet 2} \rar[][codB,swap]{\dom^{\bullet}} 
		\& C_0^{\bullet}
	\twocellA{\id,\kappa}
	\twocellB{\gamma_0}
\end{tikzcd}
\end{equation}
\begin{equation} \label{Eq:ArrowObjectMultiC}
\begin{tikzcd}[bend angle=50,ampersand replacement=\&]
	C_1^2,C_2^2 \rar[bend left][domA]{\dom,\dom} 
			\rar[][codA,domB,swap]{\dom,\cod} 
			\dar[swap]{G_0} 
		\& C_1,C_2 \dar{F_0} \\
	C_0^{\bullet 2} \rar[][codB,swap]{\cod^{\bullet}} 
		\& C_0^{\bullet}
	\twocellA{\id,\kappa}
	\twocellB{\gamma_1}
\end{tikzcd}
=
\begin{tikzcd}[bend angle=50,ampersand replacement=\&]
	C_1^2,C_2^2 \rar[bend left][domA]{\dom,\dom} 
			\rar[][codA,domB,swap]{\cod,\dom} 
			\dar[swap]{G_0} 
		\& C_1,C_2 \dar{F_0} \\
	C_0^{\bullet 2} \rar[][codB,swap]{\cod^{\bullet}} 
		\& C_0^{\bullet}
	\twocellA{\kappa,\id}
	\twocellB{\gamma_2}
\end{tikzcd}
\end{equation}
and which is universal, meaning that given any objects $X_0,X_1,X_2$, horizontal 1-cells $d_{i,0},d_{i,1}\colon X_i\to C_i$, a vertical 1-cell $G\colon X_1,X_2\to X_0^{\bullet}$, globular 2-cells $\alpha_i\colon d_{i,1}\Rightarrow d_{i,0}$, and 2-cells
\[
\begin{tikzcd}
	X_1,X_2 \rar[][domA]{d_{1,0},d_{2,0}} \dar[swap]{G}
		& C_1,C_2 \dar{F} \\
	X_0^{\bullet} \rar[][swap,codA]{d_{0,1}^{\bullet}}
		& C_0^{\bullet}
	\twocellA{\lambda_0}
\end{tikzcd}
\quad
\begin{tikzcd}
	X_1,X_2 \rar[][domA]{d_{1,1},d_{2,0}} \dar[swap]{G}
		& C_1,C_2 \dar{F} \\
	X_0^{\bullet} \rar[][swap,codA]{d_{0,0}^{\bullet}}
		& C_0^{\bullet}
	\twocellA{\lambda_0}
\end{tikzcd}
\quad
\begin{tikzcd}
	X_1,X_2 \rar[][domA]{d_{1,0},d_{2,1}} \dar[swap]{G}
		& C_1,C_2 \dar{F} \\
	X_0^{\bullet} \rar[][swap,codA]{d_{0,0}^{\bullet}}
		& C_0^{\bullet}
	\twocellA{\lambda_2}
\end{tikzcd}
\]
satisfying the three equations analogous to~\eqref{Eq:ArrowObjectMultiA}--\eqref{Eq:ArrowObjectMultiC}, there exists a unique 2-cell
\[
\begin{tikzcd}
	X_1,X_2 \rar[][domA]{\hat{\alpha}_1,\hat{\alpha}_2} \dar[swap]{G}
		& C_1^2,C_2^2 \dar{\hat{F}} \\
	X_0^{\bullet} \rar[][codA,swap]{\hat{\alpha}_0^{\bullet}}
		& C_0^{\bullet 2}
	\twocellA{\theta}
\end{tikzcd}
\]
(where $\hat{\alpha}_i$ is the 1-cell determined by $\alpha_i$ by the universal property of the arrow object $C_i$) such that
\[
\begin{tikzcd}
	X_1,X_2 \rar[][domA]{\hat{\alpha}_1,\hat{\alpha}_2} \dar[swap]{G}
		& C_1^2,C_2^2 \rar[][domB]{} \dar{\hat{F}}
		& C_1,C_2 \dar{F} \\
	X_0^{\bullet} \rar[][codA,swap]{\hat{\alpha}_0^{\bullet}}
		& C_0^{\bullet 2} \rar[][codB]{}
		& C_0^{\bullet}
	\twocellA{\theta}
	\twocellB{\gamma_i}
\end{tikzcd}
=
\begin{tikzcd}
	X_1,X_2 \rar[][domA]{} \dar[swap]{G}
		& C_1,C_2 \dar{F} \\
	X_0^{\bullet} \rar[][swap,codA]{}
		& C_0^{\bullet}
	\twocellA{\lambda_i}
\end{tikzcd}
\]
for each $i\in\{0,1,2\}$.

Similarly, we define the lift of a vertical 1-cell $F\colon(C_1,\dots,C_n)\to C_0^{\bullet}$ to arrow objects to be a vertical 1-cell $\hat{F}$ together with $(n+1)$ 2-cells $\gamma_i$ satisfying $(n+1)$ equations analogous to~\eqref{Eq:ArrowObjectMultiA}--\eqref{Eq:ArrowObjectMultiC} and which is universal in the analogous way.

\begin{definition}
	Let $\dblcat{M}$ be a double multicategory. We say $\dblcat{M}$ \emph{has arrow objects} if for every object $C$ there is an arrow object $C^2$, and if for every vertical 1-cell $F\colon(C_1,\dots,C_n)\to C_0^{\bullet}$ there is a lift to arrow objects $\hat{F}$.
\end{definition}

We have given the universal property of arrow objects and lifts of vertical 1-cells in ordinary double multicategories, but it is clear from the cyclical symmetry of the construction that a cyclic action respects arrow objects. Specifically, $(C^2)^{\bullet}=(C^{\bullet})^2$ for any object $C$, and $\sigma(\hat{F})=\widehat{\sigma F}$ for any vertical 1-cell $F$, with $\sigma(\gamma_i)=\gamma_{i+1}$.

\begin{example}
	Let $\EAdj$ be the restriction of $\MAdj$ to finitely complete and cocomplete categories (note that the functors are not required to preserve these limits or colimits). Finitely complete and cocomplete categories are closed under the formation of opposite categories, so $\EAdj$ is again a cyclic double multicategory. Then~\cref{Thm:MAdjArrowObjects} shows that $\EAdj$ has arrow objects.
\end{example}