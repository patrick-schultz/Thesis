\documentclass[10pt,oneside,draft,article]{memoir}
\usepackage{mathtools,amssymb,amsthm}
\usepackage{datetime}
\usepackage[T1]{fontenc}
\usepackage[sc]{mathpazo}
\usepackage{MnSymbol}
\usepackage{tikz, tikz-cd}
\usetikzlibrary{decorations.markings}
\linespread{1.05}

%\chapterstyle{dash}
\pagestyle{headings}

\newcommand{\Tref}[1]{{Theorem~\ref{#1}}}
\DeclareMathOperator{\id}{id}
\DeclareMathOperator{\dom}{dom}
\DeclareMathOperator{\cod}{cod}
\DeclareMathOperator{\dvert}{Vert}


\theoremstyle{plain}
\newtheorem{theorem}{Theorem}[chapter]
\newtheorem*{theoremnn}{Theorem}
\newtheorem{proposition}[theorem]{Proposition}
\newtheorem{corollary}[theorem]{Corollary}
\newtheorem{lemma}[theorem]{Lemma}

\theoremstyle{definition}
\newtheorem{definition}[theorem]{Definition}
\newtheorem{exercise}{Exercise}[chapter]

\theoremstyle{remark}
\newtheorem{example}[theorem]{Example}
\newtheorem{remark}[theorem]{Remark}


\newcommand{\cat}[1]{\mathcal{#1}}
\newcommand{\btwo}{\mathbf{2}}
\newcommand{\FF}[1]{\mathbb{F}\mathrm{F}(\mathbb{#1})}

\newcommand{\twocell}[3][]{\arrow[draw=none,#1]{#2}[auto=false]{\Downarrow #3}}
\newcommand{\twocellb}[3][]{\arrow[draw=none,to path={(dom#2)--(cod#2)\tikztonodes}]{}[anchor=center,#1]{\Downarrow #3}}
\newcommand{\twocellA}[2][]{\arrow[draw=none,to path={(domA)--(codA)\tikztonodes}]{}[anchor=center,#1]{\Downarrow #2}}
\newcommand{\twocellB}[2][]{\arrow[draw=none,to path={(domB)--(codB)\tikztonodes}]{}[anchor=center,#1]{\Downarrow #2}}
\newcommand{\twocellC}[2][]{\arrow[draw=none,to path={(domC)--(codC)\tikztonodes}]{}[anchor=center,#1]{\Downarrow #2}}
\newcommand{\twocellD}[2][]{\arrow[draw=none,to path={(domD)--(codD)\tikztonodes}]{}[anchor=center,#1]{\Downarrow #2}}

\tikzset{tick/.style={postaction={decorate,decoration={markings,mark=at position 0.5 with {\draw[-] (0,.4ex) -- (0,-.4ex);}}}}}
%\tikzset{dom/.style={shape=coordinate,alias=dom#1},
%		dom1/.style={dom=1}, dom2/.style={dom=2}}
%\tikzset{cod/.style={shape=coordinate,alias=cod#1},
%		cod1/.style={cod=1}, cod2/.style={cod=2}}
\tikzset{dom/.style={append after command={coordinate[alias=dom#1]}},
		domA/.style={dom=A}, domB/.style={dom=B},
		domC/.style={dom=C}, domD/.style={dom=D}}
\tikzset{cod/.style={append after command={coordinate[alias=cod#1]}},
		codA/.style={cod=A}, codB/.style={cod=B},
		codC/.style={cod=C}, codD/.style={cod=D}}

\begin{document}
\tightlists
\firmlists

\chapter{A universal property for the pushout product}\label{Ch:PushoutProduct} 

Define a cyclic double multicategory $\mathbb{J}$ as follows. The objects are $A_i$, $B_i$, for $i\in\{0,1,2\}$, and their duals. The horizontal 1-cells are $d^i_0,d^i_1\colon B_i\to A_i$. The vertical 1-cells are $F_i\colon (A_{i-1},A_{i+1})\to A^{\bullet}_i$ and $G_i\colon (B_{i-1},B_{i+1})\to B^{\bullet}_i$, which form two orbits under the cyclic action.

There are two types of 2-cells. There are
\[\begin{tikzcd}
	B_i \rar[][domA]{d^i_1} \dar[swap]{\id}
		& A_i \dar{\id} \\
	B_i \rar[][codA,swap]{d^i_0}
		& A_i 
	\twocellA{\alpha_i}
\end{tikzcd}\]
for each $i$. We will often draw these 2-cells globularly.

There are also 2-cells
\[\begin{tikzcd}[column sep=large]
	B_{i+1},B_{i-1} \rar[][domA]{d^{i+1}_{k_{i+1}},d^{i-1}_{k_{i-1}}} 
			\dar[swap]{G_i} 
		& A_{i+1},A_{i-1} \dar{F_i} \\
	B^{\bullet}_i \rar[][codA,swap]{d^{i\bullet}_{k_i}} 
		& A^{\bullet}_i
	\twocellA{\lambda^i_{k_{i+1},k_{i-1},k_{i}}}
\end{tikzcd}\]
for all choices of $(k_0,k_1,k_2)\in\{0,1\}^3$ except $(0,0,0)$.

Notice that there is at most one element of every hom-set, so all compositions and cyclic actions are uniquely defined. From now on, we will omit indices whenever doing so is unambiguous.

\begin{remark}
The cyclic double multicategory $\mathbb{J}$ is generated under composition by the $\alpha_i$ and the $\lambda^i_{k_{i+1},k_{i-1},k_{i}}$ with exactly one of $k_0,k_1,k_2$ equal to 1. These nine $\lambda$ generators are further generated under the cyclic action by only three, though there are many choices of which three. These generators satisfy the relations

\begin{equation}
\begin{tikzcd}[bend angle=50]
	B_1,B_2 \rar[][domA]{d_1,d_0} 
			\dar[swap]{G_0}
		& A_1,A_2 \dar{F_0} \\
	B^{\bullet}_0 \rar[][codA,domB]{d^{\bullet}_0}
			\rar[bend right][codB,swap]{d^{\bullet}_1}
		& A^{\bullet}_0
	\twocellA{\lambda}
	\twocellB{\alpha^{\bullet}}
\end{tikzcd}
=
\begin{tikzcd}[bend angle=50]
	B_1,B_2 \ar[bend left]{r}[domA]{}{d_1,d_0}
			\rar[][codA,domB,swap]{d_0,d_0} 
			\dar[swap]{G_0} 
		& A_1,A_2 \dar{F_0} \\
	B^{\bullet}_0 \rar[][codB,swap]{d^{\bullet}_1} 
		& A^{\bullet}_0
	\twocellA{\alpha,\id}
	\twocellB{\lambda}
\end{tikzcd}
\end{equation}

\begin{equation}
\begin{tikzcd}[bend angle=50]
	B_1,B_2 \rar[][domA]{d_0,d_1} 
			\dar[swap]{G_0} 
		& A_1,A_2 \dar{F_0} \\
	B^{\bullet}_0 \rar[][codA,domB]{d^{\bullet}_0}	
			\rar[bend right][codB,swap]{d^{\bullet}_1}
		& A^{\bullet}_0
	\twocellA{\lambda}
	\twocellB{\alpha^{\bullet}}
\end{tikzcd}
=
\begin{tikzcd}[bend angle=50]
	B_1,B_2 \rar[bend left][domA]{d_0,d_1} 
			\rar[][codA,domB,swap]{d_0,d_0} 
			\dar[swap]{G_0} 
		& A_1,A_2 \dar{F_0} \\
	B^{\bullet}_0 \rar[][codB,swap]{d^{\bullet}_1} 
		& A^{\bullet}_0
	\twocellA{\id,\alpha}
	\twocellB{\lambda}
\end{tikzcd}
\end{equation}

\begin{equation}
\begin{tikzcd}[bend angle=50]
	B_1,B_2 \rar[bend left][domA]{d_1,d_1} 
			\rar[][codA,domB,swap]{d_1,d_0} 
			\dar[swap]{G_0} 
		& A_1,A_2 \dar{F_0} \\
	B^{\bullet}_0 \rar[][codB,swap]{d^{\bullet}_0} 
		& A^{\bullet}_0
	\twocellA{\id,\alpha}
	\twocellB{\lambda}
\end{tikzcd}
=
\begin{tikzcd}[bend angle=50]
	B_1,B_2 \rar[bend left][domA]{d_1,d_1} 
			\rar[][codA,domB,swap]{d_0,d_1} 
			\dar[swap]{G_0} 
		& A_1,A_2 \dar{F_0} \\
	B^{\bullet}_0 \rar[][codB,swap]{d^{\bullet}_0} 
		& A^{\bullet}_0
	\twocellA{\alpha,\id}
	\twocellB{\lambda}
\end{tikzcd}
\end{equation}
and their reflections under the cyclic action.
\end{remark}

\begin{example}\label{Ex:PullbackProduct}
Let $\mathbb{M}\mathbf{Adj}$ be the double cyclic multicategory of categories, functors, and multivariable right adjunctions. Any multivariable right adjunction $F_0\colon \cat{A}_1\times \cat{A}_2\to \cat{A}_0$ extends to a functor $\widehat{\mathbb{F}}\colon\mathbb{J}\to\mathbb{M}\mathbf{Adj}$ as follows.
\begin{itemize}
	\item $B_i$ is sent to $\cat{A}_i^{\btwo}$, the arrow category of $\cat{A}_i$.
	\item The $d_1$ are sent to the domain functors $\dom\colon\cat{A}_i^{\btwo}\to\cat{A}_i$ and the $d_0$ are sent to the codomain functors $\cod\colon\cat{A}_i^{\btwo}\to\cat{A}_i$.
	\item The $\alpha$ are sent to the canonical natural transformations $\dom\Rightarrow\cod$.
	\item The $G_i$ are sent to functors $\hat{F}_i$. Given morphisms $f\colon A\to B\in\cat{A}_1$ and $g\colon X\to Y\in\cat{A}_2$, $\hat{F}_0(f,g)$ is defined as in the diagram
	\begin{equation}\label{E:PullbackProduct}
	\begin{tikzcd}[bend angle=15]
		F_0(B,Y) \arrow[bend left]{rrd}{F_0(1,g)}
			\arrow[bend right]{rdd}[swap]{F_0(f,1)}
			\arrow[dashed]{dr}{\hat{F}_0(f,g)}
		&[-4ex]& \\
		& F_0(A,Y)\!\!\underset{F_0(A,X)}{\prod}\!\!F_0(B,X)
			\rar{p_2}
			\dar[swap]{p_1}
		& F_0(B,X) \dar{F_0(f,1)} \\
		& F_0(A,Y) \rar[swap]{F_0(1,g)} & F_0(A,X)
	\end{tikzcd}
	\end{equation}
	It is a standard fact that the $\hat{F}_i$ form a two-variable adjunction between the arrow categories. 
	\item Looking at diagram \ref{E:PullbackProduct}, 
	\begin{align*}
		(\lambda^0_{1,0,0})_{f,g}&=p_1\colon \cod\hat{F}_0(f,g)\to F_0(\dom f,\cod g)\\
		(\lambda^0_{0,1,0})_{f,g}&=p_2\colon \cod\hat{F}_0(f,g)\to F_0(\cod f,\dom g)\\
		(\lambda^0_{0,0,1})_{f,g}&=\id\colon \dom\hat{F}_0(f,g)\to F_0(\cod f,\cod g).
	\end{align*}
	The three relations (1)-(3) then correspond precisely to the commutativity of the three regions in diagram \eqref{E:PullbackProduct}.
\end{itemize}
\end{example}

\begin{exercise}
Check that the mates of the morphism $p_1$ in diagram \ref{E:PullbackProduct} are $p_2$ and $\id$ in the two similar diagrams defining $\hat{F}_1$ and $\hat{F}_2$.
\end{exercise}

Let $\mathbb{I}$ be the sub-category of $\mathbb{J}$ consisting of just the 1-cells $F_i$. Let $\mathbf{CDMCat}$ denote the 2-category of cyclic double multicategories, functors, and horizontal transformations.

\begin{theorem}
Fix a functor $\mathbb{F}\colon\mathbb{I}\to\mathbb{M}\mathbf{Adj}$. Then the functor $\hat{\mathbb{F}}\colon\mathbb{J}\to\mathbb{M}\mathbf{Adj}$ constructed in example \ref{Ex:PullbackProduct} is terminal in the category $\mathbf{CDMCat}_{\mathbb{F}}(\mathbb{J},\mathbb{M}\mathbf{Adj})$ of functors on $\mathbb{J}$ restricting to $\mathbb{F}$ on $\mathbb{I}$.
\end{theorem}
\begin{proof}
Concretely, the theorem says that given the data of a functor $\mathbb{J}\to\mathbb{M}\mathbf{Adj}$, there is a unique 2-cell
\[
\begin{tikzcd}
	\cat{B}_1,\cat{B}_2 \rar[][domA]{H_1,H_2} 
			\dar[swap]{G_0} 
		& \cat{A}^{\mathbf{2}}_1,\cat{A}^{\mathbf{2}}_2 \dar{\hat{F}_0} \\
	\cat{B}^{\bullet}_0 \rar[][codA,swap]{H^{\bullet}_3}	
		& \cat{A}^{\bullet\mathbf{2}}_0
	\twocellA{\theta}
\end{tikzcd}
\]
such that

\begin{equation}\label{Eq:UProp1}
\begin{tikzcd}
	\cat{B}_1,\cat{B}_2 \rar[][domA]{H_1,H_2} 
			\dar[swap]{G_0} 
		& \cat{A}^{\mathbf{2}}_1,\cat{A}^{\mathbf{2}}_2
			\rar[][domB]{\cod,\cod}
			\dar{\hat{F}_0}
		& \cat{A}_1,\cat{A}_2 \dar{F_0}\\
	\cat{B}^{\bullet}_0 \rar[][codA,swap]{H^{\bullet}_3}	
		& \cat{A}^{\bullet\mathbf{2}}_0 \rar[][codB,swap]{\dom^{\bullet}}
		& \cat{A}^{\bullet}_0
	\twocellA{\theta}
	\twocellB{\id}
\end{tikzcd}
=
\begin{tikzcd}
	\cat{B}_1,\cat{B}_2
			\rar[][domA]{d_0,d_0} 
			\dar[swap]{G_0} 
		& \cat{A}_1,\cat{A}_2 \dar{F_0} \\
	\cat{B}^{\bullet}_0 \rar[][codA,swap]{d^{\bullet}_1} 
		& \cat{A}^{\bullet}_0
	\twocellA{\lambda}
\end{tikzcd}
\end{equation}
%
\begin{equation}\label{Eq:UProp2}
\begin{tikzcd}
	\cat{B}_1,\cat{B}_2 \rar[][domA]{H_1,H_2} 
			\dar[swap]{G_0} 
		& \cat{A}^{\mathbf{2}}_1,\cat{A}^{\mathbf{2}}_2
			\rar[][domB]{\dom,\cod}
			\dar{\hat{F}_0}
		& \cat{A}_1,\cat{A}_2 \dar{F_0}\\
	\cat{B}^{\bullet}_0 \rar[][codA,swap]{H^{\bullet}_3}	
		& \cat{A}^{\bullet\mathbf{2}}_0 \rar[][codB,swap]{\cod^{\bullet}}
		& \cat{A}^{\bullet}_0
	\twocellA{\theta}
	\twocellB{p_1}
\end{tikzcd}
=
\begin{tikzcd}
	\cat{B}_1,\cat{B}_2
			\rar[][domA]{d_1,d_0} 
			\dar[swap]{G_0} 
		& \cat{A}_1,\cat{A}_2 \dar{F_0} \\
	\cat{B}^{\bullet}_0 \rar[][codA,swap]{d^{\bullet}_0} 
		& \cat{A}^{\bullet}_0
	\twocellA{\lambda}
\end{tikzcd}
\end{equation}
%
\begin{equation}\label{Eq:UProp3}
\begin{tikzcd}
	\cat{B}_1,\cat{B}_2 \rar[][domA]{H_1,H_2} 
			\dar[swap]{G_0} 
		& \cat{A}^{\mathbf{2}}_1,\cat{A}^{\mathbf{2}}_2
			\rar[][domB]{\cod,\dom}
			\dar{\hat{F}_0}
		& \cat{A}_1,\cat{A}_2 \dar{F_0}\\
	\cat{B}^{\bullet}_0 \rar[][codA,swap]{H^{\bullet}_3}	
		& \cat{A}^{\bullet\mathbf{2}}_0 \rar[][codB,swap]{\cod^{\bullet}}
		& \cat{A}^{\bullet}_0
	\twocellA{\theta}
	\twocellB{p_2}
\end{tikzcd}
=
\begin{tikzcd}
	\cat{B}_1,\cat{B}_2
			\rar[][domA]{d_0,d_1} 
			\dar[swap]{G_0} 
		& \cat{A}_1,\cat{A}_2 \dar{F_0} \\
	\cat{B}^{\bullet}_0 \rar[][codA,swap]{d^{\bullet}_0} 
		& \cat{A}^{\bullet}_0
	\twocellA{\lambda}
\end{tikzcd}
\end{equation}

Fix objects $B_1\in\cat{B}_1$, $B_2\in\cat{B}_2$. The $H_i$ are the functors sending $B_i$ to $H_i(B_i)=\alpha_{B_i}\colon d_1B_i\to d_0B_i$. The component of $\theta$ at $(B_1,B_2)$ is a square
\[
\begin{tikzcd}
d_1G_0(B_1,B_2) \rar \dar
& F_0(d_0B_1,d_0B_2) \dar \\
d_0G_0(B_1,B_2) \rar
& F_0(d_1B_1,d_0B_2) \!\!\! \underset{F_0(d_1B_1,d_1B_2)}{\prod} \!\!\! F_0(d_0B_1,d_1B_2)
\end{tikzcd}
\]
The top arrow is uniquely determined by equation \eqref{Eq:UProp1}, while the components of the bottom arrow are uniquely determined by equations \eqref{Eq:UProp2} and \eqref{Eq:UProp3}.
\end{proof}

\chapter{Weak factorization systems}

In this section, we will briefly review the notions of functorial factorization, weak factorization system, and algebraic weak factorization system. 

\subsection{Arrow Categories}

Let $\cat{C}$ be a category. Its arrow category $\cat{C}^2$ is the category whose objects are arrows in $\cat{C}$ and whose morphisms are commutative squares. The arrow category comes with two functors $\dom,\cod\colon \cat{C}^2\to\cat{C}$, along with a natural transformation $\kappa\colon\dom\Rightarrow\cod$. The component of $\kappa$ at an object $f$ of $\cat{C}^2$ is simply $f\colon\dom f\to\cod f$. Moreover, $\cat{C}^2$ satisfies a universal property: there is an equivalence of categories
\begin{equation}\label{Eq:ArrowObject}
	\mathrm{Fun}(2,\mathrm{Fun}(\cat{X},\cat{C}))\simeq\mathrm{Fun}(\cat{X},\cat{C}^2)
\end{equation}
given by composition with $\kappa$. Here, 2 is the ordinal, i.e. the category with two objects and a single non-identity arrow. In other words, $\cat{C}^2$ is the cotensor of $\cat{C}$ with the category 2 in the 2-category $\mathcal{C}\mathrm{at}$.

We will make this universal property more explicit in the next lemma:

\begin{lemma}\label{Lem:ArrowObject}
	Let $\cat{C}$ be a category.
	\begin{itemize}
		\item[i)] For any category $\cat{X}$, pair of functors $F,G\colon\cat{X}\to\cat{C}$, and natural transformation $\alpha\colon F\Rightarrow G$, there is a unique functor $\hat{\alpha}\colon\cat{X}\to\cat{C}^2$ such that $\dom\hat{\alpha}=F$, $\cod\hat{\alpha}=G$, and
		\begin{equation}
		\begin{tikzcd}[bend angle=30]
			\cat{X} \rar{\hat{\alpha}} 
			& \cat{C}^2 \rar[bend left][domA]{\dom}
				\rar[bend right][codA,swap]{\cod}
			& \cat{C}
			\twocellA{\kappa}
		\end{tikzcd}
		=
		\begin{tikzcd}[bend angle=30]
			\cat{X} \rar[bend left][domA]{F}
				\rar[bend right][codA,swap]{G}
			& \cat{C}.
			\twocellA{\alpha}
		\end{tikzcd}
		\end{equation}
		\item[ii)] For any functors $F,F',G,G'\colon\cat{X}\to\cat{C}$ and a commutative square of natural transformations
		\[
		\begin{tikzcd}[arrows=Rightarrow]
			F \rar{\gamma} \dar[swap]{\alpha}
			& F' \dar{\beta} \\
			G \rar[swap]{\phi}
			& G',
		\end{tikzcd}
		\]
		there is a unique natural transformation $\eta\colon\hat{\alpha}\to\hat{\beta}$ such that $\dom\eta=\gamma$ and $\cod\eta=\phi$, hence
		\[
		\begin{tikzcd}[bend angle=50]
			\cat{X}   \rar[bend left][domA]{F}
				\rar[][codA,domB,description,inner ysep=0]{G}
				\rar[bend right][codB,swap]{G'}
			& \cat{C}
			\twocellA[pos=.45]{\alpha}
			\twocellB[pos=.55]{\phi}
		\end{tikzcd}
		=
		\begin{tikzcd}[bend angle=30]
			\cat{X} \rar[bend left][domA]{\hat{\alpha}}
					\rar[bend right][codA,swap]{\hat{\beta}}
			& \cat{C}^2 \rar[bend left][domB]{\dom}
				\rar[bend right][codB,swap]{\cod}
			& \cat{C}
			\twocellA{\eta}
			\twocellB{\kappa}
		\end{tikzcd}
		=
		\begin{tikzcd}[bend angle=50]
			\cat{X}   \rar[bend left][domA]{F}
				\rar[][codA,domB,description,inner ysep=0]{F'}
				\rar[bend right][codB,swap]{G'}
			& \cat{C}.
			\twocellA[pos=.45]{\gamma}
			\twocellB[pos=.55]{\beta}
		\end{tikzcd}
		\]
	\end{itemize}
\end{lemma}

\begin{definition}
	Let $\cat{D}$ be any 2-category. For any object $A$ in $\cat{D}$, the \emph{arrow object} of $A$, if it exists, is an object $A^2$ satisfying the universal property \eqref{Eq:ArrowObject}. If every object has an arrow object, i.e. if $\cat{D}$ has cotensors by 2, we will say $\cat{D}$ \emph{has arrow objects}.
\end{definition}

\subsection{Functorial Factorizations}

\begin{definition}\label{Def:CatFF}
	A functorial factorization on a category $\cat{C}$ consists of a functor $E$ and two natural transformations $\eta$ and $\epsilon$ which factor $\kappa$, as in
	\[
	\begin{tikzcd}[bend angle=30]
		C^2 \rar[bend left][domA]{\dom}
			\rar[bend right][codA,swap]{\cod}
		& C
		\twocellA{\kappa}
	\end{tikzcd}
	=
	\begin{tikzcd}[bend angle=50]
		C^2 \rar[bend left][domA]{\dom}
			\rar[][codA,domB,description]{E}
			\rar[bend right][codB,swap]{\cod}
		& C.
		\twocellA[pos=.45]{\eta}
		\twocellB[pos=.55]{\epsilon}
	\end{tikzcd}
	\]
\end{definition}

This determines for any arrow $f$ in $\cat{C}$ a factorization $f=\epsilon_f\circ\eta_f$. The factorization is natural, meaning that for any morphism $(u,v)\colon f\Rightarrow g$ in $\cat{C}^2$ (i.e. commutative square in $\cat{C}$), the two squares in
\[
\begin{tikzcd}
	\cdot \rar{u} \dar[swap]{\eta_f} & \cdot \dar{\eta_g} \\
	\cdot \rar{E(u,v)} \dar[swap]{\epsilon_f} & \cdot \dar{\epsilon_g} \\
	\cdot \rar[swap]{v} & \cdot
\end{tikzcd}
\]
commute.

A functorial factorization also determines two functors $L,R\colon C^2\to C^2$ such that $\dom L=\dom$, $\cod R=\cod$, $\cod L = \dom R = E$, $\kappa L = \eta$, and $\kappa R=\epsilon$, by the universal property of $C^2$. The components of the factorization of $f$ can then also be referred to as $Lf$ and $Rf$, now thought of as objects in $\cat{C}^2$. There are also two canonical natural transformations, $\vec{\eta}\colon \id\Rightarrow R$ and $\vec{\epsilon}\colon L\Rightarrow\id$, determined by the commuting squares
\[
\begin{tikzcd}[arrows=Rightarrow]
	\dom \rar{\eta} \dar[swap]{\kappa} & E \dar{\epsilon} \\
	\cod \rar[swap]{\id} & \cod
\end{tikzcd}
\qquad\text{and}\qquad
\begin{tikzcd}[arrows=Rightarrow]
	\dom \rar{\id} \dar[swap]{\eta} & \dom \dar{\kappa} \\
	E \rar[swap]{\epsilon} & \cod
\end{tikzcd}
\]
respectively. These make $L$ and $R$ into (co)pointed endofunctors of $\cat{C}^2$.

An algebra for the pointed endofunctor $R$ is an object $f$ in $\cat{C}^2$ equipped with a morphism $\vec{t}\colon Rf\Rightarrow f$, such that $\vec{t}\circ\vec{\eta}_f=\id_f$. Similarly, a coalgebra for the copointed endofunctor $L$ is an $f$ equipped with a morphism $\vec{s}\colon f\Rightarrow Lf$, such that $\vec{\epsilon}_f\circ\vec{s}=\id_f$.

\begin{lemma}
	Let $f\colon X\to Y$ be a morphism in $\cat{C}$. An $R$-algebra structure on $f\in\cat{C}^2$ is precisely a choice of lift $t$ in the square
	\begin{equation}\label{Eq:RAlg}
	\begin{tikzcd}
		X \rar[equals] \dar[swap]{Lf} & X \dar{f} \\
		Ef \rar[swap]{Rf} \urar[dashed]{t} & Y.
	\end{tikzcd}
	\end{equation}
	Dually, an $L$-coalgebra structure on $f$ is precisely a choice of lift $s$ in the square
	\begin{equation}\label{Eq:LAlg}
	\begin{tikzcd}
		X \rar{Lf} \dar[swap]{f} & Ef \dar{Rf} \\
		Y \rar[equals] \urar[dashed]{s} & Y.
	\end{tikzcd}
	\end{equation}
\end{lemma}

\subsection{Algebraic Weak Factorization Systems}

To simplify the discussion of weak factorization systems, we will start by introducing a notation. For any two morphisms $l$ and $r$ in $\cat{C}$, write $l\boxslash r$ to mean that for every commutative square
\begin{equation}\label{Eq:LiftingSquare}
\begin{tikzcd}
	\cdot \rar{u} \dar[swap]{l} & \cdot \dar{r} \\
	\cdot \rar[swap]{v} \urar[dashed]{w} & \cdot
\end{tikzcd}
\end{equation}
there exists a lift $w$. In this case, we will say that $l$ has the \emph{left lifting property} with respect to $r$, and that $r$ has the \emph{right lifting property} with respect to $l$. Similarly, for two classes of morphisms $\mathcal{L}$ and $\mathcal{R}$, we will say $\mathcal{L}\boxslash\mathcal{R}$ if $l\boxslash r$ for every $l\in\mathcal{L}$ and $r\in\mathcal{R}$. Finally, we will write $\mathcal{L}^{\boxslash}$ for the class of morphisms having the right lifting property with respect to every morphism of $\mathcal{L}$, and ${}^{\boxslash}\!\mathcal{R}$ for the class of morphisms having the left lifting property with respect to every morphism of $\mathcal{R}$.

\begin{definition}
	A \emph{functorial weak factorization system} on a category $\cat{C}$ consists of a functorial factorization on $\cat{C}$ and two classes $\mathcal{L}$ and $\mathcal{R}$ of morphisms in $\cat{C}$, such that
	\begin{itemize}
		\item for every morphism $f$ in $\cat{C}$, $Lf\in\mathcal{L}$ and $Rf\in\mathcal{R}$,
		\item $\mathcal{L}^{\boxslash}=\mathcal{R}$ and ${}^{\boxslash}\!\mathcal{R}=\mathcal{L}$.
	\end{itemize}
\end{definition}

It a simple and standard proof that the lifting property condition can be replaced by two simpler conditions:

\begin{lemma}
	A functorial weak factorization system can equivalently be defined to be a functorial factorization on $\cat{C}$ and two classes $\mathcal{L}$ and $\mathcal{R}$ of morphisms in $\cat{C}$, such that
	\begin{itemize}
		\item for every morphism $f$ in $\cat{C}$, $Lf\in\mathcal{L}$ and $Rf\in\mathcal{R}$,
		\item $\mathcal{L}\boxslash\mathcal{R}$,
		\item $\mathcal{L}$ and $\mathcal{R}$ are both closed under retracts.
	\end{itemize}
\end{lemma}

In fact, the functorial factorization by itself already determines the two classes of morphisms, with $\mathcal{L}$ the class of morphisms admitting an $L$-coalgebra structure, and $\mathcal{R}$ the class of morphisms admitting an $R$-algebra structure. The lifting properties also follow from directly from the functorial factorization, as the next lemma shows.

\begin{lemma}
	For any $L$-coalgebra $(l,s)$ and any $R$-algebra $(r,t)$, there is a canonical choice of lift in the square \eqref{Eq:LiftingSquare}.
\end{lemma}
\begin{proof}
	The construction is shown in the diagram
	\begin{equation}\label{Eq:CanonicalLift}
	\begin{tikzcd}
		\cdot 	\rar{u} 
				\dar[swap]{Ll} 
			& \cdot \dar[xshift=0.5ex]{Lr} \\
		\cdot 	\rar[dashed]{E(u,v)} 
				\dar[xshift=-0.5ex][swap]{Rl} 
			& \cdot \uar[xshift=-0.5ex,dashed]{t} 
				\dar{Rr} \\
		\cdot 	\uar[xshift=0.5ex,dashed][swap]{s}
				\rar[swap]{v}
			& \cdot
	\end{tikzcd}
	\end{equation}
	Commutativity of \eqref{Eq:LiftingSquare} follows immediately from \eqref{Eq:RAlg} and \eqref{Eq:LAlg}.
\end{proof}

This, together with the classical fact that the class of objects admitting a (co)algebra structure for a (co)pointed endofunctor is closed under retracts, gives a third equivalent definition of a functorial weak factorization system.

\begin{lemma}
	A functorial weak factorization system can equivalently be defined to be a functorial factorization on $\cat{C}$ such that
	\begin{itemize}
		\item for every morphism $f$ in $\cat{C}$, $Lf$ admits an $L$-coalgebra structure, and $Rf$ admits an $R$-algebra structure.
	\end{itemize}
\end{lemma}

An $R$-algebra structure on $Rf$ consists of a morphism $\vec{\mu}_f\colon R^2f\to Rf$ in $\cat{C}^2$ such that $\vec{\mu}_f\circ\vec{\eta}_f=\id_f$, while an $L$-coalgebra structure on $Lf$ consists of a morphism $\vec{\delta}_f\colon Lf\to L^2f$ such that $\vec{\epsilon}_f\circ\vec{\delta}_f=\id_f$. We might hope that it is possible to choose these structures for all $f$ in a natural way, such that they form the components of natural transformations $\vec{\mu}\colon R^2\Rightarrow R$ and $\vec{\delta}\colon L\Rightarrow L^2$.

If we want these choices of lifts to be fully coherent, we should also ask that for any $R$-algebra $(f,t)$, the lift constructed as in \eqref{Eq:CanonicalLift} for the square \eqref{Eq:RAlg} is equal to $t$, and similarly for $L$-coalgebras and \eqref{Eq:LAlg}. Lastly, we should ask that the components $\vec{\mu}_f$ and $\vec{\delta}_f$ are (co)algebra morphisms. With these conditions made, we have the definition of an \emph{algebraic weak factorization system}, first given in \cite{gt:nwfs} (there called \emph{natural} weak factorization systems), and further refined in \cite{garner:nwfs} and \cite{garner:soa}.

\begin{definition}\label{Def:Awfs}
	An \emph{algebraic weak factorization system} on a category $\cat{C}$ consists of a functorial factorization $(L,\vec{\epsilon},R,\vec{\eta})$ together with natural transformations $\vec{\mu}\colon R^2f\Rightarrow Rf$ and $\vec{\delta}\colon L\Rightarrow L^2$, such that
	\begin{itemize}
		\item $\mathbb{R}=(R,\vec{\eta},\vec{\mu})$ is a monad and $\mathbb{L}=(L,\vec{\epsilon},\vec{\delta})$ a comonad on $\cat{C}^2$, and
		\item the natural transformation $\Delta=(\delta,\mu)\colon LR\Rightarrow RL$ determined by the equation $\epsilon L\circ\delta=\mu\circ\eta R \, (=\id_E)$ as in \ref{Lem:ArrowObject} is a distributive law, which in this case reduces to the single condition $\delta\circ\mu = \mu L\circ E\Delta\circ\delta R$.
	\end{itemize}
\end{definition}

\chapter{2-Fold Double Categories}

In this section we will propose a generalization of the 2-fold monoidal categories as used in \cite{garner:soa}.

A 2-fold double category $\mathbb{D}$ is a structure which has two different underlying double categories, both of which have the same vertical category $\dvert(\mathbb{D})$. We will start with a concise formal definition, and then expand on the definition more concretely.

\begin{definition}
	A \emph{2-fold double category} $\mathbb{D}$ with vertical category $\dvert(\mathbb{D})=\cat{D}_0$ is a 2-fold monoid object in the 2-category $\mathcal{C}at/\cat{D}_0$ of categories over $\cat{D}_0$.
\end{definition}

Breaking this down, we have a category $\cat{D}$, a functor $p\colon\cat{D}\to\cat{D}_0$, two functors $\otimes, \odot\colon \cat{D}\times_{\cat{D}_0}\cat{D}\to\cat{D}$ commuting with $p$, and two functors $I,\perp\colon\cat{D}_0\to\cat{D}$ which are sections of $p$, such that $\otimes$, $\odot$, $I$, and $\perp$ satisfy all the axioms of a 2-fold monoidal category. In particular, each fiber of $p$ has a 2-fold monoidal structure.

We will find it convenient to present this structure in the form of a double category $\mathbb{D}$, as follows: 
\begin{itemize}
	\item The objects and vertical morphisms of $\mathbb{D}$ are those of $\cat{D}_0$, so that $\dvert(\mathbb{D})=\cat{D}_0$.
	\item The horizontal morphisms of $\mathbb{D}$ are the objects $X$ of $\cat{D}$, with $p(X)$ as both domain and codomain. We will draw these as marked arrows 
	\[
	\begin{tikzcd}
		p(X) \rar[tick]{X} & p(X).
	\end{tikzcd}
	\]
	\item The 2-cells are the morphisms of $\cat{D}$. So a morphism $\phi\colon X\to Y$ in $\cat{D}$ with $p(\phi)=f\colon C\to D$ would be drawn as
	\[\begin{tikzcd}
		C \rar[tick][domA]{X} \dar[swap]{f} 
			& C \dar{f} \\
		D \rar[tick][codA,swap]{Y} 
			& D
		 \twocellA{\phi}
	\end{tikzcd}\]
\end{itemize}
The two tensor products of $\cat{D}$ provide two different horizontal compositions for $\mathbb{D}$. For any object $C$ there are 2-cells
\begin{equation}\label{Eq:2FoldCoherenceCellsA}
\begin{tikzcd}[column sep=large]
	C \rar[tick][domA]{\perp_C \otimes \perp_C} \dar[equal] 
		& C \dar[equal] \\
	C \rar[tick][codA,swap]{\perp_C} 
		& C
	 \twocellA{m}
\end{tikzcd} \qquad
\begin{tikzcd}[column sep=large]
	C \rar[tick][domA]{I_C} \dar[equal] 
		& C \dar[equal] \\
	C \rar[tick][codA,swap]{I_C\odot I_C} 
		& C
	 \twocellA{c}
\end{tikzcd} \qquad
\begin{tikzcd}
	C \rar[tick][domA]{I_C} \dar[equal]
		& C \dar[equal] \\
	C \rar[tick][codA,swap]{\perp_C} 
		& C
	 \twocellA{j} 
\end{tikzcd}
\end{equation}
and for any four horizontal morphisms $\begin{tikzcd}[baseline,column sep=2.5ex] W,X,Y,Z\colon C \rar[tick]& C \end{tikzcd}$ there is a 2-cell
\begin{equation}\label{Eq:2FoldCoherenceCellsB}
\begin{tikzcd}[column sep=huge]
	C \rar[tick][domA]{(W\odot X)\otimes(Y\odot Z)} 
			\dar[equals] 
		& C \dar[equals] \\
	C \rar[tick][codA,swap]{(W\otimes Y)\odot(X\otimes Z)} 
		& C.
	\twocellA{z}
\end{tikzcd}
\end{equation}
These are natural in the sense that, for any vertical morphism $f\colon C\to D$ we have an equality
\[
\begin{tikzcd}[column sep=large]
	C \rar[tick][domA]{\perp_C \otimes \perp_C} 
			\dar[equal] 
		& C \dar[equal] \\
	C \rar[tick][codA,domB]{\perp_C} 
			\dar[swap]{f} 
		& C \dar{f} \\
	D \rar[tick][codB,swap]{\perp_D} 
		& D
	\twocellA{m}
	\twocellB{\perp_f}
\end{tikzcd}
=
\begin{tikzcd}[column sep=large]
	C \rar[tick][domA]{\perp_C \otimes \perp_C} 
			\dar[swap]{f} 
		& C \dar{f} \\
	D \rar[tick][codA,domB,swap]{\perp_D \otimes \perp_D} 
			\dar[equal] 
		& D \dar[equal] \\
	D \rar[tick][codB,swap]{\perp_D} 
		& D
	\twocellA{\perp_f \otimes \perp_f}
	\twocellB{m} 
\end{tikzcd}
\]
and similarly for $c$ and $j$, and for any four 2-cells $\theta_1,\dots,\theta_4$ of the apropriate form, we have an equality
\[
\begin{tikzcd}[column sep=17ex]
	C \rar[tick][domA]{(W\odot X)\otimes(Y\odot Z)} 
			\dar[equals] 
		& C \dar[equals] \\
	C \rar[tick][codA,domB]{(W\otimes Y)\odot(X\otimes Z)} 
			\dar[swap]{f} 
		& C \dar{f} \\
	D \rar[tick][codB,swap]{(W'\otimes Y')\odot(X'\otimes Z')} 
		& D
	\twocellA{z}
	\twocellB{(\theta_1\otimes\theta_3)\odot(\theta_2\otimes\theta_4)}
\end{tikzcd}
=
\begin{tikzcd}[column sep=17ex]
	C \rar[tick][domA]{(W\odot X)\otimes(Y\odot Z)} 
			\dar[swap]{f} 
		& C \dar{f} \\
	C \rar[tick][codA,domB,swap]{(W'\odot X')\otimes(Y'\odot Z')} 
			\dar[equals]  
		& C \dar[equals] \\
	D \rar[tick][codB,swap]{(W'\otimes Y')\odot(X'\otimes Z')} 
		& D
	\twocellA{(\theta_1\odot\theta_2)\otimes(\theta_3\odot\theta_4)}
	\twocellB{z}
\end{tikzcd}
\]

A 2-fold double category $\mathbb{D}$ has two underlying ordinary double categories: $\mathbb{D}_{\otimes}$ using $\otimes$ for the horizontal composition, and $\mathbb{D}_{\odot}$ using $\odot$.

\chapter{Monoids}

There are several possible ways to define (co)monoids/(co)monads in a double category, but there is one way in particular which interacts nicely with the 2-fold double category structure.

\begin{definition}
	A \emph{monoid} in a 2-fold double category $\mathbb{D}$ is a monoid in $\mathbb{D}_{\otimes}$. Specifically, it is a horizontal 1-cell $\begin{tikzcd}[baseline,column sep=2.5ex] X\colon C \rar[tick]& C \end{tikzcd}$, together with unit and multiplication 2-cells
	\[\begin{tikzcd}
		C \rar[tick][domA]{I_C} \dar[equal]
			& C \dar[equal] \\
		C \rar[tick][codA,swap]{X} 
			& C
		\twocellA{\eta}
	\end{tikzcd} \qquad
	\begin{tikzcd}
		C \rar[tick][domA]{X\otimes X} \dar[equal]
			& C \dar[equal] \\
		C \rar[tick][codA,swap]{X} 
			& C
		\twocellA{\mu} 
	\end{tikzcd}
	\]
	satisfying the usual unit and associativity conditions.
\end{definition}

\begin{definition}
	Given two monoids $(C,X,\eta,\mu)$ and $(D,Y,\eta',\mu')$ in $\mathbb{D}$, a morphism of monoids consists of a vertical morphism $f\colon C\to D$ and a 2-cell
	\[
	\begin{tikzcd}
		C \rar[tick][domA]{X} \dar[swap]{f}  
			& C \dar{f} \\
		D \rar[tick][codA,swap]{Y} 
			& D
		\twocellA{\phi}
	\end{tikzcd}
	\]
	which preserves the unit and multiplication, in that
	\begin{equation}\label{Eq:MonoidMorphismUnit}
	\begin{tikzcd}
		C \rar[tick][domA]{I_C} \dar[equal] 
			& C \dar[equal] \\
		C \rar[tick][codA,domB,swap]{X} \dar[swap]{f} 
			& C \dar{f} \\
		D \rar[tick][codB,swap]{Y}
			& D
		\twocellA{\eta}
		\twocellB{\phi}
	\end{tikzcd}
	=
	\begin{tikzcd}
		C \rar[tick][domA]{I_C} \dar[swap]{f} 
			& C \dar{f} \\
		D \rar[tick][codA,domB,swap]{I_D} \dar[equal] 
			& D \dar[equal] \\
		D \rar[tick][codB,swap]{Y}
			& D
		\twocellA{I_f}
		\twocellB{\eta'}
	\end{tikzcd}
	\end{equation}
	and
	\begin{equation}\label{Eq:MonoidMorphismMult}
	\begin{tikzcd}
		C \rar[tick][domA]{X\otimes X} \dar[equal] 
			& C \dar[equal] \\
		C \rar[tick][codA,domB,swap]{X} \dar[swap]{f} 
			& C \dar{f} \\
		D \rar[tick][codB,swap]{Y}
			& D
		\twocellA{\mu}
		\twocellB{\phi}
	\end{tikzcd}
	=
	\begin{tikzcd}
		C \rar[tick][domA]{X\otimes X} \dar[swap]{f} 
			& C \dar{f} \\
		D \rar[tick][codA,domB,swap]{Y\otimes Y} \dar[equal] 
			& D \dar[equal] \\
		D \rar[tick][codB,swap]{Y}
			& D.
		\twocellA{\phi\otimes\phi}
		\twocellB{\mu'}
	\end{tikzcd}
	\end{equation}
\end{definition}

\begin{definition}
	A comonoid in a 2-fold double category $\mathbb{D}$ is a comonoid in $\mathbb{D}_{\odot}$. Specifically, it is a horizontal 1-cell $\begin{tikzcd}[baseline,column sep=2.5ex] X\colon C \rar[tick]& C \end{tikzcd}$, together with counit and comultiplication 2-cells
	\[\begin{tikzcd}
		C \rar[tick][domA]{X} \dar[equal]  
			& C \dar[equal] \\
		C \rar[tick][codA,swap]{\perp_C} 
			& C
		\twocellA{\epsilon}
	\end{tikzcd} \qquad
	\begin{tikzcd}
		C \rar[tick][domA]{X} \dar[equal] 
			& C \dar[equal] \\
		C \rar[tick][codA,swap]{X\odot X} 
			& C
		\twocellA{\delta} 
	\end{tikzcd}
	\]
	satisfying the usual unit and associativity conditions.
\end{definition}

\begin{definition}
	Given two comonoids $(C,X,\epsilon,\delta)$ and $(D,Y,\epsilon',\delta')$ in $\mathbb{D}$, a morphism of comonoids consists of a vertical morphism $f\colon C\to D$ and a 2-cell
	\[
	\begin{tikzcd}
		C \rar[tick][domA]{X} \dar[swap]{f}  
			& C \dar{f} \\
		D \rar[tick][codA,swap]{Y} 
			& D
		\twocellA{\phi}
	\end{tikzcd}
	\]
	which preserves the counit and comultiplication, in that
	\begin{equation}\label{Eq:ComonoidMorphismCounit}
	\begin{tikzcd}
		C \rar[tick][domA]{X} \dar[equal] 
			& C \dar[equal] \\
		C \rar[tick][codA,domB,swap]{\perp_C} \dar[swap]{f} 
			& C \dar{f} \\
		D \rar[tick][codB,swap]{\perp_D}
			& D
		\twocellA{\epsilon}
		\twocellB{\perp_f}
	\end{tikzcd}
	=
	\begin{tikzcd}
		C \rar[tick][domA]{X} \dar[swap]{f} 
			& C \dar{f} \\
		D \rar[tick][codA,domB,swap]{Y} \dar[equal] 
			& D \dar[equal] \\
		D \rar[tick][codB,swap]{\perp_D}
			& D
		\twocellA{\phi}
		\twocellB{\epsilon'}
	\end{tikzcd}
	\end{equation}
	and
	\begin{equation}\label{Eq:ComonoidMorphismComult}
	\begin{tikzcd}
		C \rar[tick][domA]{X} \dar[equal] 
			& C \dar[equal] \\
		C \rar[tick][codA,domB,swap]{X\odot X} \dar[swap]{f} 
			& C \dar{f} \\
		D \rar[tick][codB,swap]{Y\odot Y}
			& D
		\twocellA{\delta}
		\twocellB{\phi\odot\phi}
	\end{tikzcd}
	=
	\begin{tikzcd}
		C \rar[tick][domA]{X} \dar[swap]{f} 
			& C \dar{f} \\
		D \rar[tick][codA,domB,swap]{Y} \dar[equal] 
			& D \dar[equal] \\
		D \rar[tick][codB,swap]{Y\odot Y}
			& D.
		\twocellA{\phi}
		\twocellB{\delta'}
	\end{tikzcd}
	\end{equation}
\end{definition}

The 2-fold double category structure on $\mathbb{D}$ allows us to form double categories $\mathbb{M}\mathrm{on}(\mathbb{D})$ and $\mathbb{C}\mathrm{omon}(\mathbb{D})$ of (co)monoids in $\mathbb{D}$, in which the objects and vertical morphisms are the same as in $\mathbb{D}$, and the horizontal 1-cells and 2-cells are (co)monoids and (co)monoid morphisms in $\mathbb{D}$. The only thing we still have to provide is a horizontal composition.

Given two monoids $(C,X,\eta,\mu)$ and $(C,Y,\eta',\mu')$ in $\mathbb{D}$, thought of as horizontal 1-cells $\begin{tikzcd}[baseline,column sep=2.5ex] C \rar[tick]& C \end{tikzcd}$ in $\mathbb{M}\mathrm{on}(\mathbb{D})$, the horizontal composition 
\[
\begin{tikzcd}[column sep=large]
	C \rar[tick]{(X,\eta,\mu)} & C \rar[tick]{(Y,\eta',\mu')} & C
\end{tikzcd}
\]
is the monoid with underlying horizontal 1-cell $X\odot Y$ and unit and multiplication 2-cells
\[
\begin{tikzcd}[column sep=7ex]
	C \rar[tick][domA]{I_C} \dar[equal] 
		& C \dar[equal] \\
	C \rar[tick][codA,domB]{I_C\odot I_C} 
			\dar[equal] 
		& C \dar[equal] \\
	C \rar[tick][codB,swap]{X\odot Y}
		& C
	\twocellA{c}
	\twocellB{\eta\odot\eta'}
\end{tikzcd}
\qquad
\begin{tikzcd}[column sep=14ex]
	C \rar[tick][domA]{(X\odot Y)\otimes(X\odot Y)} 
			\dar[equal] 
		& C \dar[equal] \\
	C \rar[tick][codA,domB]{(X\otimes X)\odot(Y\otimes Y)} 
			\dar[equal] 
		& C \dar[equal] \\
	C \rar[tick][codB,swap]{X\odot Y}
		& C.
	\twocellA{z}
	\twocellB{\mu\odot\mu'}
\end{tikzcd}
\]
Similarly, the horizontal composition of two 2-cells in $\mathbb{M}\mathrm{on}(\mathbb{D})$ is the $\odot$ product of the underlying 2-cells in $\mathbb{D}$. The fact that this commutes with the unit and multiplication defined above follows from the naturality of $c$ and $z$.

In this same way, we can define the horizontal composition of two 1-cells $(X,\epsilon,\delta)$ and $(Y,\epsilon',\delta')$ in $\mathbb{C}\mathrm{omon}(\mathbb{D})$ to be a comonoid with underlying horizontal 1-cell $X\otimes Y$.

This allows us to define (ordinary) categories $\mathrm{Mon}(\mathbb{C}\mathrm{omon}(\mathbb{D}))$ and $\mathrm{Comon}(\mathbb{M}\mathrm{on}(\mathbb{D}))$. Furthermore, these two categories are equivalent, leading to the next definition.

\begin{definition}
	A \emph{bimonoid} in a 2-fold double category $\mathbb{D}$ is a monoid in $\mathbb{C}\mathrm{omon}(\mathbb{D})$, or equivalently a comonoid in $\mathbb{M}\mathrm{on}(\mathbb{D})$. We can define a category of bimonoids in $\mathbb{D}$ as
	\[
		\mathrm{Bimon}(\mathbb{D}) := \mathrm{Mon}(\mathbb{C}\mathrm{omon}(\mathbb{D})) \simeq \mathrm{Comon}(\mathbb{M}\mathrm{on}(\mathbb{D}))
	\]
\end{definition}

Concretely, a bimonoid in $\mathbb{D}$ is a tuple $(X,\eta,\mu,\epsilon,\delta)$ where $X$ is a horizontal 1-cell, $(X,\eta,\mu)$ is a monoid and $(X,\epsilon,\delta)$ is a comonoid as above, such that four equations hold:
\begin{equation}\label{Eq:Bimonoid}
\begin{gathered}
\begin{tikzcd}[ampersand replacement=\&]
	C \rar[tick][domA]{I_C} \dar[equal] 
		\& C \dar[equal] \\
	C \rar[tick][codA,domB]{X} \dar[equal] 
		\& C \dar[equal] \\
	C \rar[tick][codB,swap]{X\odot X} \& C
	\twocellA{\eta}
	\twocellB{\delta}
\end{tikzcd}
=
\begin{tikzcd}[ampersand replacement=\&]
	C \rar[tick][domA]{I_C} \dar[equal] 
		\& C \dar[equal] \\
	C \rar[tick][codA,domB]{I_C\odot I_C} \dar[equal] 
		\& C \dar[equal] \\
	C \rar[tick][codB,swap]{X\odot X} \& C
	\twocellA{c}
	\twocellB{\eta\odot\eta}
\end{tikzcd}
\qquad
\begin{tikzcd}[ampersand replacement=\&]
	C \rar[tick][domA]{X\otimes X} \dar[equal] 
		\& C \dar[equal] \\
	C \rar[tick][codA,domB]{X} \dar[equal] 
		\& C \dar[equal] \\
	C \rar[tick][codB,swap]{\perp_C} \& C
	\twocellA{\mu}
	\twocellB{\epsilon}
\end{tikzcd}
=
\begin{tikzcd}[ampersand replacement=\&]
	C \rar[tick][domA]{X\otimes X} \dar[equal] 
		\& C \dar[equal] \\
	C \rar[tick][codA,domB]{\perp_C\otimes\perp_C} \dar[equal] 
		\& C \dar[equal] \\
	C \rar[tick][codB,swap]{\perp_C} \& C
	\twocellA{\epsilon\otimes\epsilon}
	\twocellB{m}
\end{tikzcd}
\\
\begin{tikzcd}[ampersand replacement=\&]
	C \rar[tick][domA]{I_C} \dar[equal] 
		\& C \dar[equal] \\
	C \rar[tick][codA,domB]{X} \dar[equal] 
		\& C \dar[equal] \\
	C \rar[tick][codB,swap]{\perp_C} \& C
	\twocellA{\eta}
	\twocellB{\epsilon}
\end{tikzcd}
=
\begin{tikzcd}[ampersand replacement=\&]
	C \rar[tick][domA]{I_C} \dar[equal] 
		\& C \dar[equal] \\
	C \rar[tick][codA,swap]{\perp_C} \& C
	\twocellA{j}
\end{tikzcd}
\qquad
\begin{tikzcd}[ampersand replacement=\&,column sep=14ex]
	C \rar[tick][domA]{X\otimes X} \dar[equal] 
		\& C \dar[equal] \\
	C \rar[tick][codA,domB]{(X\odot X)\otimes(X\odot X)} \dar[equal] 
		\& C \dar[equal] \\
	C \rar[tick][codB,domC]{(X\otimes X)\odot(X\otimes X)} \dar[equal] 
		\& C \dar[equal] \\
	C \rar[tick][codC,swap]{X\odot X} \& C
	\twocellA{\delta\otimes\delta}
	\twocellB{z}
	\twocellC{\mu\odot\mu}
\end{tikzcd}
=
\begin{tikzcd}[ampersand replacement=\&]
	C \rar[tick][domA]{X\otimes X} \dar[equal]
		\& C \dar[equal] \\
	C \rar[tick][codA,domB]{X} \dar[equal] 
		\& C \dar[equal] \\
	C \rar[tick][codB,swap]{X\odot X} \& C
	\twocellA{\mu}
	\twocellB{\delta}
\end{tikzcd}
\end{gathered}
\end{equation}
A bimonoid morphism is simply a 2-cell which is simultaineously a monoid morphism and a comonoid morphism.

\chapter{Cyclic 2-fold Double Categories}

Recall the notion of a cyclic double category from \cite{cgr:mates}. A cyclic double category $\mathbb{D}$ is a double category with an extra involutive operation. On objects and horizontal 1-cells $\begin{tikzcd}[baseline,column sep=2.5ex] X\colon C \rar[tick]& C \end{tikzcd}$, this operation is written
\[
\begin{tikzcd}
	C^{\bullet} \rar[tick][]{X^{\bullet}} & C^{\bullet}
\end{tikzcd}
\]
and respects horizontal identities and composition. The involution takes any vertical 1-cell $f\colon C\to D$ to some $\sigma f\colon D^{\bullet}\to C^{\bullet}$, and any 2-cell
\[
\begin{tikzcd}
	C \rar[tick][domA]{X} \dar[swap]{f} 
		& C \dar{f} \\
	D \rar[tick][codA,swap]{Y}
		& D
	\twocellA{\theta}
\end{tikzcd}
\qquad\text{to}\qquad
\begin{tikzcd}
	D^{\bullet} \rar[tick][domA]{Y^{\bullet}} 
			\dar[swap]{\sigma f} 
		& D^{\bullet} \dar{\sigma f}\\
	C^{\bullet} \rar[tick][codA,swap]{X^{\bullet}}
		& C^{\bullet}
	\twocellA{\sigma\theta}
\end{tikzcd}
\]
respecting vertical identities and composition.

We will generalize this to a cyclic action on a 2-fold double category. Suppose that $\mathbb{D}$ is a 2-fold double category. A cyclic action, written as above, must satisfy the following:
\begin{itemize}
	\item For every object $C$,
	\[
		I_{C^{\bullet}}=(\perp_C)^{\bullet} \qquad\text{and}\qquad \perp_{C^{\bullet}}=(I_C)^{\bullet}.
	\]
	\item For every composable pair of horizontal 1-cells $\begin{tikzcd}[baseline,column sep=2.5ex] X,Y\colon C \rar[tick]& C \end{tikzcd}$,
	\[
		(X\otimes Y)^{\bullet} = X^{\bullet}\odot Y^{\bullet} \qquad\text{and}\qquad
		(X\odot Y)^{\bullet} = X^{\bullet}\otimes Y^{\bullet}
	\]
	\item For every vertical 1-cell $f\colon C\to D$, there are equalities
	\[
	\begin{tikzcd}
		D^{\bullet} \rar[tick][codA]{I_{D^{\bullet}}} 
				\dar[swap]{\sigma f} 
			& D^{\bullet} \dar{\sigma f} \\
		C^{\bullet} \rar[tick][codA,swap]{I_{C^{\bullet}}}
			& C^{\bullet}
		\twocellA{I_{\sigma f}}
	\end{tikzcd}
	=
	\begin{tikzcd}
		D^{\bullet} \rar[tick][domA]{(\perp_D)^{\bullet}} 
				\dar[swap]{\sigma f} 
			& D^{\bullet} \dar{\sigma f} \\
		C^{\bullet} \rar[tick][codA,swap]{(\perp_C)^{\bullet}}
			& C^{\bullet}
		\twocellA{\sigma\perp_f}
	\end{tikzcd}
	\]
	\[
	\begin{tikzcd}
		D^{\bullet} \rar[tick][domA]{\perp_{D^{\bullet}}} 
				\dar[swap]{\sigma f} 
			& D^{\bullet} \dar{\sigma f} \\
		C^{\bullet} \rar[tick][codA,swap]{\perp_{C^{\bullet}}}
			& C^{\bullet}
		\twocellA{\perp_{\sigma f}}
	\end{tikzcd}
	=
	\begin{tikzcd}
		D^{\bullet} \rar[tick][domA]{(I_D)^{\bullet}} 
				\dar[swap]{\sigma f} 
			& D^{\bullet} \dar{\sigma f} \\
		C^{\bullet} \rar[tick][codA,swap]{(I_C)^{\bullet}}
			& C^{\bullet}
		\twocellA{\sigma I_f}
	\end{tikzcd}
	\]
	\item For every horizontally composable pair of 2-cells
	\[
	\begin{tikzcd}
		C \rar[tick][domA]{X} \dar[swap]{f} 
			& C \rar[tick][domB]{Y} \dar{f} 
			& C \dar{f} \\
		D \rar[tick][codA,swap]{X'}
			& D \rar[tick][codB,swap]{Y'}
			& D
		\twocellA{\theta}
		\twocellB{\phi}
	\end{tikzcd}
	\]
	there are equalities
	\[
	\begin{tikzcd}[column sep=large]
		D^{\bullet} \rar[tick][domA]{(X'\otimes Y')^{\bullet}} 
				\dar[swap]{\sigma f} 
			& D^{\bullet} \dar{\sigma f}\\
		C^{\bullet} \rar[tick][codA,swap]{(X\otimes Y)^{\bullet}}
			& C^{\bullet}
		\twocellA{\sigma(\theta\otimes\phi)}
	\end{tikzcd}
	=
	\begin{tikzcd}[column sep=large]
		D^{\bullet} \rar[tick][domA]{X'^{\bullet}\odot Y'^{\bullet}} 
				\dar[swap]{\sigma f} 
			& D^{\bullet} \dar{\sigma f}\\
		C^{\bullet} \rar[tick][codA,swap]{X^{\bullet}\odot Y^{\bullet}}
			& C^{\bullet}
		\twocellA{\sigma(\theta)\odot\sigma(\phi)}
	\end{tikzcd}
	\]
	\[
	\begin{tikzcd}[column sep=large]
		D^{\bullet} \rar[tick][domA]{(X'\odot Y')^{\bullet}} 
				\dar[swap]{\sigma f} 
			& D^{\bullet} \dar{\sigma f}\\
		C^{\bullet} \rar[tick][codA,swap]{(X\odot Y)^{\bullet}}
			& C^{\bullet}
		\twocellA{\sigma(\theta\odot\phi)}
	\end{tikzcd}
	=
	\begin{tikzcd}[column sep=large]
		D^{\bullet} \rar[tick][domA]{X'^{\bullet}\otimes Y'^{\bullet}} 
				\dar[swap]{\sigma f} 
			& D^{\bullet} \dar{\sigma f}\\
		C^{\bullet} \rar[tick][codA,swap]{X^{\bullet}\otimes Y^{\bullet}}
			& C^{\bullet}
		\twocellA{\sigma(\theta)\otimes\sigma(\phi)}
	\end{tikzcd}
	\]
\end{itemize}

One nice consequence of this definition is that a cyclic action on a 2-fold double category $\mathbb{D}$ induces a cyclic action on the category of bimonoids $\mathrm{Bimon}(\mathbb{D})$.

\begin{proposition}
	Suppose $\mathbb{D}$ is a cyclic 2-fold double category. Then the category $\mathrm{Bimon}(\mathbb{D})$ of bimonoids in $\mathbb{D}$ carries a natural cyclic action (contravariant isomorphism).
\end{proposition}
\begin{proof}
	The involution $(-)^{\bullet}$ gives an isomorphism of double categories $\mathbb{D}_{\otimes}\cong\mathbb{D}_{\odot}^{\mathrm{op}}$. Therefore it also induces an isomorphism
\[
	\mathbb{M}\mathrm{on}(\mathbb{D}) = \mathbb{M}\mathrm{on}(\mathbb{D}_{\otimes}) 
		\cong \mathbb{M}\mathrm{on}(\mathbb{D}_{\odot}^{\mathrm{op}}) 
		\cong \mathbb{C}\mathrm{omon}(\mathbb{D}_{\odot})^{\mathrm{op}} 
		= \mathbb{C}\mathrm{omon}(\mathbb{D})^{\mathrm{op}}
\]
as well as an isomorphism
\begin{multline*}
	\mathrm{Bimon}(\mathbb{D}) = \mathrm{Comon}(\mathbb{M}\mathrm{on}(\mathbb{D}))
		\cong \mathrm{Comon}(\mathbb{C}\mathrm{omon}(\mathbb{D})^{\mathrm{op}})
		\\ \cong \mathrm{Mon}(\mathbb{C}\mathrm{omon}(\mathbb{D}))^{\mathrm{op}}
		= \mathrm{Bimon}(\mathbb{D})^{\mathrm{op}}.
\end{multline*}
\end{proof}

In more concrete terms, the involution takes a bimonoid $(X,\eta,\mu,\epsilon,\delta)$ to $(X,\eta,\mu,\epsilon,\delta)^{\bullet}=(X^\bullet,\epsilon^\bullet,\delta^\bullet,\eta^\bullet,\delta^\bullet)$, swapping the monoid and comonoid structures. This is again a bimonoid, as the top two equations of \eqref{Eq:Bimonoid} are interchanged under the involution, while the bottom two equations are self-dual.

The action of the involution on bimonoid morphisms can be broken down as in the following lemma.

\begin{lemma}
	Let $(X,\eta,\mu,\epsilon,\delta)$ and $(Y,\eta',\mu',\epsilon',\delta')$ be bimonoids in a cyclic 2-fold double category, and let $\phi$ be a 2-cell
	\[
	\begin{tikzcd}
		C \rar[tick][domA]{X} \dar[swap]{f} & C \dar{f} \\
		D \rar[tick][codA,swap]{Y} & D.
		\twocellA{\phi}
	\end{tikzcd}
	\]
	Then $\phi$ is a monoid morphism $X\to Y$ if and only if $\phi^\bullet$ is a comonoid morphism $Y^\bullet\to X^\bullet$. Dually, $\phi$ is a comonoid morphism $X\to Y$ if and only if $\phi^\bullet$ is a monoid morphism $Y^\bullet\to X^\bullet$.
\end{lemma}
\begin{proof}
	Simply notice that the involution interchanges equations \eqref{Eq:MonoidMorphismUnit} and \eqref{Eq:MonoidMorphismMult} with \eqref{Eq:ComonoidMorphismCounit} and \eqref{Eq:ComonoidMorphismComult}.
\end{proof}

\chapter{Functorial Factorizations}

Now let $\mathbb{D}$ be a cyclic double category, and assume it has arrow objects in the sense of Section~\ref{Ch:PushoutProduct}. Let us spell out what this means for the single variable case:
\begin{itemize}
	\item For every object $C$ there is a diagram
	\[
	\begin{tikzcd}[bend angle=30]
		C^2 \rar[bend left][domA]{\dom}
			\rar[bend right][codA,swap]{\cod}
		& C.
		\twocellA{\kappa}
	\end{tikzcd}
	\]
	\item Any 2-cell 
	\[
	\begin{tikzcd}[bend angle=30]
		A \rar[bend left][domA]{d_1}
			\rar[bend right][codA,swap]{d_0}
		& C
		\twocellA{\alpha}
	\end{tikzcd}
	\]
	uniquely factors through $\kappa$, as
	\[
	\begin{tikzcd}[bend angle=30]
		A \rar{\hat{\alpha}} 
			& C^2 \rar[bend left][domA]{\dom}
				\rar[bend right][codA,swap]{\cod}
			& C.
			\twocellA{\kappa}
	\end{tikzcd}
	\]
	\item For every vertical 1-cell $F\colon C\to D$ there is a vertical 1-cell $\hat{F}\colon C^2\to D^2$ and 2-cells
	\[
	\begin{tikzcd}
		C^2 \rar[][domA]{\dom} \dar[swap]{\hat{F}} & C \dar{F} \\
		D^2 \rar[][codA,swap]{\dom} & D
		\twocellA{\gamma_1}
	\end{tikzcd}
	\qquad
	\begin{tikzcd}
		C^2 \rar[][domA]{\cod} \dar[swap]{\hat{F}} & C \dar{F} \\
		D^2 \rar[][codA,swap]{\cod} & D
		\twocellA{\gamma_0}
	\end{tikzcd}
	\]
	such that
	\[
	\begin{tikzcd}[bend angle=50]
		C^2 \rar[][domA]{\dom} 
			\dar[swap]{\hat{F}} 
		& C \dar{F} \\
		D^2 \rar[][codA,domB]{\dom}	
			\rar[bend right][codB,swap]{\cod}
		& D
		\twocellA{\gamma_1}
		\twocellB{\kappa}
	\end{tikzcd}
	=
	\begin{tikzcd}[bend angle=50]
		C^2 \rar[bend left][domA]{\dom} 
			\rar[][codA,domB,swap]{\cod} 
			\dar[swap]{\hat{F}} 
		& C \dar{F} \\
		D^2 \rar[][codB,swap]{\cod} & D
		\twocellA{\kappa}
		\twocellB{\gamma_0}
	\end{tikzcd}
	\]
	\item Given any 2-cells
	\[
	\begin{tikzcd}[bend angle=30]
		A \rar[bend left][domA]{d_1}
			\rar[bend right][codA,swap]{d_0}
		& C
		\twocellA{\alpha}
	\end{tikzcd}
	\quad
	\begin{tikzcd}[bend angle=30]
		B \rar[bend left][domA]{d'_1}
			\rar[bend right][codA,swap]{d'_0}
		& D
		\twocellA{\alpha'}
	\end{tikzcd}
	\]
	and
	\[
	\begin{tikzcd}
		A \rar[][domA]{d_1} \dar[swap]{G} & C \dar{F} \\
		B \rar[][codA,swap]{d'_1} & D
		\twocellA{\lambda_1}
	\end{tikzcd}
	\qquad
	\begin{tikzcd}
		A \rar[][domA]{d_0} \dar[swap]{G} & C \dar{F} \\
		B \rar[][codA,swap]{d'_0} & D
		\twocellA{\lambda_0}
	\end{tikzcd}
	\]
	such that
	\[
	\begin{tikzcd}[bend angle=50]
		A \rar[][domA]{d_1} 
			\dar[swap]{G} 
		& C \dar{F} \\
		B \rar[][codA,domB]{d'_1}	
			\rar[bend right][codB,swap]{d'_0}
		& D
		\twocellA{\lambda_1}
		\twocellB{\alpha'}
	\end{tikzcd}
	=
	\begin{tikzcd}[bend angle=50]
		A \rar[bend left][domA]{d_1} 
			\rar[][codA,domB,swap]{d_0} 
			\dar[swap]{G} 
		& C \dar{F} \\
		B \rar[][codB,swap]{d'_0} & D
		\twocellA{\alpha}
		\twocellB{\lambda_0}
	\end{tikzcd}
	\]
	there is a unique 2-cell
	\[
	\begin{tikzcd}
		A \rar[][domA]{\hat{\alpha}} \dar[swap]{G} & C^2 \dar{\hat{F}} \\
		B \rar[][codA,swap]{\hat{\alpha}'} & D^2
		\twocellA{\theta}
	\end{tikzcd}
	\]
	such that the horizontal composition of $\theta$ with $\gamma_0$ and $\gamma_1$ is respectively equal to $\lambda_0$ and $\lambda_1$.
\end{itemize}

\begin{remark}
	[TODO: Remark that this generalizes the 2-dimensional part of the usual universal property in 2-categories.]
\end{remark}

We will now define a 2-fold double category $\FF{D}$ of functorial factorizations in $\mathbb{D}$, as follows:
\begin{itemize}
	\item The objects and vertical 1-cells are the same as in $\mathbb{D}$.

	\item Horizontal 1-cells $\begin{tikzcd}[baseline,column sep=2.5ex] C \rar[tick]& C \end{tikzcd}$ in $\FF{D}$ are tuples $(E,\eta,\epsilon)$, where $E\colon C^2\to C$ is a horizontal 1-cell in $\mathbb{D}$, and
	\[
	\begin{tikzcd}[bend angle=30]
		C^2 \rar[bend left][domA]{\dom}
			\rar[bend right][codA,swap]{E}
		& C
		\twocellA{\eta}
	\end{tikzcd}
	\qquad
	\begin{tikzcd}[bend angle=30]
		C^2 \rar[bend left][domA]{E}
			\rar[bend right][codA,swap]{\cod}
		& C
		\twocellA{\epsilon}
	\end{tikzcd}
	\]
	are 2-cells in $\mathbb{D}$ such that 
	\[
	\begin{tikzcd}[bend angle=50]
		C^2 \rar[bend left][domA]{\dom}
			\rar[][codA,domB,description]{E}
			\rar[bend right][codB,swap]{\cod}
		& C
		\twocellA[pos=.45]{\eta}
		\twocellB[pos=.55]{\epsilon}
	\end{tikzcd}
	=
	\begin{tikzcd}[bend angle=30]
		C^2 \rar[bend left][domA]{\dom}
			\rar[bend right][codA,swap]{\cod}
		& C.
		\twocellA{\kappa}
	\end{tikzcd}
	\]

	By the universal property of $C^2$, this also determines horizontal 1-cells $L,R\colon C^2\to C^2$ such that $\dom\circ L=\dom$, $\cod\circ R=\cod$, $\cod\circ L=\dom\circ R=E$, $\kappa\circ L=\eta$, and $\kappa\circ R=\epsilon$, and 2-cells
	\[
	\begin{tikzcd}[bend angle=30]
		C^2 \rar[bend left][domA]{L}
			\rar[bend right][codA,swap]{\id}
		& C^2.
		\twocellA{\vec{\epsilon}}
	\end{tikzcd}
	\qquad
	\begin{tikzcd}[bend angle=30]
		C^2 \rar[bend left][domA]{\id}
			\rar[bend right][codA,swap]{R}
		& C^2.
		\twocellA{\vec{\eta}}
	\end{tikzcd}
	\]
	such that $\dom\circ\vec{\epsilon}=\id_{\dom}$, $\cod\circ\vec{\epsilon}=\epsilon$, $\dom\circ\vec{\eta}=\eta$, and $\cod\circ\vec{\eta}=\id_{\cod}$.

	\item The horizontal composition $(E_1,\eta_1,\epsilon_1)\otimes(E_2,\eta_2,\epsilon_2)$ of two horizontal 1-cells
	\[
	\begin{tikzcd}[column sep=large]
		C \rar[tick][]{(E_1,\eta_1,\epsilon_1)} & C \rar[tick][]{(E_2,\eta_2,\epsilon_2)} & C
	\end{tikzcd}
	\]
	in $\FF{D}$ is a horizontal 1-cell $(E_{1\otimes2},\eta_{1\otimes2},\epsilon_{1\otimes2})$, where
	\[
	E_{1\otimes2} =
	\begin{tikzcd}
		C^2 \rar{R_1} & C^2 \rar{E_2} & C
	\end{tikzcd}
	\]
	\[
	\eta_{1\otimes2} =
	\begin{tikzcd}[bend angle=30,baseline=(B.base)]
		|[alias=B]| C^2 \rar[bend left][domA]{\id}
				\rar[bend right][codA,swap]{R_1} 
			& C^2 \rar[bend left][domB]{\dom}
				\rar[bend right][codB,swap]{E_2} 
			& C
		\twocellA{\vec{\eta_1}}
		\twocellB{\eta_2}
	\end{tikzcd}
	\]
	\[
	\epsilon_{1\otimes2} =
	\begin{tikzcd}[bend angle=30,baseline=(B.base)]
		|[alias=B]| C^2 \rar{R_1}
			& C^2 \rar[bend left][domA]{E_2}
				\rar[bend right][codA,swap]{\cod} 
			& C
		\twocellA{\epsilon_2}
	\end{tikzcd}
	\]
	which also determines that $R_{1\otimes2}=R_2\circ R_1$.

	\item The horizontal unit $I_C$ for $\otimes$ is $(\dom,\id,\kappa)$.

	\item The second horizontal composition $(E_1,\eta_1,\epsilon_1)\odot(E_2,\eta_2,\epsilon_2)$ is a horizontal 1-cell $(E_{1\odot2},\eta_{1\odot2},\epsilon_{1\odot2})$, where
	\[
	E_{1\odot2} =
	\begin{tikzcd}
		C^2 \rar{L_1} & C^2 \rar{E_2} & C
	\end{tikzcd}
	\]
	\[
	\eta_{1\odot2} =
	\begin{tikzcd}[bend angle=30,baseline=(B.base)]
		|[alias=B]| C^2 \rar{L_1}
			& C^2 \rar[bend left][domA]{\dom}
				\rar[bend right][codA,swap]{E_2} 
			& C
		\twocellA{\eta_2}
	\end{tikzcd}
	\]
	\[
	\epsilon_{1\odot2} =
	\begin{tikzcd}[bend angle=30,baseline=(B.base)]
		|[alias=B]| C^2 \rar[bend left][domA]{L_1}
				\rar[bend right][codA,swap]{\id} 
			& C^2 \rar[bend left][domB]{E_2}
				\rar[bend right][codB,swap]{\dom} 
			& C
		\twocellA{\vec{\epsilon_1}}
		\twocellB{\epsilon_2}
	\end{tikzcd}
	\]
	which also determines that $L_{1\odot2}=L_2\circ L_1$.

	\item The horizontal unit $\perp_C$ for $\odot$ is $(\cod,\kappa,\id)$.

	\item 2-cells
	\[
	\begin{tikzcd}[column sep=large]
		C \rar[tick][domA]{(E_1,\eta_1,\epsilon_1)} \dar[swap]{F}  & C \dar{F} \\
		D \rar[tick][codA,swap]{(E_2,\eta_2,\epsilon_2)} & D
		\twocellA{\theta}
	\end{tikzcd}
	\]
	in $\FF{D}$ are given by 2-cells
	\[
	\begin{tikzcd}
		C^2 \rar[][domA]{E_1} \dar[swap]{\hat{F}}  
			& C \dar{F} \\
		D^2 \rar[][codA,swap]{E_2} 
			& D
		\twocellA{\theta}
	\end{tikzcd}
	\]
	in $\mathbb{D}$ such that
	\begin{gather}
		\begin{tikzcd}[bend angle=50,ampersand replacement=\&]
			C^2 \rar[][domA]{E_1} 
				\dar[swap]{\hat{F}} 
			\& C \dar{F} \\
			D^2 \rar[][codA,domB]{E_2}	
				\rar[bend right][codB,swap]{\cod}
			\& D
			\twocellA{\theta}
			\twocellB{\epsilon_2}
		\end{tikzcd}
		=
		\begin{tikzcd}[bend angle=50,ampersand replacement=\&]
			C^2 \rar[bend left][domA]{E_1} 
				\rar[][codA,domB,swap]{\cod} 
				\dar[swap]{\hat{F}} 
			\& C \dar{F} \\
			D^2 \rar[][codB,swap]{\cod} \& D
			\twocellA{\epsilon_1}
			\twocellB{\gamma_0}
		\end{tikzcd}\label{Eq:FF2CellA}
		\\ \shortintertext{and}
		\begin{tikzcd}[bend angle=50,ampersand replacement=\&]
			C^2 \rar[][domA]{\dom} 
				\dar[swap]{\hat{F}} 
			\& C \dar{F} \\
			D^2 \rar[][codA,domB]{\dom}	
				\rar[bend right][codB,swap]{E_2}
			\& D
			\twocellA{\gamma_1}
			\twocellB{\eta_2}
		\end{tikzcd}\label{Eq:FF2CellB}
		=
		\begin{tikzcd}[bend angle=50,ampersand replacement=\&]
			C^2 \rar[bend left][domA]{\dom} 
				\rar[][codA,domB,swap]{E_1} 
				\dar[swap]{\hat{F}} 
			\& C \dar{F} \\
			D^2 \rar[][codB,swap]{E_2} \& D
			\twocellA{\eta_1}
			\twocellB{\theta}
		\end{tikzcd}
	\end{gather}

	This also determines unique 2-cells
	\[
	\begin{tikzcd}
		C^2 \rar[][domA]{R_1} \dar[swap]{\hat{F}} & C^2 \dar{\hat{F}} \\
		D^2 \rar[][codA,swap]{R_2} & D^2
		\twocellA{\theta^R}
	\end{tikzcd}
	\quad\text{and}\quad
	\begin{tikzcd}
		C^2 \rar[][domA]{L_1} \dar[swap]{\hat{F}} & C^2 \dar{\hat{F}} \\
		D^2 \rar[][codA,swap]{L_2} & D^2
		\twocellA{\theta^L}
	\end{tikzcd}
	\]
	such that composing horizontally with $\gamma_0$ or $\gamma_1$ gives $\gamma_0$, $\gamma_1$, or $\theta$ as appropriate. For instance:
	\[
	\begin{tikzcd}
		C^2 \rar[][domA]{R_1} 
				\dar[swap]{\hat{F}} 
			& C^2 \rar[][domB]{\dom} 
				\dar{\hat{F}}  
			& C \dar{F} \\
		D^2 \rar[][codA,swap]{R_2} 
			& D^2 \rar[][codB,swap]{\dom} 
			& D
		\twocellA{\theta^R} 
		\twocellB{\gamma_1}
	\end{tikzcd}
	=
	\begin{tikzcd}
		C^2 \rar[][domA]{E_1} \dar[swap]{\hat{F}}
			& C \dar{F} \\
		D^2 \rar[][codA,swap]{E_2} & D
		\twocellA{\theta}
	\end{tikzcd}
	\]

	\item Given a pair of composable 2-cells in $\FF{D}$ as in
	\[
	\begin{tikzcd}[column sep=large]
		C \rar[tick][domA]{(E_1,\eta_1,\epsilon_1)} 
				\dar[swap]{F} 
			& C \dar{F} \rar[tick][domB]{(E_2,\eta_2,\epsilon_2)} 
			& C \dar{F} \\
		D \rar[tick][codA,swap]{(E'_1,\eta'_1,\epsilon'_1)} 
			& D \rar[tick][codB,swap]{(E'_2,\eta'_2,\epsilon'_2)}
			& D
		\twocellA{\theta_1} 
		\twocellB{\theta_2}
	\end{tikzcd}
	\]
	the composite $\theta_1\otimes\theta_2$ is given by
	\[
	\begin{tikzcd}
		C^2 \rar[][domA]{R_1} 
				\dar[swap]{\hat{F}} 
			& C^2 \rar[][domB]{E_2} 
				\dar{\hat{F}} 
			& C \dar{F} \\
		D^2 \rar[][codA,swap]{R'_1}
			& D^2 \rar[][codB,swap]{E'_2}
			& D
		\twocellA{\theta_1^R}
		\twocellB{\theta_2}
	\end{tikzcd}
	\]
	while the composite $\theta_1\odot\theta_2$ is given by
	\[
	\begin{tikzcd}
		C^2 \rar[][domA]{L_1} 
				\dar[swap]{\hat{F}} 
			& C^2 \rar[][domB]{E_2} 
				\dar{\hat{F}} 
			& C \dar{F} \\
		D^2 \rar[][codA,swap]{L'_1}
			& D^2 \rar[][codB,swap]{E'_2}
			& D
		\twocellA{\theta_1^L}
		\twocellB{\theta_2}
	\end{tikzcd}
	\]

	It is a straightforward exercise to check that these definitions satisfy equations \eqref{Eq:FF2CellA} and \eqref{Eq:FF2CellB}. To illustrate, we will demonstrate that $\theta_1\otimes\theta_2$ satisfies \eqref{Eq:FF2CellA}:
	\begin{align*}
	\begin{tikzcd}[column sep=large,bend angle=50,ampersand replacement=\&]
		C^2 \rar[][domA]{E_{1\otimes2}} 
				\dar[swap]{\hat{F}} 
			\& C \dar{F} \\
		D^2 \rar[][codA,domB]{E_{1'\otimes2'}}	
				\rar[bend right][codB,swap]{\cod}
			\& D
		\twocellA{\theta_1\otimes\theta_2}
		\twocellB{\epsilon_{1'\otimes2'}}
	\end{tikzcd}
	&=
	\begin{tikzcd}[bend angle=50,ampersand replacement=\&]
		C^2 \rar[][domA]{R_1}
				\dar[swap]{\hat{F}}
			\& C^2 \rar[][domB]{E_2}
				\dar[swap]{\hat{F}}
			\& C \dar{F} \\
		D^2 \rar[][codA,swap]{R'_1}
			\& D^2 \rar[][codB,domC]{E'_2}
				\rar[bend right][codC,swap]{\cod}
			\& D
		\twocellA{\theta_1^R}
		\twocellB{\theta_2}
		\twocellC{\epsilon'_2}
	\end{tikzcd}
	\\
	&=
	\begin{tikzcd}[bend angle=50,ampersand replacement=\&]
		C^2 \rar[][domA]{R_1}
				\dar[swap]{\hat{F}}
			\& C^2 \rar[bend left][domB]{E_2}
				\rar[][codB,domC,swap]{\cod}
				\dar[swap]{\hat{F}}
			\& C \dar{F} \\
		D^2 \rar[][codA,swap]{R'_1}
			\& D^2 \rar[][codC,swap]{\cod}
			\& D
		\twocellA{\theta_1^R}
		\twocellB{\epsilon_2}
		\twocellC{\gamma_0}
	\end{tikzcd}
	\\
	&=
	\begin{tikzcd}[bend angle=50,ampersand replacement=\&]
		C^2 	\rar[bend left][domA]{E_{1\otimes2}} 
				\rar[][codA,domB,swap]{\cod} 
				\dar[swap]{\hat{F}} 
			\& C \dar{F} \\
		D^2 \rar[][codB,swap]{\cod} 
			\& D
		\twocellA{\epsilon_{1\otimes2}}
		\twocellB{\gamma_0}
	\end{tikzcd}
	\end{align*}
\end{itemize}

It is straightforward to check that $\otimes$ and $\odot$ are each associative and unital. It takes more work to provide the compatibility between $\otimes$ and $\odot$, which is the content of the proof of the next proposition.

\begin{proposition}
	$\FF{D}$ has the structure of a 2-fold double category.
\end{proposition}
\begin{proof}
The primary structure of $\FF{D}$ was given in the first part of this section. What is left is to provide the coherence data \eqref{Eq:2FoldCoherenceCellsA} and \eqref{Eq:2FoldCoherenceCellsB}.

First, note that $I_C$ is initial in the sense that, given any vertical morphism $F\colon C\to D$ and any functorial factorization $(E,\eta,\epsilon)$ on $D$, there is a unique 2-cell
\[
\begin{tikzcd}
	C \rar[tick][domA]{I_C}
			\dar[swap]{F}
		& C \dar{F} \\
	D 	\rar[tick][codA,swap]{(E,\eta,\epsilon)}
		& D
	\twocellA{}
\end{tikzcd}
\]
given by
\[
\begin{tikzcd}[bend angle=50]
	C^2 \rar[][domA]{\dom}
			\dar[swap]{\hat{F}}
		& C \dar{F} \\
	D^2 	\rar[][codA,domB]{\dom}	
			\rar[bend right][codB,swap]{E}
		& D.
	\twocellA{\gamma_1}
	\twocellB{\eta}
\end{tikzcd}
\]
Similarly, $\perp_C$ is terminal. Thus there is only one possible way to define the 2-cells $m$, $c$, and $j$, and naturality and all other coherence equations follows immediately from this uniqueness.

We still need to construct the 2-cell $z$, which will take some work. We begin by defining 2-cells
\[
\begin{tikzcd}[column sep=large]
	C 		\rar[tick][domA]{E_1\odot E_2}
			\dar[equal]
		& C \dar[equal] \\
	C 		\rar[tick][codA,swap]{E_1}
		& C
	\twocellA{p_{E_1,E_2}}
\end{tikzcd}
\qquad\text{and}\qquad
\begin{tikzcd}[column sep=large]
	C 		\rar[tick][domA]{E_1}
			\dar[equal]
		& C \dar[equal] \\
	C 		\rar[tick][codA,swap]{E_1\otimes E_2}
		& C.
	\twocellA{i_{E_1,E_2}}
\end{tikzcd}
\]
for any pair of functorial factorizations.
The 2-cell $p$ is given by the underlying 2-cell in $\mathbb{D}$
\[
\begin{tikzcd}[bend angle=30]
	C^2 \rar{L_1}
		& C^2 \rar[bend left][domA]{E_2}
			\rar[bend right][codA,swap]{\cod} 
		& C
	\twocellA{\epsilon_2}
\end{tikzcd}
\]
and $i$ is given by
\[
\begin{tikzcd}[bend angle=30]
	C^2 \rar{R_1}
		& C^2 \rar[bend left][domA]{\dom}
			\rar[bend right][codA,swap]{E_2} 
		& C.
	\twocellA{\eta_2}
\end{tikzcd}
\]
To illustrate the verification that these give well-defined 2-cells in $\FF{D}$, we will show that $i$ satisfies \eqref{Eq:FF2CellA} (keep in mind that when $F$ is an identity, $\gamma_0$ and $\gamma_1$ are also identities):
\begin{align*}
	\begin{tikzcd}[bend angle=50,ampersand replacement=\&]
		C^2 \rar{L_1}
			\& C^2 \rar[bend left][domA]{\dom}
				\rar[][codA,domB,description,inner ysep=0]{E_2}
				\rar[bend right][codB,swap]{\cod}
			\& C
		\twocellA[pos=.45]{\eta_2}
		\twocellB[pos=.55]{\epsilon_2}
	\end{tikzcd}
	&=
	\begin{tikzcd}[bend angle=30,ampersand replacement=\&]
		C^2 \rar{L_1}
			\& C^2 \rar[bend left][domA]{\dom}
				\rar[bend right][codA,swap]{\cod} 
			\& C
		\twocellA{\kappa}
	\end{tikzcd} \\
	&=
	\begin{tikzcd}[bend angle=30,ampersand replacement=\&]
		C^2 \rar[bend left][domA]{\dom}
				\rar[bend right][codA,swap]{E_1} 
			\& C.
		\twocellA{\eta_1}
	\end{tikzcd}
\end{align*}

Moreover, it is straightforward to check that $i$ and $p$ are natural families of 2-cells. Specifically, for any pair of 2-cells $\theta_1$ and $\theta_2$
\begin{align*}
\begin{tikzcd}[ampersand replacement=\&]
	C \rar[tick][domA]{E_1\odot E_2}
			\dar[equal]
		\& C \dar[equal] \\
	C \rar[tick][codA,domB]{E_1}
			\dar[swap]{F}
		\& C \dar{F} \\
	D \rar[tick][codB,swap]{E'_1}
		\& D
	\twocellA{p_{E_1,E_2}}
	\twocellB{\theta_1}
\end{tikzcd}
&=
\begin{tikzcd}[ampersand replacement=\&]
	C \rar[tick][domA]{E_1\odot E_2}
			\dar[swap]{F}
		\& C \dar{F} \\
	D \rar[tick][codA,domB]{E'_1\odot E'_2}
			\dar[equal]
		\& D \dar[equal] \\
	D \rar[tick][codB,swap]{E'_1}
		\& D
	\twocellA{\theta_1\odot\theta_2}
	\twocellB{p_{E'_1,E'_2}}
\end{tikzcd}
\\
\begin{tikzcd}[ampersand replacement=\&]
	C \rar[tick][domA]{E_1}
			\dar[equal]
		\& C \dar[equal] \\
	C \rar[tick][codA,domB]{E_1\otimes E_2}
			\dar[swap]{F}
		\& C \dar{F} \\
	D \rar[tick][codB,swap]{E'_1\otimes E'_2}
		\& D
	\twocellA{i_{E_1,E_2}}
	\twocellB{\theta_1\otimes\theta_2}
\end{tikzcd}
&=
\begin{tikzcd}[ampersand replacement=\&]
	C \rar[tick][domA]{E_1}
			\dar[swap]{F}
		\& C \dar{F} \\
	D \rar[tick][codA,domB]{E'_1}
			\dar[equal]
		\& D \dar[equal] \\
	D 		\rar[tick][codB,swap]{E'_1\otimes E'_2}
		\& D
	\twocellA{\theta_1}
	\twocellB{i_{E'_1,E'_2}}
\end{tikzcd}
\end{align*}

As with any 2-cell in $\FF{D}$, $p$ and $i$ induce 2-cells in $\mathbb{D}$
\[
\begin{tikzcd}[bend angle=30]
	C^2 \rar[bend left][domA]{R_{1\odot 2}}
			\rar[bend right][codA,swap]{R_1} 
		& C^2
	\twocellA{p^R}
\end{tikzcd}
\qquad\text{and}\qquad
\begin{tikzcd}[bend angle=30]
	C^2 \rar[bend left][domA]{L_1}
			\rar[bend right][codA,swap]{L_{1\otimes 2}} 
		& C^2.
	\twocellA{i^L}
\end{tikzcd}
\]
such that
\begin{align}
	\begin{tikzcd}[bend angle=30,ampersand replacement=\&]
		C^2 \rar[bend left][domA]{R_{1\odot 2}}
				\rar[bend right][codA,swap]{R_1} 
			\& C^2 \rar{\dom}
			\& C
		\twocellA{p^R}
	\end{tikzcd}
	&=
	\begin{tikzcd}[bend angle=30,ampersand replacement=\&]
		C^2 \rar{L_1}
			\& C^2 \rar[bend left][domA]{E_2}
				\rar[bend right][codA,swap]{\cod} 
			\& C
		\twocellA{\epsilon_2}
	\end{tikzcd} \label{Eq:DomPR}
	\\
	\begin{tikzcd}[bend angle=30,ampersand replacement=\&]
		C^2 \rar[bend left][domA]{L_1}
				\rar[bend right][codA,swap]{L_{1\otimes 2}} 
			\& C^2 \rar{\cod}
			\& C
		\twocellA{i^L}
	\end{tikzcd}
	&=
	\begin{tikzcd}[bend angle=30,ampersand replacement=\&]
		C^2 \rar{R_1}
			\& C^2 \rar[bend left][domA]{\dom}
				\rar[bend right][codA,swap]{E_2} 
			\& C
		\twocellA{\eta_2}
	\end{tikzcd} \label{Eq:CodIL}
\end{align}

Now suppose given three functorial factorizations $E_1,E_2,E_3$ on an object $C$. We define a 2-cell in $\mathbb{D}$
\[
\begin{tikzcd}[row sep=0ex, bend angle=15]
	{} & C^2 \drar[bend left][]{L_3} & \\
	|[alias=domA]| C^2 \urar[bend left][]{R_{1\otimes 2}} \drar[bend right][swap]{L_{1\otimes 3}}
		&& |[alias=codA]| C^2 \\
	{} & C^2 \urar[bend right][swap]{R_2} &
	\twocellA{w}
\end{tikzcd}
\]
such that
\begin{align}
	\begin{tikzcd}[row sep=0ex,bend angle=15,ampersand replacement=\&,baseline=(B.base)]
		\& C^2 \drar[bend left][]{L_3} \&\& \\
		|[alias=B,alias=domA]| C^2 \urar[bend left][]{R_{1\otimes 2}} 
				\drar[bend right][swap]{L_{1\otimes 3}}
			\&\& |[alias=codA]| C^2 \rar{\dom} 
			\& C \\
		\& C^2 \urar[bend right][swap]{R_2} \&\&
		\twocellA{w}
	\end{tikzcd}
	&=
	\begin{tikzcd}[bend angle=30,ampersand replacement=\&,baseline=(B.base)]
		|[alias=B]| C^2 \rar[bend left][domA]{L_1}
				\rar[bend right][codA,swap]{L_{1\otimes 3}}
			\& C^2 \rar{E_2}
			\& C
		\twocellA{i^L}
	\end{tikzcd} \label{Eq:DomW}
	\\
	\begin{tikzcd}[row sep=0ex, bend angle=15,ampersand replacement=\&,baseline=(B.base)]
		\& C^2 \drar[bend left][]{L_3} \&\& \\
		|[alias=B,alias=domA]| C^2 \urar[bend left][]{R_{1\otimes 2}} 
				\drar[bend right][swap]{L_{1\otimes 3}}
			\&\& |[alias=codA]| C^2 \rar{\cod} \& C \\
		\& C^2 \urar[bend right][swap]{R_2} \&\&
		\twocellA{w}
	\end{tikzcd}
	&=
	\begin{tikzcd}[bend angle=30,ampersand replacement=\&,baseline=(B.base)]
		|[alias=B]| C^2 \rar[bend left][domA]{R_{1\odot 2}}
				\rar[bend right][codA,swap]{R_1}
			\& C^2 \rar{E_3}
			\& C.
			\twocellA{p^R}
	\end{tikzcd} \label{Eq:CodW}
\end{align}

Using the universal property for $C^2$, it suffices to check that
\[
\begin{tikzcd}[bend angle=60]
	C^2 \rar[bend left][domA]{L_1}
			\rar[][codA,swap]{L_{1\otimes 3}}
		& C^2 \rar[][domB]{E_2}
			\rar[bend right][codB,swap]{\cod}
		& C
	\twocellA{i^L}
	\twocellB{\epsilon_2}
\end{tikzcd}
=
\begin{tikzcd}[bend angle=60]
	C^2 \rar[][domA]{R_{1\odot 2}}
			\rar[bend right][codA,swap]{R_1}
		& C^2 \rar[bend left][domB]{\dom}
			\rar[][codB,swap]{E_3}
		& C
	\twocellA{p^R}
	\twocellB{\eta_3}
\end{tikzcd}
\]
and a quick check using equations \eqref{Eq:DomPR} and \eqref{Eq:CodIL} shows that both are equal to
\[
\begin{tikzcd}[row sep=0ex,bend angle=60]
	{} & C^2 \drar[bend left][domA]{E_2}
			\drar[][codA,sloped,swap,pos=.6,inner sep=.1ex]{\cod}
		& \\
	C^2 \urar{L_1} \drar[swap]{R_1}
		&& C \\
	{} & C^2 \urar[][domB,sloped,pos=.7,inner sep=.2ex]{\dom}
			\urar[bend right][codB,swap]{E_3} &
	\twocellA{\epsilon_2}
	\twocellB{\eta_3}
\end{tikzcd}
\]
where the inner diamond is the equality $\cod L_1=\dom R_1=E_1$.

We also check that $w$ is natural with respect to 2-cells in $\FF{D}$ in the following sense: given three 2-cells $\theta_1$, $\theta_2$, and $\theta_3$, there is an equality
\[
\begin{tikzcd}[row sep=0ex, bend angle=15]
	{} & C^2 \drar[bend left][inner sep=.2ex]{L_3} 
		& \\
	|[alias=domA]| C^2 	\urar[bend left][inner sep=.2ex]{R_{1\odot 2}} 
			\drar[bend right][domB,pos=.3,inner sep=.2ex]{L_{1\otimes 3}}
			\ar{dd}[swap]{\hat{F}}
		&& |[alias=codA]| C^2 \ar{dd}{\hat{F}} \\
	{} & C^2 \urar[bend right][domC,pos=.6,inner sep=.2ex]{R_2} 
			\ar{dd}{\hat{F}}
		& \\[3ex]
	D^2 \drar[bend right][codB,swap,inner sep=.2ex]{L'_{1\otimes 3}}
		&& D^2 \\
	{} & D^2 \urar[bend right][codC,swap]{R'_2}
		&
	\twocellA{w}
	\twocellB[pos=.4,xshift=2pt]{(\theta_1\otimes\theta_3)^L}
	\twocellC[pos=.4]{\theta_2^R}
\end{tikzcd}
=
\begin{tikzcd}[row sep=0ex, bend angle=15]
	{} & C^2 \drar[bend left][domB,inner sep=.2ex]{L_3}
			\ar{dd}{\hat{F}}
		& \\
	C^2 \urar[bend left][domA,inner sep=.2ex]{R_{1\odot 2}}
			\ar{dd}[swap]{\hat{F}}
		&& C^2 \ar{dd}{\hat{F}} \\[3ex]
	{} & D^2 \drar[bend left][codB,swap,inner sep=.2ex,pos=.6]{L'_3}
		& \\
	|[alias=domC]| D^2 \urar[bend left][codA,swap,inner sep=.2ex,pos=.3]{R'_{1\odot 2}}
			\drar[bend right][swap,inner sep=.2ex]{L'_{1\otimes 3}}
		&& |[alias=codC]| D^2. \\
	{} & D^2 \urar[bend right][swap,inner sep=.2ex]{R'_2}
		&
	\twocellA[pos=.6]{(\theta_1\odot\theta_2)^R}
	\twocellB[pos=.6]{\theta_3^L}
	\twocellC{w}
\end{tikzcd}
\]
To verify this equation, it suffices to check equality upon right composition with $\gamma_0$ and $\gamma_1$. We will illustrate the $\gamma_1$ case, making use of the naturality of $i$:
\begin{multline*}
\begin{tikzcd}[row sep=0ex, bend angle=15, ampersand replacement=\&]
	{} \& C^2 \drar[bend left][inner sep=.2ex]{L_3} 
		\&\& \\
	|[alias=domA]| C^2 \urar[bend left,inner sep=.2ex][]{R_{1\odot 2}} 
			\drar[bend right][domB,pos=.3,inner sep=.2ex]{L_{1\otimes 3}}
			\ar{dd}[swap]{\hat{F}}
		\&\& |[alias=codA]| C^2 \ar{dd}{\hat{F}}
			\rar[][domD]{\dom}
		\& C \ar{dd}{F} \\
	{} \& C^2 \urar[bend right][domC,inner sep=.2ex,pos=.6]{R_2} 
			\ar{dd}{\hat{F}}
		\&\& \\[3ex]
	D^2 \drar[bend right][codB,swap,inner sep=.2ex]{L'_{1\otimes 3}}
		\&\& D^2 \rar[][codD,swap]{\dom}
		\& D \\
	{} \& D^2 \urar[bend right][codC,swap,inner sep=.2ex]{R'_2}
		\&\&
	\twocellA{w}
	\twocellB[pos=.4,xshift=2pt]{(\theta_1\otimes\theta_3)^L}
	\twocellC[pos=.4]{\theta_2^R}
	\twocellD{\gamma_1}
\end{tikzcd}
=
\begin{tikzcd}[bend angle=50, row sep=6ex, ampersand replacement=\&]
	C^2 \rar[bend left][domA]{L_1}
			\rar[][codA,domB,swap]{L_{1\otimes 3}}
			\dar[swap]{\hat{F}}
		\& C^2 \rar[][domC]{E_2}
			\dar{\hat{F}}
		\& C \dar{F} \\
	D^2 \rar[][codB,swap]{L_{1'\otimes 3'}}
		\& D^2 \rar[][codC,swap]{E'_2}
		\& D
	\twocellA{i^L}
	\twocellB{(\theta_1\otimes\theta_3)^L}
	\twocellC{\theta_2}
\end{tikzcd}
\\
=
\begin{tikzcd}[bend angle=50, row sep=6ex, ampersand replacement=\&]
	C^2 \rar[][domA]{L_1}
			\dar[swap]{\hat{F}}
		\& C^2 \rar[][domB]{E_2}
			\dar{\hat{F}}
		\& C \dar{F} \\
	D^2 \rar[][codA,domC]{L'_1}
			\rar[bend right][codC,swap]{L'_{1\otimes 3}}
		\& D^2 \rar[][codB,swap]{E'_2}
		\& D
	\twocellA{\theta_1^L}
	\twocellB{\theta_2}
	\twocellC{i^L}
\end{tikzcd}
=
\begin{tikzcd}[row sep=0ex, bend angle=15, ampersand replacement=\&]
	{} \& C^2 \drar[bend left][domB,inner sep=.2ex]{L_3}
			\ar{dd}{\hat{F}}
		\&\& \\
	C^2 \urar[bend left][domA,inner sep=.2ex]{R_{1\odot 2}}
			\ar{dd}[swap]{\hat{F}}
		\&\& C^2 \ar{dd}{\hat{F}}
			\rar[][domD]{\dom}
		\& C \ar{dd}{F} \\[3ex]
	{} \& D^2 \drar[bend left][codB,swap,inner sep=.2ex,pos=.6]{L'_3}
		\&\& \\
	D^2 	\urar[bend left][swap]{R'_{1\odot 2}}
			\drar[bend right][swap]{L'_{1\otimes 3}}
			\twocell{rr}{w}
		\&
		\& D^2 \rar[swap]{\dom}
		\& D. \\
	{}
		\& D^2 \urar[bend right][swap]{R'_2}
		\&\&
	\twocellA{(\theta_1\odot\theta_2)^R}
	\twocellB{\theta_3^L}
	\twocellD{\gamma_1}
\end{tikzcd}
\end{multline*}

Finally, given four functorial factorizations $E_1, E_2, E_3, E_4$ on an object $C$, we define the 2-cell 
\[
\begin{tikzcd}[column sep=14ex]
	C \rar[tick]{(1\odot 2)\otimes(3\odot 4)} 
			\dar[equal] 
			\twocell{dr}{z_{1,2,3,4}}
		& C \dar[equal] \\
	C \rar[tick][swap]{(1\otimes 3)\odot(2\otimes 4)}
		& C
\end{tikzcd}
\]
in $\FF{D}$, where $(1\odot 2)$ is shorthand for $(E_1,\eta_1,\epsilon_1)\odot(E_2,\eta_2,\epsilon_2)$, to have the underlying 2-cell in $\mathbb{D}$
\[
\begin{tikzcd}[row sep=0ex, bend angle=15,ampersand replacement=\&]
	\& C^2 \drar[bend left][]{L_3} \&\& \\
	C^2 \urar[bend left][]{R_{1\otimes 2}} \drar[bend right][swap]{L_{1\otimes 3}}
			\twocell{rr}{w}
		\&\& C^2 \rar{E_4} \& C. \\
	\& C^2 \urar[bend right][swap]{R_2} \&\&
\end{tikzcd}
\]
The naturality of $z$ follows immediately from that of $w$, but we still need to check that this satisfies equations \eqref{Eq:FF2CellA} and \eqref{Eq:FF2CellB}. We will leave the details to the reader, but note that \eqref{Eq:FF2CellB} comes down to the verification of the equality
\[
\begin{tikzcd}[row sep=0ex, bend angle=15, ampersand replacement=\&, baseline=(B.base)]
	\& C^2 \drar[bend left]{L_3} \&\& \\
	C^2 	\urar[bend left=70]{\id}
			\twocell[bend left=40]{ur}{\vec{\eta}_{1\odot 2}}
			\urar[bend left][swap]{R_{1\otimes 2}}
			\twocell{rr}{w}
			\drar[bend right][swap]{L_{1\otimes 3}}
		\&\& C^2 \rar[bend left=50]{\dom}
			\twocell[bend left=25]{r}{\eta_4}
			\rar[swap]{E_4}
		\& |[alias=B]| C \\
	\& C^2 \urar[bend right][swap]{R_2} \&\&
\end{tikzcd}
=
\begin{tikzcd}[bend angle=30, baseline=(B.base)]
	|[alias=B]| C^2 \rar{L_{1\otimes 3}}
		& C^2 \rar[bend left]{\id}
			\twocell{r}{\vec{\eta}_2}
			\rar[bend right][swap]{R_2}
		& C^2 \rar[bend left]{\dom}
			\twocell{r}{\eta_4}
			\rar[bend right][swap]{E_4}
		& C,
\end{tikzcd}
\]
which follows from equation \eqref{Eq:DomW} and the fact that $\dom\circ i^L = \id_{\dom}$.
\end{proof}

Up to this point, we have demonstrated that given any double category $\mathbb{D}$ having arrow objects, there is a 2-fold double category $\FF{D}$ of functorial factorizations in $\mathbb{D}$. The last thing we want to say about this construction is that a cyclic action on $\mathbb{D}$ lifts to one on $\FF{D}$, and hence also to one on $\mathrm{Bimon}(\mathbb{\FF{D}})$.

The cyclic action on objects and vertical morphisms is given directly by that on $\mathbb{D}$. Given a horizontal 1-cell $(E,\eta,\epsilon)$ on an object $C$, we define the 1-cell $(E,\eta,\epsilon)^{\bullet}$ on $C^{\bullet}$ to be $(E^{\bullet},\epsilon^{\bullet},\eta^{\bullet})$. This also implies that the cyclic action swaps $L$ and $R$ for any given functorial factorization.

A quick look at the definitions of the two horizontal compositions is now enough to see that for any two functorial factorizations $E_1$ and $E_2$, we have
\[
	(E_1\otimes E_2)^{\bullet} = E_1^{\bullet}\odot E_2^{\bullet}
	\qquad\text{and}\qquad
	(E_1\odot E_2)^{\bullet} = E_1^{\bullet}\otimes E_2^{\bullet}
\]

Similarly, the cyclic action on 2-cells in $\FF{D}$ is given by the cyclic action in $\mathbb{D}$ on the underlying 2-cell. This gives a valid 2-cell in $\FF{D}$ since the cyclic action simply swaps the equations \eqref{Eq:FF2CellA} and \eqref{Eq:FF2CellB}. 

\chapter{Algebraic Weak Factorization Systems}

For this section, let $\mathcal{D}$ be the 2-category of (small) categories $\mathcal{C}\mathrm{at}$, and let $\mathbb{D}$ be the canonical double category of squares in $\mathcal{D}$, whose horizontal and vertical morphisms are simply the 1-cells of $\mathcal{D}$, and whose 2-cells are the 2-cells of $\mathcal{D}$ of the appropriate shape.

In this section we will show that bimonoids in $\FF{D}$ are precisely algebraic weak factorization systems, and more generally that the morphisms in $\mathrm{Bimon}(\mathbb{D})$ are given by (co)lax morphisms of algebraic weak factorization systems.

Suppose that $E=(E,\eta,\epsilon)$ is a functorial factorization on a category $\cat{C}$, and consider a monoid structure on $E$. As $I_C$ is initial, the unit of the monoid is forced, and is simply $\eta$. The multiplication is given by a natural transformation $\mu\colon ER\Rightarrow E$ satisfying equations \eqref{Eq:FF2CellA} and \eqref{Eq:FF2CellB}, which now take the form $\epsilon\circ\mu = \epsilon R$ and $\mu\circ(\eta\cdot\vec{\eta})=\eta$.

The unit axioms for the monoid give the equations $\mu\circ E\vec{\eta}=\id_E=\mu\circ\eta R$, which together imply the equation $\mu\circ(\eta\cdot\vec{\eta})=\eta$ above. And finally, writing $\vec{\mu}=\mu^R\colon R^2\to R$ for the natural transformation induced by the 2-cell $\mu$, the associativity axiom gives the equation $\mu\circ E\vec{\mu}=\mu\circ\mu R$.

\begin{proposition}
	A monoid structure on an object $(E,\eta,\epsilon)$ in $\FF{\mathbb{D}}$ is given by a natural transformation $\mu\colon ER\Rightarrow E$, satisfying equations
	\begin{equation}
		\epsilon\circ\mu=\epsilon R \qquad 
			\mu\circ E\vec{\eta}=\id_E=\mu\circ\eta R \qquad 
			\mu\circ E\vec{\mu}=\mu\circ\mu R.
	\end{equation}
	This determines a monad $\mathbb{R}=(R,\vec{\eta},\vec{\mu})$, such that $\dom\vec{\mu}=\mu$ and $\cod\vec{\mu}=\id_{\cod}$.

	Similarly, a comonoid structure on $(E,\eta,\epsilon)$ is given by a natural transformation $\delta\colon L\Rightarrow EL$, satisfying equations
	\begin{equation}
		\delta\circ\eta=\eta L \qquad 
			E\vec{\epsilon}\circ\delta=\id_E=\epsilon L\circ\delta \qquad
			E\vec{\delta}\circ\delta=\delta L\circ\delta,
	\end{equation}
	which determines a comonad $\mathbb{L}=(L,\vec{\epsilon},\vec{\delta})$, such that $\dom\vec{\delta}=\id_{\dom}$ and $\cod\vec{\delta}=\delta$.
\end{proposition}

Hence a functorial factorization which simultaineously has a monoid structure and a comonoid structure in $\FF{\mathbb{D}}$ is precisely a weak factorization system, missing only the second bullet of Definition~\ref{Def:Awfs}, the distributive law condition. This is not surprising, as it is the only condition requiring a compatability between the monad and comonad structures. We will see that a bialgebra in $\FF{\mathbb{D}}$ adds precisely this compatibility.

\begin{proposition}
	A bimonoid structure on a horizontal morphism $(E,\eta,\epsilon)\colon C\to C$ in $\FF{\mathbb{D}}$ is precisely a weak factorization system on $C$ with underlying functorial factorization system $(E,\eta,\epsilon)$. 
\end{proposition}
\begin{proof}
	We have already shown how the monoid an comonoid structures give rise to the monod and comonad of the awfs. All that remains is to show that the equations \eqref{Eq:Bimonoid} amount to just the distributive law, i.e. the equation
	\begin{equation}\label{Eq:AwfsDistributiveLaw}
	\begin{tikzcd}[row sep=tiny, baseline=(B.base)]
		& C^2 \drar{L} \ar[bend left=30]{drr}{E} 
			\twocell[bend left=10]{drr}{\delta} &&\\
		|[alias=B]| C^2 \urar{R} \drar[swap]{L} \twocell{rr}{\Delta}
			&& C^2 \rar{E} & C \\
		& C^2 \urar[swap]{R} \ar[bend right=30]{urr}[swap]{E} 
			\twocell[bend right=10]{urr}{\mu} &&
	\end{tikzcd}
	=
	\begin{tikzcd}[row sep=2ex, column sep=small, bend angle=25, baseline=(B.base)]
		& C^2 \drar[bend left]{E} & \\
		|[alias=B]| C^2 \urar[bend left]{R}
			\twocell[bend left=30]{rr}{\mu}
			\ar{rr}[description]{E}
			\twocell[bend right=30]{rr}{\delta}
			\drar[bend right][swap]{L}
		&& C. \\
		& C^2 \urar[bend right][swap]{E} &
	\end{tikzcd}
	\end{equation}

	First of all, notice that the first three equations of \eqref{Eq:Bimonoid} follow trivially from the initiality of $I_C$ and the terminality of $\perp_C$ in $\FF{\mathbb{D}}$, hence they do not impose any further conditions.

	The fourth equation here takes the form
	\[
	\begin{tikzcd}
		C^2 \rar{R} \twocell{dr}{\delta^R} \dar[equal]
			& C^2 \ar{rr}{E} \twocell{drr}{\delta} \dar[equal]
			&& C \dar[equal] \\
		C^2 \rar[swap]{R_{E\odot E}} \twocell{drr}{w} \dar[equal]
			& C^2 \rar{L}
			& C^2 \rar{E} \twocell{dr}{\id_E} \dar[equal]
			& C \dar[equal] \\
		C^2 \rar{L_{E\otimes E}} \twocell{dr}{\mu^L} \dar[equal]
			& C^2 \rar[swap]{R} \twocell{drr}{\mu} \dar[equal]
			& C^2 \rar[swap]{E}
			& C \dar[equal] \\
		C^2 \rar[swap]{L} & C^2 \ar{rr}[swap]{E} && C
	\end{tikzcd}
	=
	\begin{tikzcd}
		C^2 \rar{R} \twocell{drr}{\mu} \dar[equal]
			& C^2 \rar{E}
			& C \dar[equal] \\
		C^2 \ar{rr}{E} \twocell{drr}{\delta} \dar[equal]
			&& C \dar[equal] \\
		C^2 \rar[swap]{L} & C^2 \rar[swap]{E} & C,
	\end{tikzcd}
	\]
	and so to prove \eqref{Eq:AwfsDistributiveLaw}, it suffices to show that
	\[
	\begin{tikzcd}[row sep=tiny, baseline=(B.base)]
		{} & C^2 \drar[bend left=20]{L} & \\
		|[alias=B]| C^2 \urar[bend left=60]{R}
				\twocell[bend left=30]{ur}{\delta^R}
				\urar[sloped,swap,pos=0.2]{R_{E\odot E}}
				\twocell{rr}{w}
				\drar[sloped,pos=0.2]{L_{E\otimes E}}
				\twocell[bend right=30]{dr}{\mu^L}
				\drar[bend right=60][swap]{L}
			&& C^2 \\
		{} & C^2 \urar[bend right=20][swap]{R} &
	\end{tikzcd}
	=
	\begin{tikzcd}[row sep=0ex, column sep=4ex, bend angle=15, baseline=(B.base)]
		{} & C^2 \drar[bend left]{L} & \\
		|[alias=B]| C^2 \urar[bend left]{R}
				\twocell{rr}{\Delta}
				\drar[bend right][swap]{L}
			&& C^2. \\
		{} & C^2 \urar[bend right][swap]{R} &
	\end{tikzcd}
	\]
	We can check this using the universal property of $C^2$ by composing with $\dom$ and $\cod$. First, use \eqref{Eq:DomW} and \eqref{Eq:CodW} to check that
	\begin{gather*}
	\begin{tikzcd}[row sep=tiny, baseline=(B.base),ampersand replacement=\&]
		{} \& C^2 \drar[bend left=20]{L} \&\& \\
		|[alias=B]| C^2 \urar[bend left=60]{R}
				\twocell[bend left=30]{ur}{\delta^R}
				\urar[sloped,swap,pos=0.2]{R_{E\odot E}}
				\twocell{rr}{w}
				\drar[sloped,pos=0.2]{L_{E\otimes E}}
				\twocell[bend right=30]{dr}{\mu^L}
				\drar[bend right=60][swap]{L}
			\&\& C^2 \rar{\dom} \& C \\
		{} \& C^2 \urar[bend right=20][swap]{R} \&\&
	\end{tikzcd}
	=
	\begin{tikzcd}[baseline=(B.base),ampersand replacement=\&]
		C^2 \ar[bend left=55]{rr}{E}
				\twocell[bend left=30]{rr}{\delta}
				\rar[bend left=30]{L}
				\twocell{r}{i^L}
				\rar[bend right=30][swap,inner sep=0.5pt]{L_{E\otimes E}}
				\twocell[bend right=55,looseness=2]{r}{\mu^L}
				\rar[bend right=85,looseness=2][swap]{L}
			\& C^2 \rar[swap]{E} \& C
	\end{tikzcd}
	\\
	\begin{tikzcd}[row sep=tiny, baseline=(B.base),ampersand replacement=\&]
		{} \& C^2 \drar[bend left=20]{L} \&\& \\
		|[alias=B]| C^2 \urar[bend left=60]{R}
				\twocell[bend left=30]{ur}{\delta^R}
				\urar[sloped,swap,pos=0.2]{R_{E\odot E}}
				\twocell{rr}{w}
				\drar[sloped,pos=0.2]{L_{E\otimes E}}
				\twocell[bend right=30]{dr}{\mu^L}
				\drar[bend right=60][swap]{L}
			\&\& C^2 \rar{\cod} \& C \\
		{} \& C^2 \urar[bend right=20][swap]{R} \&\&
	\end{tikzcd}
	=
	\begin{tikzcd}[baseline=(B.base),ampersand replacement=\&]
		C^2 \rar[bend left=85,looseness=2]{R}
				\twocell[bend left=55,looseness=2]{r}{\delta^R}
				\rar[bend left=30][inner sep=0.5pt]{R_{E\odot E}}
				\twocell{r}{p^R}
				\rar[bend right=30][swap]{R}
				\twocell[bend right=30]{rr}{\mu}
				\ar[bend right=55]{rr}[swap]{E}
			\& C^2 \rar{E} \& C.
	\end{tikzcd}
	\end{gather*}
	Then use the definitions of $i$ and $p$ to check that $\mu\circ i=\mu\circ\eta R=\id_E$ and $p\circ\delta=\epsilon L\circ\delta=\id_E$, so that the first row above just equals $\delta$, and the second row equals $\mu$. Since $\Delta$ also (by definition) satisfies $\dom\Delta=\delta$ and $\cod\Delta=\mu$, we are done.
\end{proof}

\bibliographystyle{alpha}
\bibliography{all}

\end{document}


\[
\begin{tikzpicture}[commutative diagrams/every diagram]
	\node (IX) at (0,0) {$F_0(I,X)$};
	\node (JX) at (0,2) {$F_0(J,X)$};
	\node (IY) at (-2,0) {$F_0(I,Y)$};
	\node (PJ) at (-1.7,1.7) {$\cdot$};
	\node (JY) at (-3.6,3) {$F_0(J,Y)$};
	\node (DG) at (-3.6,4.8) {$\cdot$};
	\node (CG) at (-2.4,3.5) {$\cdot$};

	\path[commutative diagrams/.cd, every arrow, every label] 
		(IY) edge (IX)
		(JX) edge (IX)
		(PJ) edge (JX) edge (IY)
		(JY) edge (IY) edge (JX) edge (PJ)
		(DG) edge (JY) edge (CG)
		(CG) edge (IY) edge (JX) edge (PJ);
\end{tikzpicture}
\]