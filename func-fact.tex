% -*- root: thesis.tex -*-

\chapter{Functorial factorizations}\label{Ch:FuncFact}

Let $\mathbb{D}$ be a cyclic double category, and assume it has arrow objects in the sense of \cref{Sec:ArrowObjects}. In this section, we will define a 2-fold double category $\FFD$ of functorial factorizations in $\mathbb{D}$, as follows:
\begin{itemize}
	\item The objects and vertical 1-cells are the same as in $\mathbb{D}$.

	\item Horizontal 1-cells $\begin{tikzcd}[baseline,column sep=2.5ex] C \rar[tick]& C \end{tikzcd}$ in $\FFD$ are tuples $(E,\eta,\epsilon)$, where $E\colon C^2\to C$ is a horizontal 1-cell in $\mathbb{D}$, and
	\[
	\begin{tikzcd}[bend angle=30]
		C^2 \rar[bend left][domA]{\dom}
			\rar[bend right][codA,swap]{E}
		& C
		\twocellA{\eta}
	\end{tikzcd}
	\qquad
	\begin{tikzcd}[bend angle=30]
		C^2 \rar[bend left][domA]{E}
			\rar[bend right][codA,swap]{\cod}
		& C
		\twocellA{\epsilon}
	\end{tikzcd}
	\]
	are 2-cells in $\mathbb{D}$ such that 
	\[
	\begin{tikzcd}[bend angle=50]
		C^2 \rar[bend left][domA]{\dom}
			\rar[][codA,domB,description]{E}
			\rar[bend right][codB,swap]{\cod}
		& C
		\twocellA[pos=.45]{\eta}
		\twocellB[pos=.55]{\epsilon}
	\end{tikzcd}
	=
	\begin{tikzcd}[bend angle=30]
		C^2 \rar[bend left][domA]{\dom}
			\rar[bend right][codA,swap]{\cod}
		& C.
		\twocellA{\kappa}
	\end{tikzcd}
	\]

	By the universal property of $C^2$, this also determines horizontal 1-cells $L,R\colon C^2\to C^2$ such that $\dom\circ L=\dom$, $\cod\circ R=\cod$, $\cod\circ L=\dom\circ R=E$, $\kappa\circ L=\eta$, and $\kappa\circ R=\epsilon$, and 2-cells
	\[
	\begin{tikzcd}[bend angle=30]
		C^2 \rar[bend left][domA]{L}
			\rar[bend right][codA,swap]{\id}
		& C^2.
		\twocellA{\vec{\epsilon}}
	\end{tikzcd}
	\qquad
	\begin{tikzcd}[bend angle=30]
		C^2 \rar[bend left][domA]{\id}
			\rar[bend right][codA,swap]{R}
		& C^2.
		\twocellA{\vec{\eta}}
	\end{tikzcd}
	\]
	such that $\dom\circ\vec{\epsilon}=\id_{\dom}$, $\cod\circ\vec{\epsilon}=\epsilon$, $\dom\circ\vec{\eta}=\eta$, and $\cod\circ\vec{\eta}=\id_{\cod}$.

	\item The horizontal composition $(E_1,\eta_1,\epsilon_1)\otimes(E_2,\eta_2,\epsilon_2)$ of two horizontal 1-cells
	\[
	\begin{tikzcd}[column sep=large]
		C \rar[tick][]{(E_1,\eta_1,\epsilon_1)} & C \rar[tick][]{(E_2,\eta_2,\epsilon_2)} & C
	\end{tikzcd}
	\]
	in $\FFD$ is a horizontal 1-cell $(E_{1\otimes2},\eta_{1\otimes2},\epsilon_{1\otimes2})$, where
	\[
	E_{1\otimes2} =
	\begin{tikzcd}
		C^2 \rar{R_1} & C^2 \rar{E_2} & C
	\end{tikzcd}
	\]
	\[
	\eta_{1\otimes2} =
	\begin{tikzcd}[bend angle=30,baseline=(B.base)]
		|[alias=B]| C^2 \rar[bend left][domA]{\id}
				\rar[bend right][codA,swap]{R_1} 
			& C^2 \rar[bend left][domB]{\dom}
				\rar[bend right][codB,swap]{E_2} 
			& C
		\twocellA{\vec{\eta_1}}
		\twocellB{\eta_2}
	\end{tikzcd}
	\]
	\[
	\epsilon_{1\otimes2} =
	\begin{tikzcd}[bend angle=30,baseline=(B.base)]
		|[alias=B]| C^2 \rar{R_1}
			& C^2 \rar[bend left][domA]{E_2}
				\rar[bend right][codA,swap]{\cod} 
			& C
		\twocellA{\epsilon_2}
	\end{tikzcd}
	\]
	which also determines that $R_{1\otimes2}=R_2\circ R_1$.

	\item The horizontal unit $I_C$ for $\otimes$ is $(\dom,\id,\kappa)$.

	\item The second horizontal composition $(E_1,\eta_1,\epsilon_1)\odot(E_2,\eta_2,\epsilon_2)$ is a horizontal 1-cell $(E_{1\odot2},\eta_{1\odot2},\epsilon_{1\odot2})$, where
	\[
	E_{1\odot2} =
	\begin{tikzcd}
		C^2 \rar{L_1} & C^2 \rar{E_2} & C
	\end{tikzcd}
	\]
	\[
	\eta_{1\odot2} =
	\begin{tikzcd}[bend angle=30,baseline=(B.base)]
		|[alias=B]| C^2 \rar{L_1}
			& C^2 \rar[bend left][domA]{\dom}
				\rar[bend right][codA,swap]{E_2} 
			& C
		\twocellA{\eta_2}
	\end{tikzcd}
	\]
	\[
	\epsilon_{1\odot2} =
	\begin{tikzcd}[bend angle=30,baseline=(B.base)]
		|[alias=B]| C^2 \rar[bend left][domA]{L_1}
				\rar[bend right][codA,swap]{\id} 
			& C^2 \rar[bend left][domB]{E_2}
				\rar[bend right][codB,swap]{\cod} 
			& C
		\twocellA{\vec{\epsilon_1}}
		\twocellB{\epsilon_2}
	\end{tikzcd}
	\]
	which also determines that $L_{1\odot2}=L_2\circ L_1$.

	\item The horizontal unit $\perp_C$ for $\odot$ is $(\cod,\kappa,\id)$.

	\item 2-cells
	\[
	\begin{tikzcd}[column sep=large]
		C \rar[tick][domA]{(E_1,\eta_1,\epsilon_1)} \dar[swap]{F}  & C \dar{F} \\
		D \rar[tick][codA,swap]{(E_2,\eta_2,\epsilon_2)} & D
		\twocellA{\theta}
	\end{tikzcd}
	\]
	in $\FFD$ are given by 2-cells
	\[
	\begin{tikzcd}
		C^2 \rar[][domA]{E_1} \dar[swap]{\hat{F}}  
			& C \dar{F} \\
		D^2 \rar[][codA,swap]{E_2} 
			& D
		\twocellA{\theta}
	\end{tikzcd}
	\]
	in $\mathbb{D}$ such that
	\begin{gather}
		\begin{tikzcd}[bend angle=50,ampersand replacement=\&]
			C^2 \rar[][domA]{E_1} 
				\dar[swap]{\hat{F}} 
			\& C \dar{F} \\
			D^2 \rar[][codA,domB]{E_2}	
				\rar[bend right][codB,swap]{\cod}
			\& D
			\twocellA{\theta}
			\twocellB{\epsilon_2}
		\end{tikzcd}
		=
		\begin{tikzcd}[bend angle=50,ampersand replacement=\&]
			C^2 \rar[bend left][domA]{E_1} 
				\rar[][codA,domB,swap]{\cod} 
				\dar[swap]{\hat{F}} 
			\& C \dar{F} \\
			D^2 \rar[][codB,swap]{\cod} \& D
			\twocellA{\epsilon_1}
			\twocellB{\gamma_0}
		\end{tikzcd}\label{Eq:FF2CellA}
		\\ \shortintertext{and}
		\begin{tikzcd}[bend angle=50,ampersand replacement=\&]
			C^2 \rar[][domA]{\dom} 
				\dar[swap]{\hat{F}} 
			\& C \dar{F} \\
			D^2 \rar[][codA,domB]{\dom}	
				\rar[bend right][codB,swap]{E_2}
			\& D
			\twocellA{\gamma_1}
			\twocellB{\eta_2}
		\end{tikzcd}\label{Eq:FF2CellB}
		=
		\begin{tikzcd}[bend angle=50,ampersand replacement=\&]
			C^2 \rar[bend left][domA]{\dom} 
				\rar[][codA,domB,swap]{E_1} 
				\dar[swap]{\hat{F}} 
			\& C \dar{F} \\
			D^2 \rar[][codB,swap]{E_2} \& D
			\twocellA{\eta_1}
			\twocellB{\theta}
		\end{tikzcd}
	\end{gather}

	This also determines unique 2-cells
	\[
	\begin{tikzcd}
		C^2 \rar[][domA]{R_1} \dar[swap]{\hat{F}} & C^2 \dar{\hat{F}} \\
		D^2 \rar[][codA,swap]{R_2} & D^2
		\twocellA{\theta^R}
	\end{tikzcd}
	\quad\text{and}\quad
	\begin{tikzcd}
		C^2 \rar[][domA]{L_1} \dar[swap]{\hat{F}} & C^2 \dar{\hat{F}} \\
		D^2 \rar[][codA,swap]{L_2} & D^2
		\twocellA{\theta^L}
	\end{tikzcd}
	\]
	such that composing horizontally with $\gamma_0$ or $\gamma_1$ gives $\gamma_0$, $\gamma_1$, or $\theta$ as appropriate. For instance:
	\[
	\begin{tikzcd}
		C^2 \rar[][domA]{R_1} 
				\dar[swap]{\hat{F}} 
			& C^2 \rar[][domB]{\dom} 
				\dar{\hat{F}}  
			& C \dar{F} \\
		D^2 \rar[][codA,swap]{R_2} 
			& D^2 \rar[][codB,swap]{\dom} 
			& D
		\twocellA{\theta^R} 
		\twocellB{\gamma_1}
	\end{tikzcd}
	=
	\begin{tikzcd}
		C^2 \rar[][domA]{E_1} \dar[swap]{\hat{F}}
			& C \dar{F} \\
		D^2 \rar[][codA,swap]{E_2} & D
		\twocellA{\theta}
	\end{tikzcd}
	\]

	\item Given a pair of composable 2-cells in $\FFD$ as in
	\[
	\begin{tikzcd}[column sep=large]
		C \rar[tick][domA]{(E_1,\eta_1,\epsilon_1)} 
				\dar[swap]{F} 
			& C \dar{F} \rar[tick][domB]{(E_2,\eta_2,\epsilon_2)} 
			& C \dar{F} \\
		D \rar[tick][codA,swap]{(E'_1,\eta'_1,\epsilon'_1)} 
			& D \rar[tick][codB,swap]{(E'_2,\eta'_2,\epsilon'_2)}
			& D
		\twocellA{\theta_1} 
		\twocellB{\theta_2}
	\end{tikzcd}
	\]
	the composite $\theta_1\otimes\theta_2$ is given by
	\[
	\begin{tikzcd}
		C^2 \rar[][domA]{R_1} 
				\dar[swap]{\hat{F}} 
			& C^2 \rar[][domB]{E_2} 
				\dar{\hat{F}} 
			& C \dar{F} \\
		D^2 \rar[][codA,swap]{R'_1}
			& D^2 \rar[][codB,swap]{E'_2}
			& D
		\twocellA{\theta_1^R}
		\twocellB{\theta_2}
	\end{tikzcd}
	\]
	while the composite $\theta_1\odot\theta_2$ is given by
	\[
	\begin{tikzcd}
		C^2 \rar[][domA]{L_1} 
				\dar[swap]{\hat{F}} 
			& C^2 \rar[][domB]{E_2} 
				\dar{\hat{F}} 
			& C \dar{F} \\
		D^2 \rar[][codA,swap]{L'_1}
			& D^2 \rar[][codB,swap]{E'_2}
			& D
		\twocellA{\theta_1^L}
		\twocellB{\theta_2}
	\end{tikzcd}
	\]

	It is a straightforward exercise to check that these definitions satisfy equations~\eqref{Eq:FF2CellA} and~\eqref{Eq:FF2CellB}. To illustrate, we will demonstrate that $\theta_1\otimes\theta_2$ satisfies~\eqref{Eq:FF2CellA}:
	\begin{align*}
	\begin{tikzcd}[column sep=large,bend angle=50,ampersand replacement=\&]
		C^2 \rar[][domA]{E_{1\otimes2}} 
				\dar[swap]{\hat{F}} 
			\& C \dar{F} \\
		D^2 \rar[][codA,domB]{E_{1'\otimes2'}}	
				\rar[bend right][codB,swap]{\cod}
			\& D
		\twocellA{\theta_1\otimes\theta_2}
		\twocellB{\epsilon_{1'\otimes2'}}
	\end{tikzcd}
	&=
	\begin{tikzcd}[bend angle=50,ampersand replacement=\&]
		C^2 \rar[][domA]{R_1}
				\dar[swap]{\hat{F}}
			\& C^2 \rar[][domB]{E_2}
				\dar[swap]{\hat{F}}
			\& C \dar{F} \\
		D^2 \rar[][codA,swap]{R'_1}
			\& D^2 \rar[][codB,domC]{E'_2}
				\rar[bend right][codC,swap]{\cod}
			\& D
		\twocellA{\theta_1^R}
		\twocellB{\theta_2}
		\twocellC{\epsilon'_2}
	\end{tikzcd}
	\\
	&=
	\begin{tikzcd}[bend angle=50,ampersand replacement=\&]
		C^2 \rar[][domA]{R_1}
				\dar[swap]{\hat{F}}
			\& C^2 \rar[bend left][domB]{E_2}
				\rar[][codB,domC,swap]{\cod}
				\dar[swap]{\hat{F}}
			\& C \dar{F} \\
		D^2 \rar[][codA,swap]{R'_1}
			\& D^2 \rar[][codC,swap]{\cod}
			\& D
		\twocellA{\theta_1^R}
		\twocellB{\epsilon_2}
		\twocellC{\gamma_0}
	\end{tikzcd}
	\\
	&=
	\begin{tikzcd}[bend angle=50,ampersand replacement=\&]
		C^2 	\rar[bend left][domA]{E_{1\otimes2}} 
				\rar[][codA,domB,swap]{\cod} 
				\dar[swap]{\hat{F}} 
			\& C \dar{F} \\
		D^2 \rar[][codB,swap]{\cod} 
			\& D
		\twocellA{\epsilon_{1\otimes2}}
		\twocellB{\gamma_0}
	\end{tikzcd}
	\end{align*}
\end{itemize}

\begin{example}
	Functorial factorizations in the double category $\mathbb{D}=\mathbb{S}\mathrm{q}(\mathcal{C}\mathrm{at})$ of squares in the 2-category of categories are precisely functorial factorizations as defined in \cref{Sec:FuncFact}.
\end{example}

It is straightforward to check that $\otimes$ and $\odot$ are each associative and unital. It takes more work to provide the compatibility between $\otimes$ and $\odot$, which is the content of the proof of the next proposition.

\begin{proposition}\label{Prop:FF2Fold}
	$\FFD$ has the structure of a 2-fold double category.
\end{proposition}
\begin{proof}
The primary structure of $\FFD$ was given in the first part of this section. What is left is to provide the coherence data~\eqref{Eq:2FoldCoherenceCellsA} and~\eqref{Eq:2FoldCoherenceCellsB}.

First, note that $I_C$ is initial in the sense that, given any vertical morphism $F\colon C\to D$ and any functorial factorization $(E,\eta,\epsilon)$ on $D$, there is a unique 2-cell
\[
\begin{tikzcd}
	C \rar[tick][domA]{I_C}
			\dar[swap]{F}
		& C \dar{F} \\
	D 	\rar[tick][codA,swap]{(E,\eta,\epsilon)}
		& D
	\twocellA{}
\end{tikzcd}
\]
given by
\[
\begin{tikzcd}[bend angle=50]
	C^2 \rar[][domA]{\dom}
			\dar[swap]{\hat{F}}
		& C \dar{F} \\
	D^2 	\rar[][codA,domB]{\dom}	
			\rar[bend right][codB,swap]{E}
		& D.
	\twocellA{\gamma_1}
	\twocellB{\eta}
\end{tikzcd}
\]
Similarly, $\perp_C$ is terminal. Thus there is only one possible way to define the 2-cells $m$, $c$, and $j$, and naturality and all other coherence equations follows immediately from this uniqueness.

We still need to construct the 2-cell $z$, which will take some work. We begin by defining 2-cells
\[
\begin{tikzcd}[column sep=large]
	C 		\rar[tick][domA]{E_1\odot E_2}
			\dar[equal]
		& C \dar[equal] \\
	C 		\rar[tick][codA,swap]{E_1}
		& C
	\twocellA{p_{E_1,E_2}}
\end{tikzcd}
\qquad\text{and}\qquad
\begin{tikzcd}[column sep=large]
	C 		\rar[tick][domA]{E_1}
			\dar[equal]
		& C \dar[equal] \\
	C 		\rar[tick][codA,swap]{E_1\otimes E_2}
		& C.
	\twocellA{i_{E_1,E_2}}
\end{tikzcd}
\]
for any pair of functorial factorizations.
The 2-cell $p$ is given by the underlying 2-cell in $\mathbb{D}$
\[
\begin{tikzcd}[bend angle=30]
	C^2 \rar{L_1}
		& C^2 \rar[bend left][domA]{E_2}
			\rar[bend right][codA,swap]{\cod} 
		& C
	\twocellA{\epsilon_2}
\end{tikzcd}
\]
and $i$ is given by
\[
\begin{tikzcd}[bend angle=30]
	C^2 \rar{R_1}
		& C^2 \rar[bend left][domA]{\dom}
			\rar[bend right][codA,swap]{E_2} 
		& C.
	\twocellA{\eta_2}
\end{tikzcd}
\]
To illustrate the verification that these give well-defined 2-cells in $\FFD$, we will show that $i$ satisfies~\eqref{Eq:FF2CellA} (keep in mind that when $F$ is an identity, $\gamma_0$ and $\gamma_1$ are also identities):
\begin{align*}
	\begin{tikzcd}[bend angle=50,ampersand replacement=\&]
		C^2 \rar{L_1}
			\& C^2 \rar[bend left][domA]{\dom}
				\rar[][codA,domB,description,inner ysep=0]{E_2}
				\rar[bend right][codB,swap]{\cod}
			\& C
		\twocellA[pos=.45]{\eta_2}
		\twocellB[pos=.55]{\epsilon_2}
	\end{tikzcd}
	&=
	\begin{tikzcd}[bend angle=30,ampersand replacement=\&]
		C^2 \rar{L_1}
			\& C^2 \rar[bend left][domA]{\dom}
				\rar[bend right][codA,swap]{\cod} 
			\& C
		\twocellA{\kappa}
	\end{tikzcd} \\
	&=
	\begin{tikzcd}[bend angle=30,ampersand replacement=\&]
		C^2 \rar[bend left][domA]{\dom}
				\rar[bend right][codA,swap]{E_1} 
			\& C.
		\twocellA{\eta_1}
	\end{tikzcd}
\end{align*}

Moreover, it is straightforward to check that $i$ and $p$ are natural families of 2-cells. Specifically, for any pair of 2-cells $\theta_1$ and $\theta_2$
\begin{align*}
\begin{tikzcd}[ampersand replacement=\&]
	C \rar[tick][domA]{E_1\odot E_2}
			\dar[equal]
		\& C \dar[equal] \\
	C \rar[tick][codA,domB]{E_1}
			\dar[swap]{F}
		\& C \dar{F} \\
	D \rar[tick][codB,swap]{E'_1}
		\& D
	\twocellA{p_{E_1,E_2}}
	\twocellB{\theta_1}
\end{tikzcd}
&=
\begin{tikzcd}[ampersand replacement=\&]
	C \rar[tick][domA]{E_1\odot E_2}
			\dar[swap]{F}
		\& C \dar{F} \\
	D \rar[tick][codA,domB]{E'_1\odot E'_2}
			\dar[equal]
		\& D \dar[equal] \\
	D \rar[tick][codB,swap]{E'_1}
		\& D
	\twocellA{\theta_1\odot\theta_2}
	\twocellB{p_{E'_1,E'_2}}
\end{tikzcd}
\\
\begin{tikzcd}[ampersand replacement=\&]
	C \rar[tick][domA]{E_1}
			\dar[equal]
		\& C \dar[equal] \\
	C \rar[tick][codA,domB]{E_1\otimes E_2}
			\dar[swap]{F}
		\& C \dar{F} \\
	D \rar[tick][codB,swap]{E'_1\otimes E'_2}
		\& D
	\twocellA{i_{E_1,E_2}}
	\twocellB{\theta_1\otimes\theta_2}
\end{tikzcd}
&=
\begin{tikzcd}[ampersand replacement=\&]
	C \rar[tick][domA]{E_1}
			\dar[swap]{F}
		\& C \dar{F} \\
	D \rar[tick][codA,domB]{E'_1}
			\dar[equal]
		\& D \dar[equal] \\
	D 		\rar[tick][codB,swap]{E'_1\otimes E'_2}
		\& D
	\twocellA{\theta_1}
	\twocellB{i_{E'_1,E'_2}}
\end{tikzcd}
\end{align*}

As with any 2-cell in $\FFD$, $p$ and $i$ induce 2-cells in $\mathbb{D}$
\[
\begin{tikzcd}[bend angle=30]
	C^2 \rar[bend left][domA]{R_{1\odot 2}}
			\rar[bend right][codA,swap]{R_1} 
		& C^2
	\twocellA{p^R}
\end{tikzcd}
\qquad\text{and}\qquad
\begin{tikzcd}[bend angle=30]
	C^2 \rar[bend left][domA]{L_1}
			\rar[bend right][codA,swap]{L_{1\otimes 2}} 
		& C^2.
	\twocellA{i^L}
\end{tikzcd}
\]
such that
\begin{align}
	\begin{tikzcd}[bend angle=30,ampersand replacement=\&]
		C^2 \rar[bend left][domA]{R_{1\odot 2}}
				\rar[bend right][codA,swap]{R_1} 
			\& C^2 \rar{\dom}
			\& C
		\twocellA{p^R}
	\end{tikzcd}
	&=
	\begin{tikzcd}[bend angle=30,ampersand replacement=\&]
		C^2 \rar{L_1}
			\& C^2 \rar[bend left][domA]{E_2}
				\rar[bend right][codA,swap]{\cod} 
			\& C
		\twocellA{\epsilon_2}
	\end{tikzcd} \label{Eq:DomPR}
	\\
	\begin{tikzcd}[bend angle=30,ampersand replacement=\&]
		C^2 \rar[bend left][domA]{L_1}
				\rar[bend right][codA,swap]{L_{1\otimes 2}} 
			\& C^2 \rar{\cod}
			\& C
		\twocellA{i^L}
	\end{tikzcd}
	&=
	\begin{tikzcd}[bend angle=30,ampersand replacement=\&]
		C^2 \rar{R_1}
			\& C^2 \rar[bend left][domA]{\dom}
				\rar[bend right][codA,swap]{E_2} 
			\& C
		\twocellA{\eta_2}
	\end{tikzcd} \label{Eq:CodIL}
\end{align}

Now suppose given three functorial factorizations $E_1,E_2,E_3$ on an object $C$. We define a 2-cell in $\mathbb{D}$
\[
\begin{tikzcd}[row sep=0ex, bend angle=15]
	{} & C^2 \drar[bend left][]{L_3} & \\
	|[alias=domA]| C^2 \urar[bend left][]{R_{1\odot 2}} \drar[bend right][swap]{L_{1\otimes 3}}
		&& |[alias=codA]| C^2 \\
	{} & C^2 \urar[bend right][swap]{R_2} &
	\twocellA{w}
\end{tikzcd}
\]
such that
\begin{align}
	\begin{tikzcd}[row sep=0ex,bend angle=15,ampersand replacement=\&,baseline=(B.base)]
		\& C^2 \drar[bend left][]{L_3} \&\& \\
		|[alias=B,alias=domA]| C^2 \urar[bend left][]{R_{1\odot 2}} 
				\drar[bend right][swap]{L_{1\otimes 3}}
			\&\& |[alias=codA]| C^2 \rar{\dom} 
			\& C \\
		\& C^2 \urar[bend right][swap]{R_2} \&\&
		\twocellA{w}
	\end{tikzcd}
	&=
	\begin{tikzcd}[bend angle=30,ampersand replacement=\&,baseline=(B.base)]
		|[alias=B]| C^2 \rar[bend left][domA]{L_1}
				\rar[bend right][codA,swap]{L_{1\otimes 3}}
			\& C^2 \rar{E_2}
			\& C
		\twocellA{i^L}
	\end{tikzcd} \label{Eq:DomW}
	\\
	\begin{tikzcd}[row sep=0ex, bend angle=15,ampersand replacement=\&,baseline=(B.base)]
		\& C^2 \drar[bend left][]{L_3} \&\& \\
		|[alias=B,alias=domA]| C^2 \urar[bend left][]{R_{1\odot 2}} 
				\drar[bend right][swap]{L_{1\otimes 3}}
			\&\& |[alias=codA]| C^2 \rar{\cod} \& C \\
		\& C^2 \urar[bend right][swap]{R_2} \&\&
		\twocellA{w}
	\end{tikzcd}
	&=
	\begin{tikzcd}[bend angle=30,ampersand replacement=\&,baseline=(B.base)]
		|[alias=B]| C^2 \rar[bend left][domA]{R_{1\odot 2}}
				\rar[bend right][codA,swap]{R_1}
			\& C^2 \rar{E_3}
			\& C.
			\twocellA{p^R}
	\end{tikzcd} \label{Eq:CodW}
\end{align}

Using the universal property for $C^2$, it suffices to check that
\[
\begin{tikzcd}[bend angle=60]
	C^2 \rar[bend left][domA]{L_1}
			\rar[][codA,swap]{L_{1\otimes 3}}
		& C^2 \rar[][domB]{E_2}
			\rar[bend right][codB,swap]{\cod}
		& C
	\twocellA{i^L}
	\twocellB{\epsilon_2}
\end{tikzcd}
=
\begin{tikzcd}[bend angle=60]
	C^2 \rar[][domA]{R_{1\odot 2}}
			\rar[bend right][codA,swap]{R_1}
		& C^2 \rar[bend left][domB]{\dom}
			\rar[][codB,swap]{E_3}
		& C
	\twocellA{p^R}
	\twocellB{\eta_3}
\end{tikzcd}
\]
and a quick check using equations~\eqref{Eq:DomPR} and~\eqref{Eq:CodIL} shows that both are equal to
\[
\begin{tikzcd}[row sep=0ex,bend angle=60]
	{} & C^2 \drar[bend left][domA]{E_2}
			\drar[][codA,sloped,swap,pos=.6,inner sep=.1ex]{\cod}
		& \\
	C^2 \urar{L_1} \drar[swap]{R_1}
		&& C \\
	{} & C^2 \urar[][domB,sloped,pos=.7,inner sep=.2ex]{\dom}
			\urar[bend right][codB,swap]{E_3} &
	\twocellA{\epsilon_2}
	\twocellB{\eta_3}
\end{tikzcd}
\]
where the inner diamond is the equality $\cod L_1=\dom R_1=E_1$.

We also check that $w$ is natural with respect to 2-cells in $\FFD$ in the following sense: given three 2-cells $\theta_1$, $\theta_2$, and $\theta_3$, there is an equality
\[
\begin{tikzcd}[row sep=0ex, bend angle=15]
	{} & C^2 \drar[bend left][inner sep=.2ex]{L_3} 
		& \\
	|[alias=domA]| C^2 	\urar[bend left][inner sep=.2ex]{R_{1\odot 2}} 
			\drar[bend right][domB,pos=.3,inner sep=.2ex]{L_{1\otimes 3}}
			\ar{dd}[swap]{\hat{F}}
		&& |[alias=codA]| C^2 \ar{dd}{\hat{F}} \\
	{} & C^2 \urar[bend right][domC,pos=.6,inner sep=.2ex]{R_2} 
			\ar{dd}{\hat{F}}
		& \\[3ex]
	D^2 \drar[bend right][codB,swap,inner sep=.2ex]{L'_{1\otimes 3}}
		&& D^2 \\
	{} & D^2 \urar[bend right][codC,swap]{R'_2}
		&
	\twocellA{w}
	\twocellB[pos=.4,xshift=2pt]{(\theta_1\otimes\theta_3)^L}
	\twocellC[pos=.4]{\theta_2^R}
\end{tikzcd}
=
\begin{tikzcd}[row sep=0ex, bend angle=15]
	{} & C^2 \drar[bend left][domB,inner sep=.2ex]{L_3}
			\ar{dd}{\hat{F}}
		& \\
	C^2 \urar[bend left][domA,inner sep=.2ex]{R_{1\odot 2}}
			\ar{dd}[swap]{\hat{F}}
		&& C^2 \ar{dd}{\hat{F}} \\[3ex]
	{} & D^2 \drar[bend left][codB,swap,inner sep=.2ex,pos=.6]{L'_3}
		& \\
	|[alias=domC]| D^2 \urar[bend left][codA,swap,inner sep=.2ex,pos=.3]{R'_{1\odot 2}}
			\drar[bend right][swap,inner sep=.2ex]{L'_{1\otimes 3}}
		&& |[alias=codC]| D^2. \\
	{} & D^2 \urar[bend right][swap,inner sep=.2ex]{R'_2}
		&
	\twocellA[pos=.6]{(\theta_1\odot\theta_2)^R}
	\twocellB[pos=.6]{\theta_3^L}
	\twocellC{w}
\end{tikzcd}
\]
To verify this equation, it suffices to check equality upon right composition with $\gamma_0$ and $\gamma_1$. We will illustrate the $\gamma_1$ case, making use of the naturality of $i$:
\begin{multline*}
\begin{tikzcd}[row sep=0ex, bend angle=15, ampersand replacement=\&]
	{} \& C^2 \drar[bend left][inner sep=.2ex]{L_3} 
		\&\& \\
	|[alias=domA]| C^2 \urar[bend left,inner sep=.2ex][]{R_{1\odot 2}} 
			\drar[bend right][domB,pos=.3,inner sep=.2ex]{L_{1\otimes 3}}
			\ar{dd}[swap]{\hat{F}}
		\&\& |[alias=codA]| C^2 \ar{dd}{\hat{F}}
			\rar[][domD]{\dom}
		\& C \ar{dd}{F} \\
	{} \& C^2 \urar[bend right][domC,inner sep=.2ex,pos=.6]{R_2} 
			\ar{dd}{\hat{F}}
		\&\& \\[3ex]
	D^2 \drar[bend right][codB,swap,inner sep=.2ex]{L'_{1\otimes 3}}
		\&\& D^2 \rar[][codD,swap]{\dom}
		\& D \\
	{} \& D^2 \urar[bend right][codC,swap,inner sep=.2ex]{R'_2}
		\&\&
	\twocellA{w}
	\twocellB[pos=.4,xshift=2pt]{(\theta_1\otimes\theta_3)^L}
	\twocellC[pos=.4]{\theta_2^R}
	\twocellD{\gamma_1}
\end{tikzcd}
=
\begin{tikzcd}[bend angle=50, row sep=6ex, ampersand replacement=\&]
	C^2 \rar[bend left][domA]{L_1}
			\rar[][codA,domB,swap]{L_{1\otimes 3}}
			\dar[swap]{\hat{F}}
		\& C^2 \rar[][domC]{E_2}
			\dar{\hat{F}}
		\& C \dar{F} \\
	D^2 \rar[][codB,swap]{L_{1'\otimes 3'}}
		\& D^2 \rar[][codC,swap]{E'_2}
		\& D
	\twocellA{i^L}
	\twocellB{(\theta_1\otimes\theta_3)^L}
	\twocellC{\theta_2}
\end{tikzcd}
\\
=
\begin{tikzcd}[bend angle=50, row sep=6ex, ampersand replacement=\&]
	C^2 \rar[][domA]{L_1}
			\dar[swap]{\hat{F}}
		\& C^2 \rar[][domB]{E_2}
			\dar{\hat{F}}
		\& C \dar{F} \\
	D^2 \rar[][codA,domC]{L'_1}
			\rar[bend right][codC,swap]{L'_{1\otimes 3}}
		\& D^2 \rar[][codB,swap]{E'_2}
		\& D
	\twocellA{\theta_1^L}
	\twocellB{\theta_2}
	\twocellC{i^L}
\end{tikzcd}
=
\begin{tikzcd}[row sep=0ex, bend angle=15, ampersand replacement=\&]
	{} \& C^2 \drar[bend left][domB,inner sep=.2ex]{L_3}
			\ar{dd}{\hat{F}}
		\&\& \\
	C^2 \urar[bend left][domA,inner sep=.2ex]{R_{1\odot 2}}
			\ar{dd}[swap]{\hat{F}}
		\&\& C^2 \ar{dd}{\hat{F}}
			\rar[][domD]{\dom}
		\& C \ar{dd}{F} \\[3ex]
	{} \& D^2 \drar[bend left][codB,swap,inner sep=.2ex,pos=.6]{L'_3}
		\&\& \\
	|[alias=domC]| D^2 
			\urar[bend left][codA,swap,inner sep=.2ex,pos=.3]{R'_{1\odot 2}}
			\drar[bend right][swap,inner sep=.2ex]{L'_{1\otimes 3}}
		\&\& |[alias=codC]| D^2 \rar[swap]{\dom}
		\& D. \\
	{}
		\& D^2 \urar[bend right][swap]{R'_2}
		\&\&
	\twocellA{(\theta_1\odot\theta_2)^R}
	\twocellB{\theta_3^L}
	\twocellC{w}
	\twocellD{\gamma_1}
\end{tikzcd}
\end{multline*}

Finally, given four functorial factorizations $E_1, E_2, E_3, E_4$ on an object $C$, we define the 2-cell 
\[
\begin{tikzcd}[column sep=14ex]
	C \rar[tick][domA]{(1\odot 2)\otimes(3\odot 4)} 
			\dar[equal] 
		& C \dar[equal] \\
	C \rar[tick][codA,swap]{(1\otimes 3)\odot(2\otimes 4)}
		& C
	\twocellA{z_{1,2,3,4}}
\end{tikzcd}
\]
in $\FFD$, where $(1\odot 2)$ is shorthand for $(E_1,\eta_1,\epsilon_1)\odot(E_2,\eta_2,\epsilon_2)$, to have the underlying 2-cell in $\mathbb{D}$
\[
\begin{tikzcd}[row sep=0ex, bend angle=15,ampersand replacement=\&]
	\& C^2 \drar[bend left][]{L_3} \&\& \\
	|[alias=domA]| C^2 \urar[bend left][]{R_{1\odot 2}} 
			\drar[bend right][swap]{L_{1\otimes 3}}
		\&\& |[alias=codA]| C^2 \rar{E_4} \& C. \\
	\& C^2 \urar[bend right][swap]{R_2} \&\&
	\twocellA{w}
\end{tikzcd}
\]
The naturality of $z$ follows immediately from that of $w$, but we still need to check that this satisfies equations~\eqref{Eq:FF2CellA} and~\eqref{Eq:FF2CellB}. We will leave the details to the reader, but note that~\eqref{Eq:FF2CellB} comes down to the verification of the equality
\[
\begin{tikzcd}[row sep=0ex, bend angle=15, ampersand replacement=\&, baseline=(B.base)]
	\& C^2 \drar[bend left][inner sep=.2ex]{L_3} \&\& \\
	|[alias=domB]| C^2 	\urar[bend left=65,looseness=1.2][domA]{\id}
			\urar[bend left][codA,swap,inner sep=.2ex,pos=.3]{R_{1\odot 2}}
			\drar[bend right][swap,inner sep=.2ex]{L_{1\otimes 3}}
		\&\& |[alias=codB]| C^2 \rar[bend left=50][domC]{\dom}
			\rar[][codC,swap]{E_4}
		\& |[alias=B]| C \\
	\& C^2 \urar[bend right][swap]{R_2} \&\&
	\twocellA[pos=.6]{\vec{\eta}_{1\odot 2}}
	\twocellB{w}
	\twocellC{\eta_4}
\end{tikzcd}
=
\begin{tikzcd}[bend angle=30, baseline=(B.base)]
	|[alias=B]| C^2 \rar{L_{1\otimes 3}}
		& C^2 \rar[bend left][domA]{\id}
			\rar[bend right][codA,swap]{R_2}
		& C^2 \rar[bend left][domB]{\dom}
			\rar[bend right][codB,swap]{E_4}
		& C,
	\twocellA{\vec{\eta}_2}
	\twocellB{\eta_4}
\end{tikzcd}
\]
which follows from equation~\eqref{Eq:DomW} and the fact that $\dom\circ i^L = \id_{\dom}$.
\end{proof}

\begin{lemma}
	There is a strict double functor $R\colon\FFD_{\otimes}\to\mathbb{D}$ whose behavior on 2-cells is
	\[
	\begin{tikzcd}[column sep=large]
		C \rar[tick][domA]{(E_1,\eta_1,\epsilon_1)} \dar[swap]{F}  & C \dar{F} \\
		D \rar[tick][codA,swap]{(E_2,\eta_2,\epsilon_2)} & D
		\twocellA{\theta}
	\end{tikzcd}
	\quad \mapsto \quad
	\begin{tikzcd}
		C^2 \rar[tick][domA]{R_1} \dar[swap]{\hat{F}}
			& C^2 \dar{\hat{F}} \\
		D^2 \rar[tick][codA,swap]{R_2}
			& D^2
		\twocellA{\theta^R}
	\end{tikzcd}
	\]
	and a double functor $L\colon\FFD_{\odot}\to\mathbb{D}$ whose behavior on 2-cells is
	\[
	\begin{tikzcd}[column sep=large]
		C \rar[tick][domA]{(E_1,\eta_1,\epsilon_1)} \dar[swap]{F}  & C \dar{F} \\
		D \rar[tick][codA,swap]{(E_2,\eta_2,\epsilon_2)} & D
		\twocellA{\theta}
	\end{tikzcd}
	\quad \mapsto \quad
	\begin{tikzcd}
		C^2 \rar[tick][domA]{L_1} \dar[swap]{\hat{F}}
			& C^2 \dar{\hat{F}} \\
		D^2 \rar[tick][codA,swap]{L_2}
			& D^2
		\twocellA{\theta^L}
	\end{tikzcd}
	\]
\end{lemma}
\begin{corollary}\label{Cor:RLMon}
	$R$ and $L$ respectively induce functors $\Mon(\FFD)\to\Mon(\mathbb{D})$ and $\Comon(\FFD)\to\Comon(\mathbb{D})$.
\end{corollary}

Up to this point, we have demonstrated that given any double category $\mathbb{D}$ having arrow objects, there is a 2-fold double category $\FFD$ of functorial factorizations in $\mathbb{D}$. The last thing we want to say about this construction is that a cyclic action on $\mathbb{D}$ lifts to one on $\FFD$, and hence also to one on $\mathrm{Bimon}(\mathbb{\FFD})$.

The cyclic action on objects and vertical morphisms is given directly by that on $\mathbb{D}$. Given a horizontal 1-cell $(E,\eta,\epsilon)$ on an object $C$, we define the 1-cell $(E,\eta,\epsilon)^{\bullet}$ on $C^{\bullet}$ to be $(E^{\bullet},\epsilon^{\bullet},\eta^{\bullet})$. This also implies that the cyclic action swaps $L$ and $R$ for any given functorial factorization.

A quick look at the definitions of the two horizontal compositions is now enough to see that for any two functorial factorizations $E_1$ and $E_2$, we have
\[
	(E_1\otimes E_2)^{\bullet} = E_1^{\bullet}\odot E_2^{\bullet}
	\qquad\text{and}\qquad
	(E_1\odot E_2)^{\bullet} = E_1^{\bullet}\otimes E_2^{\bullet}
\]

Similarly, the cyclic action on 2-cells in $\FFD$ is given by the cyclic action in $\mathbb{D}$ on the underlying 2-cell. This gives a valid 2-cell in $\FFD$ since the cyclic action simply swaps the equations~\eqref{Eq:FF2CellA} and~\eqref{Eq:FF2CellB}.

%TODO: define category \Awfs{\dblcat{D}}=\Bimon(\FFD), mention it inherits cyclic action.
