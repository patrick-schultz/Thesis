% -*- root: thesis.tex -*-

\usepackage{mathtools,amssymb,amsthm}
\usepackage{verbatim}
\usepackage{datetime}
\usepackage[T1]{fontenc}
\usepackage[sc]{mathpazo}
\usepackage{mathrsfs}
\usepackage{euscript}
\usepackage{MnSymbol}
\usepackage{paralist}
\usepackage{tikz, tikz-cd}
\usepackage{setspace}	% used for single- and double- spacing
\usepackage{fancyhdr}	% used for page numbering
\usepackage{indentfirst} % this indents first paragraph of each section
\usepackage{cleveref}
%\usepackage{todo}
\usetikzlibrary{decorations.markings}
%\linespread{1.05}

%\chapterstyle{dash}
%\pagestyle{headings}

% page layout margins
% Note: 1in = 72.27pt;  
% thus - (lower margin) = 
%	1in + \headsep+ topmargin + \headheight + \textheight - 11in
	\hoffset = .55in
	\oddsidemargin = 0in
	\voffset = 0in
	\topmargin = 0.1in
	\headheight = 15pt 	% do not make too small
	\headsep = 15pt	% do not make too small
	\textheight = 613pt
	\footskip = 0in
	\textwidth = 5.9in
	\marginparsep = 0.1in 
	\marginparwidth = 0.9in 
	\marginparpush = 0in

	\newcommand{\TOP}{\vspace*{25pt}} 
	% this sets the extra vertical space in the following preliminary pages:
				% approval, abstract, CV, acknowledgement


% Formatting the header.  This controls page number placement.  
% See documentation for fancyhdr.sty on CTAN for more information
	\pagestyle{fancyplain}
	\lhead{}
	\chead{}
	\rhead{\thepage}
	\lfoot{}
	\cfoot{}
	\rfoot{}
	\renewcommand{\headrulewidth}{0in}
	\renewcommand{\footrulewidth}{0in}

 


% ---------------- More Details  ------------------

%%%
%  From the manual:
%    * It is preferable not to right justify text, as this frequently causes 
%      inconsistency in spacing between words.
%
%    * It is preferable not to justify the right margin.
%%
% As I disagree with this assesment and it is optional, I have removed it.
	%\raggedright 	% ragged right edge
	\doublespacing 	
	% default spacing; should be double for Prelim Pages to work out correctly
	

	
% ----Theorem Numbering and formatting -------------------------------------
	
	% this sets how your theorems, etc, are numbered
	% options: (blank), chapter, section
	\newcommand{\Number}{chapter} 	
	\theoremstyle{plain}
		\newtheorem{theorem}{Theorem}[\Number]
		\newtheorem*{theorem*}{Theorem}
		\newtheorem{lemma}[theorem]{Lemma}
		\newtheorem*{lemma*}{Lemma}
		\newtheorem{proposition}[theorem]{Proposition}
		\newtheorem{corollary}[theorem]{Corollary} 
	\theoremstyle{definition}
		\newtheorem{definition}[theorem]{Definition}
	\theoremstyle{remark}
		\newtheorem{remark}[theorem]{Remark}
		\newtheorem{example}[theorem]{Example}


%\newcommand{\Tref}[1]{{Theorem~\ref{#1}}}
\DeclareMathOperator{\id}{id}
\DeclareMathOperator{\dom}{dom}
\DeclareMathOperator{\cod}{cod}
\DeclareMathOperator{\dvert}{Vert}


% \theoremstyle{plain}
% \newtheorem{theorem}{Theorem}[chapter]
% \newtheorem*{theorem*}{Theorem}
% \newtheorem{proposition}[theorem]{Proposition}
% \newtheorem{corollary}[theorem]{Corollary}
% \newtheorem{lemma}[theorem]{Lemma}
% \newtheorem*{lemma*}{Lemma}

% \theoremstyle{definition}
% \newtheorem{definition}[theorem]{Definition}
% \newtheorem{exercise}{Exercise}[chapter]

% \theoremstyle{remark}
% \newtheorem{example}[theorem]{Example}
% \newtheorem{remark}[theorem]{Remark}

\newcommand{\prodb}{\mathbin{\Pi}}

\newcommand{\cat}[1]{\mathscr{#1}}
\newcommand{\Cat}[1]{\mathbf{#1}}
%\newcommand{\hom}{\mathrm{hom}}
\newcommand{\twocat}[1]{\mathcal{#1}}
\newcommand{\dblcat}[1]{\mathbb{#1}}
\newcommand{\btwo}{\mathbf{2}}
\newcommand{\FF}{\mathbb{F}\mathrm{F}}
\newcommand{\FFD}{\FF(\dblcat{D})}
\newcommand{\Mon}{\mathrm{Mon}}
\newcommand{\DMon}{\mathbb{M}\mathrm{on}}
\newcommand{\Comon}{\mathrm{Comon}}
\newcommand{\DComon}{\mathbb{C}\mathrm{omon}}
\newcommand{\Bimon}{\mathrm{Bimon}}
\newcommand{\Sq}{\mathbb{S}\mathrm{q}}
\newcommand{\Span}{\mathbb{S}\mathrm{pan}}
\newcommand{\Hor}{\twocat{H}or}
\newcommand{\LAdj}{\dblcat{L}\Cat{Adj}}
\newcommand{\RAdj}{\dblcat{R}\Cat{Adj}}
\newcommand{\MAdjC}{\Cat{MAdj}}
\newcommand{\MAdj}{\dblcat{M}\Cat{Adj}}

\newcommand{\op}[1]{{#1}^{\text{op}}}
\newcommand{\vop}[1]{{#1}^{\text{vop}}}
\newcommand{\hop}[1]{{#1}^{\text{hop}}}

\newcommand{\Alg}{\mathrm{Alg}}
\newcommand{\Coalg}{\mathrm{Coalg}}
\newcommand{\RAlg}{\mathbb{R}\text{-}\Alg}
\newcommand{\LCoalg}[1][]{\mathbb{L}_{#1}\text{-}\Coalg}
\newcommand{\LCoalgA}{\mathbb{L}_1\text{-}\Coalg}
\newcommand{\LCoalgB}{\mathbb{L}_2\text{-}\Coalg}

\newcommand{\twocell}[3][]{\arrow[draw=none,to path={(dom#2.center)--(cod#2.center)\tikztonodes}]{}[anchor=center,#1]{\Downarrow #3}}
\newcommand{\twocellalt}[3][]{\arrow[draw=none,to path={(dom#2.center)--(cod#2.center)\tikztonodes}]{}[anchor=center,#1]{#3}}
\newcommand{\twocellA}[2][]{\twocell[#1]{A}{#2}}
\newcommand{\twocellB}[2][]{\twocell[#1]{B}{#2}}
\newcommand{\twocellC}[2][]{\twocell[#1]{C}{#2}}
\newcommand{\twocellD}[2][]{\twocell[#1]{D}{#2}}
\newcommand{\twocellE}[2][]{\twocell[#1]{E}{#2}}
\newcommand{\twocellF}[2][]{\twocell[#1]{F}{#2}}

\tikzset{tick/.style={postaction={decorate,decoration={markings,mark=at position 0.5 with {\draw[-] (0,.4ex) -- (0,-.4ex);}}}}}
\tikzset{dom/.style={append after command={coordinate[alias=dom#1]}},
		domA/.style={dom=A}, domB/.style={dom=B},
		domC/.style={dom=C}, domD/.style={dom=D},
		domE/.style={dom=E}, domF/.style={dom=F}}
\tikzset{cod/.style={append after command={coordinate[alias=cod#1]}},
		codA/.style={cod=A}, codB/.style={cod=B},
		codC/.style={cod=C}, codD/.style={cod=D},
		codE/.style={cod=E}, codF/.style={cod=F}}