% -*- root: thesis.tex -*-

\usepackage{mathtools,amssymb,amsthm}
\usepackage{datetime}
\usepackage[T1]{fontenc}
\usepackage[sc]{mathpazo}
\usepackage{MnSymbol}
\usepackage{paralist}
\usepackage{tikz, tikz-cd}
\usetikzlibrary{decorations.markings}
\linespread{1.05}

%\chapterstyle{dash}
\pagestyle{headings}

\newcommand{\Tref}[1]{{Theorem~\ref{#1}}}
\DeclareMathOperator{\id}{id}
\DeclareMathOperator{\dom}{dom}
\DeclareMathOperator{\cod}{cod}
\DeclareMathOperator{\dvert}{Vert}


\theoremstyle{plain}
\newtheorem{theorem}{Theorem}[chapter]
\newtheorem*{theoremnn}{Theorem}
\newtheorem{proposition}[theorem]{Proposition}
\newtheorem{corollary}[theorem]{Corollary}
\newtheorem{lemma}[theorem]{Lemma}

\theoremstyle{definition}
\newtheorem{definition}[theorem]{Definition}
\newtheorem{exercise}{Exercise}[chapter]

\theoremstyle{remark}
\newtheorem{example}[theorem]{Example}
\newtheorem{remark}[theorem]{Remark}


\newcommand{\cat}[1]{\mathcal{#1}}
\newcommand{\btwo}{\mathbf{2}}
\newcommand{\FF}[1]{\mathbb{F}\mathrm{F}(\mathbb{#1})}
\newcommand{\Mon}{\mathrm{Mon}}
\newcommand{\Comon}{\mathrm{Comon}}
\newcommand{\Sq}{\mathbb{S}\mathrm{q}}
\newcommand{\Alg}{\mathrm{Alg}}
\newcommand{\Coalg}{\mathrm{Coalg}}

\newcommand{\twocell}[3][]{\arrow[draw=none,to path={(dom#2.center)--(cod#2.center)\tikztonodes}]{}[anchor=center,#1]{\Downarrow #3}}
\newcommand{\twocellalt}[3][]{\arrow[draw=none,to path={(dom#2.center)--(cod#2.center)\tikztonodes}]{}[anchor=center,#1]{#3}}
\newcommand{\twocellA}[2][]{\twocell[#1]{A}{#2}}
\newcommand{\twocellB}[2][]{\twocell[#1]{B}{#2}}
\newcommand{\twocellC}[2][]{\twocell[#1]{C}{#2}}
\newcommand{\twocellD}[2][]{\twocell[#1]{D}{#2}}
\newcommand{\twocellE}[2][]{\twocell[#1]{E}{#2}}
\newcommand{\twocellF}[2][]{\twocell[#1]{F}{#2}}

\tikzset{tick/.style={postaction={decorate,decoration={markings,mark=at position 0.5 with {\draw[-] (0,.4ex) -- (0,-.4ex);}}}}}
\tikzset{dom/.style={append after command={coordinate[alias=dom#1]}},
		domA/.style={dom=A}, domB/.style={dom=B},
		domC/.style={dom=C}, domD/.style={dom=D},
		domE/.style={dom=E}, domF/.style={dom=F}}
\tikzset{cod/.style={append after command={coordinate[alias=cod#1]}},
		codA/.style={cod=A}, codB/.style={cod=B},
		codC/.style={cod=C}, codD/.style={cod=D},
		codE/.style={cod=E}, codF/.style={cod=F}}