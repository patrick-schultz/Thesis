% -*- root: thesis.tex -*-

\chapter{Introduction}

The theory of model categories has a long history, and has proven itself indispensable to several recent advances in mathematics, such as higher category theory, so-called spectral algebraic geometry, even finding applications in computer science and the foundations of mathematics with homotopy type theory.

In the modern treatment, a model category is defined to consist of two \emph{weak factorization systems} on a category $\cat{C}$ (e.g.~\cite{mp:more-concise}). A weak factorization system is a structure which consists of two classes of morphisms of $\cat{C}$, call them $\mathcal{L}$ and $\mathcal{R}$, such that solutions to certain lifting problems involving one morphism from each class always exist, plus an axiom that every morphism of $\cat{C}$ factors as a morphism from $\mathcal{L}$ followed by a morphism from $\mathcal{R}$. In the past 20 or so years, most authors have added the requirement that this factorization can be chosen in a natural/functorial way.

Taking this one step further, in~\cite{gt:nwfs} the category theorists Marco Grandis and Walter Tholen proposed a strengthening of weak factorization systems which they called \emph{natural} weak factorization systems, today most often referred to as \emph{algebraic} weak factorization systems, or awfs for short. An awfs strengthens the structure in a way which provides a canonical \emph{choice} of solution to every lifting problem, in such a way that these choices are coherent or natural in a precise sense. The structure of an awfs consists of a monad and a comonad on the category of arrows satisfying some axioms, and the categories of algebras and coalgebras for these respectively provide an algebraic analogue of the right and left classes of maps of the factorization system.

It at first seems as though this extra structure is \emph{too} strict, and that examples would be hard to find. But in~\cite{garner:nwfs} and~\cite{garner:soa}, the category theorist Richard Garner provided a modification of Quillen's small object argument which generates algebraic weak factorization systems, and which furthermore has much nicer convergence properties than Quillen's original construction, and often generates a smaller and easier to understand factorization. Best of all, Garner's small object argument operates under almost identical assumptions as Quillen's, so that in practice any cofibrantly generated weak factorization system can be strengthened to an algebraic one.

In her Ph.D. thesis, \cite{riehl:nwfs-model} and~\cite{riehl:nwfs-monoidal}, Emily Riehl began the project of developing a full-fledged theory of \emph{algebraic model structures}, built out of two awfs analogously to an ordinary model structure. Since then, she and her collaborators have continued to develop and find applications of this theory, e.g.~\cite{cgr:mates}, \cite{br:funct-facts}, and~\cite{bmr:six}. Of particular interest for us, she gives the first definition of algebraic Quillen functors.

If we define a lax functor of weak factorization systems to be a functor between categories each equipped with a wfs which takes morphisms in the right class of the first to morphisms in the right class of the second, then a right Quillen functor between model categories is simply a functor which is a lax functor with respect to both weak factorization systems making up the model structures. Likewise, a colax functor of wfs preserves the left classes, and a left Quillen functor is colax with respect to both wfs. It is a basic fact from model category theory that given an adjunction between weak factorization systems, the left adjoint is colax if and only if the right adjoint is lax.

An algebraic version of Quillen functors should continue to have this property, as the definition Riehl gives does, but making this precise requires some pieces of classical category theory: the mates correspondence, and double categories. The mates correspondence is a natural bijection between natural transformations involving two pairs of adjoint functors which generalizes the hom-set bijection of an adjunction. The naturality of the mates correspondence is best formulated using double categories, and for this reason double categories play a central role in this thesis. Double categories are a kind of two-dimensional categorical structure, similar to a 2-category but with separate classes of vertical and horizontal morphisms, and with square shaped 2-cells which can be composed both vertically and horizontally. Double categories were first defined by Ehresmann in the `60's and then largely ignored, but have recently enjoyed a resurgence of interest, see e.g.~~\cite{shulman:framed}, \cite{dpp:spans}, \cite{fiore:monads}.

In~\cite{garner:nwfs} and~\cite{garner:soa}, Garner proves as a technical tool that algebraic weak factorization systems can be seen as bialgebras in a category of functorial factorizations, supporting the intuition that an awfs is given by a functorial factorization equipped with (co)algebraic structure. The category of functorial factorizations he constructs is not a symmetric or braided monoidal category, but a so-called two-fold monoidal category, which is a generalization of braided monoidal category having two compatible monoidal structures, and in which the definition of bialgebra still makes sense.

We find this a very nice conceptual way of understanding algebraic weak factorization systems, but it has the shortcoming of being unable to say anything about functors between awfs on different categories. It is one of our primary goals of this thesis to extend this awfs-as-bialgebras perspective to include the (co)lax morphisms of awfs defined in~\cite{riehl:nwfs-model}. To do this, we have had to find a common generalization of double categories, used to formalize the mates correspondance and the duality relating lax and colax morphisms, and two-fold monoidal categories, in which the notion of bialgebra makes sense. We call this common generalization a \emph{two-fold double category}.

We show that a kind of bialgebra can be defined in any two-fold double category, which we call bimonads, and that the natural generalization of bialgebra morphism bifurcates into lax and colax morphisms of bimonads. One main result of this thesis is that there is a two-fold double category of functorial factorizations (in any 2-category), and that bimonads and (co)lax morphisms of bimonads in this two-fold double category correspond precisely to awfs and (co)lax morphisms of awfs.

In the second part of her thesis, published as~\cite{riehl:nwfs-monoidal}, Riehl develops a theory of monoidal algebraic model categories, ultimately based on an algebraic strengthening of the notion of two-variable Quillen adjunction. Classically, a 2-variable Quillen adjunction is a functor of two variables with both adjoints (one in each variable), such that the induced pushout-product of two maps in the left classes is again in the left class. The primary motivation for this definition is to be able to define monoidal model categories, in which the tensor product is part of a 2-variable Quillen adjunction. To give an algebraic version of this definition, Riehl had to extend the mates correspondence to multivariable adjunctions, which she does with her coauthors in~\cite{cgr:mates}. The mates correspondence for multivariable adjunctions is most easily understood in terms of cyclic double multicategories, a kind of structure defined in~\cite{cgr:mates} which generalizes double categories to allow for morphisms with multiple inputs, with a cyclic action which formalizes the mates correspondence.

In order to incorporate multivariable morphisms into the bialgebraic view of awfs, we have developed a common generalization of two-fold double categories and cyclic double multicategories. Another main result of this thesis is that the pushout product---central to the definition of Quillen bifunctor, and hence to monoidal model categories, simplicial model categories, etc.---satisfies a universal property in the framework of cyclic double multicategories. The author is particularly pleased with this result, as the need for the pushout product in the axioms of monoidal model categories and simplicial model categories had always seemed slightly mysterious and ad hoc. This universal property provides a conceptual explanation: the pushout product defines the universal way of lifting a multivariable adjunction to arrow categories. This also allows us to define multivariable morphisms of bimonads in a cyclic two-fold double multicategory, generalizing the multivariable morphisms of awfs given in~\cite{riehl:nwfs-monoidal}.

A primary motivation for this work was to develop the theory of awfs at a high level of generality. In particular, all of the constructions and theorems of this thesis work just as well in any 2-category satisfying minor completeness conditions as they do in the 2-category of categories. For example, in~\cite{bmr:six} the authors make use of \emph{enriched} algebraic weak factorization systems, in which stronger enriched lifting properties are required. (Note that this is different than enriched model categories in the sense of, e.g., simplicial model categories.) This thesis provides a framework in which the core theory of awfs can be developed in great generality, saving the effort of reproving results for enriched awfs and any other variations of awfs yet to be considered, and it makes a start of proving the most important results in this greater generality.

%\subsection*{Overview}
\section*{Overview}

In \cref{Ch:Wfs}, we review the definitions of algebraic weak factorization system and morphisms of algebraic weak factorization systems, trying to lead up to the (abstract) definitions in a natural way.

In \cref{Ch:Double} we review the definition of double category, as well recording some constructions which will be needed later on. Of these, the definitions of arrow objects in a double category and of fully-faithful lax double functors are (to the best of our knowledge) original.

In \cref{Ch:2Fold} we define two-fold double categories. Generalizing bialgebras in a two-fold monoidal category, we define bimonads and (co)lax morphisms of bimonads in a two-fold double category.

In~\cite{cgr:mates}, the authors show that the mates correspondence can be conveniently expressed as the existence of a cyclic action on a double category of adjunctions. In \cref{Ch:Cyclic}, we show how to generalize the cyclic action as in~\cite{cgr:mates} to the two-fold double categories defined in \cref{Ch:2Fold}, defining what we call a cyclic two-fold double category. We show that a cyclic action interacts well with bimonads in a two-fold double category, extending to a cyclic action on the category of bimonads. This cyclic action is the abstract form of the fact that an algebraic Quillen adjunction can be specified \emph{either} by a lax stucture on the right adjoint \emph{}

In \cref{Ch:FuncFact}, we begin the core work of this thesis, constructing a cyclic two-fold double category of functorial factorizations in an arbitrary double category which has all arrow objects. Then in \cref{Ch:Awfs} we show that given any 2-category $\twocat{D}$ with arrow objects, bimonads in the cyclic two-fold double category of adjunctions in $\twocat{D}$ are precisely algebraic weak factorization systems in $\twocat{D}$. This is essentially proven already in~\cite{garner:nwfs}, but now the bimonad morphisms correspond precisely to the colax morphisms of awfs defined in~\cite{riehl:nwfs-model}, which are the algebraic version of functors preserving the left class of the factorization system. The cyclic action induced on the category of bimonads provides a concise expression of the self-duality of the awfs structure, and the algebraic analogue of the classical fact that a left adjoint preserves the left class of a factorization system if and only if the right adjoint preserves the right class.

Garner proves in~\cite{garner:nwfs} and~\cite{garner:soa} that instead of specifying both the monad and comonad halves of the awfs structure, it is equivalent to define the comonad, plus a functorial composition on the category of coalgebras. This generalizes the classical fact that the left (and right) class of maps is closed under composition, but more importantly provides a convenient technical tool for constructing algebraic weak factorization systems. Similarly, in~\cite{riehl:nwfs-model} Riehl proves that an equivalent definition of colax morphism of awfs is a functor which lifts to the categories of coalgebras for the comonads, and which also preserves the composition of coalgebras. She uses both of these theorems repeatedly throughout the paper.

In \cref{Ch:RAlg} we lay the groundwork towards proving a generalization of these theorems in the framework of cyclic two-fold double categories by reviewing the standard universal property for Eilenberg-Mac Lane categories for monads and comonads, first given in~\cite{street:ftm}, and showing the particular form this universal property takes in the special case of comonads arising from an awfs. Then in \cref{Ch:Composition}, we give the (surprisingly difficult and technical) proofs that the results mentioned above about composition of coalgebras continue to hold at our higher level of generality.

In \cref{Ch:PushoutProduct} we show that a natural generalization of the universal property for arrow objects in a double category, defined in \cref{Sec:ArrowObjects}, to cyclic double multicategories in fact uniquely characterizes the pushout/pullback product. This allows us to abstract away the the pushout/pullback product, isolating precisely the properties which are necessary to make the theory of multivariable Quillen adjunctions work, and providing a conceptual explanation for the appearance of pushout/pullback products in the definitions of monoidal model category, simplicial model category, etc.

In \cref{Ch:DblMulti} we define a common generalization of the cyclic two-fold double categories of \cref{Ch:Cyclic} and the cyclic double multicategories of~\cite{cgr:mates}, which we call a cyclic two-fold double multicategory. We give a definition of multivariable morphisms of bimonads, showing that this definition is stable under the cyclic action. We then generalize our construction of a cyclic two-fold double category of functorial factorizations from \cref{Ch:FuncFact} to a cyclic two-fold double multicategory, and show that multivariable morphisms of bimonads recover the definition of multivariable adjunction of awfs given in~\cite{riehl:nwfs-monoidal}. The fact that these multivariable morphisms are stable under the cyclic action generalizes the classical fact that if a functor which is part of a 2-variable adjunction preserves the left classes, in that the pushout product of morphisms in the left classes is in the left class, then each of the two adjoints satisfy similar properties involving a mix of left and right classes.