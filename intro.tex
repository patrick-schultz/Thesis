% -*- root: thesis.tex -*-

\chapter{Introduction}

The theory of model categories has a long history, and has proven itself indespensible to several recent advances in mathematics, such as higher category theory, so-called spectral algebraic geometry, even finding applications in computer science and the foundations of mathematics with homotopy type theory.

In the modern treatment, a model category is defined to consist of two \emph{weak factorization systems} on a category $\cat{C}$ (e.g.~\cite{mp:more-concise}). A weak factorization system is a structure which consists of two classes of morphisms of $\cat{C}$, call them $\mathcal{L}$ and $\mathcal{R}$, such that solutions to certain lifting problems involving one morphism from each class always exist, plus an axiom that every morphism of $\cat{C}$ factors as a morphism from $\mathcal{L}$ followed by a morphism from $\mathcal{R}$. In the past 20 or so years, most authors have added the requirement that this factorization can be chosen in a natural/functorial way.

Taking this one step further, in \cite{gt:nwfs} the category theorists Marco Grandis and Walter Tholen proposed a strengthening of weak factorization systems which they called \emph{natural} weak factorization systems, today most often referred to as \emph{algebraic} weak factorization systems, or awfs for short. An awfs strengthens the structure in a way which provides a canonical \emph{choice} of solution to every lifting problem, in such a way that these choices are coherent or natural in a precise sense.

It at first seems as though this extra structure is \emph{too} strict, and that examples would be hard to find. But in \cite{garner:nwfs} and \cite{garner:soa}, the category theorist Richard Garner provided a modification of Quillen's small object argument which generates algebraic weak factorization systems, and which furthermore has much nicer convergence properties that Quillen's original construction, and often generates a smaller and easier to understand factorization. Best of all, Garner's small object argument operates under almost identical assumptions as Quillen's, so that in practice any cofibrantly generated weak factorization system can be strengthened to an algebraic one.

In her Ph.D. thesis, \cite{riehl:nwfs-model} and \cite{riehl:nwfs-monoidal}, Emily Riehl began the project of developing a full-fledged theory of \emph{algebraic model structures}, built out of two awfs analgously to an ordinary model structure, in the second part extending this to a theory of multivariable Quillen adjunctions and monoidal algebraic model structures. Since then, she and her collaborators have continued to develop and find applications of this theory, e.g.~\cite{cgr:mates}, \cite{br:funct-facts}, and \cite{bmr:six}.

One appealing aspect of the theory of algebraic weak factorization systems, and by extension algebraic model categories, is that it is possible to express the definition entirely in terms of functors and natural transformations, opening the door to generalizing the theory by carrying out the same construction in other contexts with analogues of functors and natural transformations. The simplest and most familiar such context is a 2-category, but we have found that the most natural setting for the general theory of awfs---including the appropiate generalization of Quillen adjunctions---is a double category, a less familiar structure which has been attracting more attention in recent years. %TODO: Give references

In this paper, we aim to build a general setting in which the structure of an algebraic weak factorization system makes sense, a structure we call a \emph{cyclic two-fold double category}, and then to translate some of the most important results from the existing theory to this larger context. We well then give an extension of this structure, called a \emph{cyclic two-fold double multi-category}, a modification of the cyclic double multi-categories of \cite{cgr:mates}, in which multivariable morphisms of awfs also make sense. Finally, we will give an example application of the increased generality by showing that the theory of enriched categories has the necessary structure needed to form a special case of our theory. We expect that other examples of algebraic model structures on structures other than plain categories will be found useful, and hope that this general framework will help to quickly translate the existing theory to new settings.