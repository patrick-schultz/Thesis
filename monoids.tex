% -*- root: thesis.tex -*-

\chapter{Monoids}

There are several possible ways to define (co)monoids/(co)monads in a double category, but there is one way in particular which interacts nicely with the 2-fold double category structure.

\begin{definition}
	A \emph{monoid} in a 2-fold double category $\mathbb{D}$ is a monoid in $\mathbb{D}_{\otimes}$. Specifically, it is a horizontal 1-cell $\begin{tikzcd}[baseline,column sep=2.5ex] X\colon C \rar[tick]& C \end{tikzcd}$, together with unit and multiplication 2-cells
	\[\begin{tikzcd}
		C \rar[tick][domA]{I_C} \dar[equal]
			& C \dar[equal] \\
		C \rar[tick][codA,swap]{X} 
			& C
		\twocellA{\eta}
	\end{tikzcd} \qquad
	\begin{tikzcd}
		C \rar[tick][domA]{X\otimes X} \dar[equal]
			& C \dar[equal] \\
		C \rar[tick][codA,swap]{X} 
			& C
		\twocellA{\mu} 
	\end{tikzcd}
	\]
	satisfying the usual unit and associativity conditions.
\end{definition}

\begin{definition}
	Given two monoids $(C,X,\eta,\mu)$ and $(D,Y,\eta',\mu')$ in $\mathbb{D}$, a morphism of monoids consists of a vertical morphism $f\colon C\to D$ and a 2-cell
	\[
	\begin{tikzcd}
		C \rar[tick][domA]{X} \dar[swap]{f}  
			& C \dar{f} \\
		D \rar[tick][codA,swap]{Y} 
			& D
		\twocellA{\phi}
	\end{tikzcd}
	\]
	which preserves the unit and multiplication, in that
	\begin{equation}\label{Eq:MonoidMorphismUnit}
	\begin{tikzcd}
		C \rar[tick][domA]{I_C} \dar[equal] 
			& C \dar[equal] \\
		C \rar[tick][codA,domB,swap]{X} \dar[swap]{f} 
			& C \dar{f} \\
		D \rar[tick][codB,swap]{Y}
			& D
		\twocellA{\eta}
		\twocellB{\phi}
	\end{tikzcd}
	=
	\begin{tikzcd}
		C \rar[tick][domA]{I_C} \dar[swap]{f} 
			& C \dar{f} \\
		D \rar[tick][codA,domB,swap]{I_D} \dar[equal] 
			& D \dar[equal] \\
		D \rar[tick][codB,swap]{Y}
			& D
		\twocellA{I_f}
		\twocellB{\eta'}
	\end{tikzcd}
	\end{equation}
	and
	\begin{equation}\label{Eq:MonoidMorphismMult}
	\begin{tikzcd}
		C \rar[tick][domA]{X\otimes X} \dar[equal] 
			& C \dar[equal] \\
		C \rar[tick][codA,domB,swap]{X} \dar[swap]{f} 
			& C \dar{f} \\
		D \rar[tick][codB,swap]{Y}
			& D
		\twocellA{\mu}
		\twocellB{\phi}
	\end{tikzcd}
	=
	\begin{tikzcd}
		C \rar[tick][domA]{X\otimes X} \dar[swap]{f} 
			& C \dar{f} \\
		D \rar[tick][codA,domB,swap]{Y\otimes Y} \dar[equal] 
			& D \dar[equal] \\
		D \rar[tick][codB,swap]{Y}
			& D.
		\twocellA{\phi\otimes\phi}
		\twocellB{\mu'}
	\end{tikzcd}
	\end{equation}
\end{definition}

\begin{definition}
	A comonoid in a 2-fold double category $\mathbb{D}$ is a comonoid in $\mathbb{D}_{\odot}$. Specifically, it is a horizontal 1-cell $\begin{tikzcd}[baseline,column sep=2.5ex] X\colon C \rar[tick]& C \end{tikzcd}$, together with counit and comultiplication 2-cells
	\[\begin{tikzcd}
		C \rar[tick][domA]{X} \dar[equal]  
			& C \dar[equal] \\
		C \rar[tick][codA,swap]{\perp_C} 
			& C
		\twocellA{\epsilon}
	\end{tikzcd} \qquad
	\begin{tikzcd}
		C \rar[tick][domA]{X} \dar[equal] 
			& C \dar[equal] \\
		C \rar[tick][codA,swap]{X\odot X} 
			& C
		\twocellA{\delta} 
	\end{tikzcd}
	\]
	satisfying the usual unit and associativity conditions.
\end{definition}

\begin{definition}
	Given two comonoids $(C,X,\epsilon,\delta)$ and $(D,Y,\epsilon',\delta')$ in $\mathbb{D}$, a morphism of comonoids consists of a vertical morphism $f\colon C\to D$ and a 2-cell
	\[
	\begin{tikzcd}
		C \rar[tick][domA]{X} \dar[swap]{f}  
			& C \dar{f} \\
		D \rar[tick][codA,swap]{Y} 
			& D
		\twocellA{\phi}
	\end{tikzcd}
	\]
	which preserves the counit and comultiplication, in that
	\begin{equation}\label{Eq:ComonoidMorphismCounit}
	\begin{tikzcd}
		C \rar[tick][domA]{X} \dar[equal] 
			& C \dar[equal] \\
		C \rar[tick][codA,domB,swap]{\perp_C} \dar[swap]{f} 
			& C \dar{f} \\
		D \rar[tick][codB,swap]{\perp_D}
			& D
		\twocellA{\epsilon}
		\twocellB{\perp_f}
	\end{tikzcd}
	=
	\begin{tikzcd}
		C \rar[tick][domA]{X} \dar[swap]{f} 
			& C \dar{f} \\
		D \rar[tick][codA,domB,swap]{Y} \dar[equal] 
			& D \dar[equal] \\
		D \rar[tick][codB,swap]{\perp_D}
			& D
		\twocellA{\phi}
		\twocellB{\epsilon'}
	\end{tikzcd}
	\end{equation}
	and
	\begin{equation}\label{Eq:ComonoidMorphismComult}
	\begin{tikzcd}
		C \rar[tick][domA]{X} \dar[equal] 
			& C \dar[equal] \\
		C \rar[tick][codA,domB,swap]{X\odot X} \dar[swap]{f} 
			& C \dar{f} \\
		D \rar[tick][codB,swap]{Y\odot Y}
			& D
		\twocellA{\delta}
		\twocellB{\phi\odot\phi}
	\end{tikzcd}
	=
	\begin{tikzcd}
		C \rar[tick][domA]{X} \dar[swap]{f} 
			& C \dar{f} \\
		D \rar[tick][codA,domB,swap]{Y} \dar[equal] 
			& D \dar[equal] \\
		D \rar[tick][codB,swap]{Y\odot Y}
			& D.
		\twocellA{\phi}
		\twocellB{\delta'}
	\end{tikzcd}
	\end{equation}
\end{definition}

The 2-fold double category structure on $\mathbb{D}$ allows us to form double categories $\mathbb{M}\mathrm{on}(\mathbb{D})$ and $\mathbb{C}\mathrm{omon}(\mathbb{D})$ of (co)monoids in $\mathbb{D}$, in which the objects and vertical morphisms are the same as in $\mathbb{D}$, and the horizontal 1-cells and 2-cells are (co)monoids and (co)monoid morphisms in $\mathbb{D}$. The only thing we still have to provide is a horizontal composition.

Given two monoids $(C,X,\eta,\mu)$ and $(C,Y,\eta',\mu')$ in $\mathbb{D}$, thought of as horizontal 1-cells $\begin{tikzcd}[baseline,column sep=2.5ex] C \rar[tick]& C \end{tikzcd}$ in $\mathbb{M}\mathrm{on}(\mathbb{D})$, the horizontal composition 
\[
\begin{tikzcd}[column sep=large]
	C \rar[tick]{(X,\eta,\mu)} & C \rar[tick]{(Y,\eta',\mu')} & C
\end{tikzcd}
\]
is the monoid with underlying horizontal 1-cell $X\odot Y$ and unit and multiplication 2-cells
\[
\begin{tikzcd}[column sep=7ex]
	C \rar[tick][domA]{I_C} \dar[equal] 
		& C \dar[equal] \\
	C \rar[tick][codA,domB]{I_C\odot I_C} 
			\dar[equal] 
		& C \dar[equal] \\
	C \rar[tick][codB,swap]{X\odot Y}
		& C
	\twocellA{c}
	\twocellB{\eta\odot\eta'}
\end{tikzcd}
\qquad
\begin{tikzcd}[column sep=14ex]
	C \rar[tick][domA]{(X\odot Y)\otimes(X\odot Y)} 
			\dar[equal] 
		& C \dar[equal] \\
	C \rar[tick][codA,domB]{(X\otimes X)\odot(Y\otimes Y)} 
			\dar[equal] 
		& C \dar[equal] \\
	C \rar[tick][codB,swap]{X\odot Y}
		& C.
	\twocellA{z}
	\twocellB{\mu\odot\mu'}
\end{tikzcd}
\]
Similarly, the horizontal composition of two 2-cells in $\mathbb{M}\mathrm{on}(\mathbb{D})$ is the $\odot$ product of the underlying 2-cells in $\mathbb{D}$. The fact that this commutes with the unit and multiplication defined above follows from the naturality of $c$ and $z$.

In this same way, we can define the horizontal composition of two 1-cells $(X,\epsilon,\delta)$ and $(Y,\epsilon',\delta')$ in $\mathbb{C}\mathrm{omon}(\mathbb{D})$ to be a comonoid with underlying horizontal 1-cell $X\otimes Y$.

This allows us to define (ordinary) categories $\mathrm{Mon}(\mathbb{C}\mathrm{omon}(\mathbb{D}))$ and $\mathrm{Comon}(\mathbb{M}\mathrm{on}(\mathbb{D}))$. Furthermore, these two categories are equivalent, leading to the next definition.

\begin{definition}
	A \emph{bimonoid} in a 2-fold double category $\mathbb{D}$ is a monoid in $\mathbb{C}\mathrm{omon}(\mathbb{D})$, or equivalently a comonoid in $\mathbb{M}\mathrm{on}(\mathbb{D})$. We can define a category of bimonoids in $\mathbb{D}$ as
	\[
		\mathrm{Bimon}(\mathbb{D}) := \mathrm{Mon}(\mathbb{C}\mathrm{omon}(\mathbb{D})) \simeq \mathrm{Comon}(\mathbb{M}\mathrm{on}(\mathbb{D}))
	\]
\end{definition}

Concretely, a bimonoid in $\mathbb{D}$ is a tuple $(X,\eta,\mu,\epsilon,\delta)$ where $X$ is a horizontal 1-cell, $(X,\eta,\mu)$ is a monoid and $(X,\epsilon,\delta)$ is a comonoid as above, such that four equations hold:
\begin{equation}\label{Eq:Bimonoid}
\begin{gathered}
\begin{tikzcd}[ampersand replacement=\&]
	C \rar[tick][domA]{I_C} \dar[equal] 
		\& C \dar[equal] \\
	C \rar[tick][codA,domB]{X} \dar[equal] 
		\& C \dar[equal] \\
	C \rar[tick][codB,swap]{X\odot X} \& C
	\twocellA{\eta}
	\twocellB{\delta}
\end{tikzcd}
=
\begin{tikzcd}[ampersand replacement=\&]
	C \rar[tick][domA]{I_C} \dar[equal] 
		\& C \dar[equal] \\
	C \rar[tick][codA,domB]{I_C\odot I_C} \dar[equal] 
		\& C \dar[equal] \\
	C \rar[tick][codB,swap]{X\odot X} \& C
	\twocellA{c}
	\twocellB{\eta\odot\eta}
\end{tikzcd}
\qquad
\begin{tikzcd}[ampersand replacement=\&]
	C \rar[tick][domA]{X\otimes X} \dar[equal] 
		\& C \dar[equal] \\
	C \rar[tick][codA,domB]{X} \dar[equal] 
		\& C \dar[equal] \\
	C \rar[tick][codB,swap]{\perp_C} \& C
	\twocellA{\mu}
	\twocellB{\epsilon}
\end{tikzcd}
=
\begin{tikzcd}[ampersand replacement=\&]
	C \rar[tick][domA]{X\otimes X} \dar[equal] 
		\& C \dar[equal] \\
	C \rar[tick][codA,domB]{\perp_C\otimes\perp_C} \dar[equal] 
		\& C \dar[equal] \\
	C \rar[tick][codB,swap]{\perp_C} \& C
	\twocellA{\epsilon\otimes\epsilon}
	\twocellB{m}
\end{tikzcd}
\\
\begin{tikzcd}[ampersand replacement=\&]
	C \rar[tick][domA]{I_C} \dar[equal] 
		\& C \dar[equal] \\
	C \rar[tick][codA,domB]{X} \dar[equal] 
		\& C \dar[equal] \\
	C \rar[tick][codB,swap]{\perp_C} \& C
	\twocellA{\eta}
	\twocellB{\epsilon}
\end{tikzcd}
=
\begin{tikzcd}[ampersand replacement=\&]
	C \rar[tick][domA]{I_C} \dar[equal] 
		\& C \dar[equal] \\
	C \rar[tick][codA,swap]{\perp_C} \& C
	\twocellA{j}
\end{tikzcd}
\qquad
\begin{tikzcd}[ampersand replacement=\&,column sep=14ex]
	C \rar[tick][domA]{X\otimes X} \dar[equal] 
		\& C \dar[equal] \\
	C \rar[tick][codA,domB]{(X\odot X)\otimes(X\odot X)} \dar[equal] 
		\& C \dar[equal] \\
	C \rar[tick][codB,domC]{(X\otimes X)\odot(X\otimes X)} \dar[equal] 
		\& C \dar[equal] \\
	C \rar[tick][codC,swap]{X\odot X} \& C
	\twocellA{\delta\otimes\delta}
	\twocellB{z}
	\twocellC{\mu\odot\mu}
\end{tikzcd}
=
\begin{tikzcd}[ampersand replacement=\&]
	C \rar[tick][domA]{X\otimes X} \dar[equal]
		\& C \dar[equal] \\
	C \rar[tick][codA,domB]{X} \dar[equal] 
		\& C \dar[equal] \\
	C \rar[tick][codB,swap]{X\odot X} \& C
	\twocellA{\mu}
	\twocellB{\delta}
\end{tikzcd}
\end{gathered}
\end{equation}
A bimonoid morphism is simply a 2-cell which is simultaineously a monoid morphism and a comonoid morphism.